\documentclass[a4paper]{article}

\usepackage[utf8]{inputenc}

\usepackage{natbib}
\usepackage{tgpagella}
\usepackage[T1]{fontenc}

\usepackage{amstext}
\usepackage{booktabs}
\usepackage{hyperref}
\usepackage{linguex}
\usepackage{mathtext}
\usepackage{nicematrix}
\usepackage{relsize}
\usepackage{tikz-dependency}
\usepackage[backgroundcolor=blue!20!white]{todonotes}

\title{Superframes Manual}
\author{Kilian Evang}
\date{Last updated: \today}

% frame and role names
\newcommand{\fr}[1]{\textsf{#1}}
\newcommand{\frs}[1]{\mbox{\textsf{#1}}} % frame suffixes start with hyphen, prevent line break
\newcommand{\rl}[1]{\textsf{#1}}

% make a row bold in a NiceTabular
\newcommand{\boro}{\RowStyle{\bfseries}}

\begin{document}

% less white space in examples
\setlength{\Exindent}{0pt}
\setlength{\Exlabelsep}{0pt}
\setlength{\SubExleftmargin}{6pt}
\setlength{\SubSubExleftmargin}{6pt}

\maketitle

%\begin{abstract}
%\end{abstract}

\tableofcontents

\section{Introduction}

\begin{table}
    \resizebox{\textwidth}{!}{
        \begin{NiceTabular}{lllllll}
            \toprule
            Superframe & Roles & & & & & Sec. \\
            \midrule
            \boro\fr{SCENE} & \rl{initial-scene} & \rl{participant} & \rl{scene} & \rl{transitory-scene} & \rl{target-scene} & \ref{sec:SCENE} \\
            \boro\fr{IDENTIFICATION} & & \rl{identified} & \rl{identifier} & & & \ref{sec:IDENTIFICATION} \\
            \boro\fr{RANK} & & \rl{has-rank} & \rl{rank} & & & \ref{sec:RANK} \\
            \boro\fr{CLASS} & \rl{initial-class} & \rl{has-class} & \rl{class} & & \rl{target-class} & \ref{sec:CLASS} \\
            \boro\fr{EXISTENCE} & & & \rl{exists} & & & \ref{sec:EXISTENCE} \\
            \fr{TRANSFORMATION-CREATION} & & \rl{material} & & & \rl{created} & \ref{sec:TRANSFORMATION-CREATION} \\
            \fr{REPRODUCTION} & & \rl{original} & & & \rl{copy} & \ref{sec:REPRODUCTION} \\
            \boro\fr{QUALITY} & & \rl{has-quality} & \rl{quality} & & & \ref{sec:QUALITY} \\
            \boro\fr{STATE} & \rl{initial-state} & \rl{has-state} & \rl{state} & & \rl{target-state} & \ref{sec:STATE} \\
            \fr{DESTRUCTION} & & \rl{destroyed} & & & & \ref{sec:DESTRUCTION} \\
            %\fr{MENTAL-STATE} & & \rl{has-mental-state} & \rl{mental-state} & & & \ref{sec:MENTAL-STATE} \\
            \boro\fr{EXPERIENCE} & & \rl{experiencer} & \rl{experienced} & & & \ref{sec:EXPERIENCE} \\
            \boro\fr{ACTIVITY} & & \rl{is-active} & \rl{activity} & & & \ref{sec:ACTIVITY} \\
            \boro\fr{MODE} & & \rl{has-mode} & \rl{mode} & & & \ref{sec:MODE} \\
            \midrule
            \boro\fr{ACCOMPANIMENT} & & \rl{accompanied} & \rl{accompanier} & & & \ref{sec:ACCOMPANIMENT} \\
            \fr{DEPICTIVE} & & \rl{has-depictive} & \rl{depictive} & & & \ref{sec:DEPICTIVE} \\
            \boro\fr{ATTRIBUTE} & & \rl{has-attribute} & \rl{attribute} & & & \ref{sec:ATTRIBUTE} \\
            \boro\fr{ASSET} & & \rl{has-asset} & \rl{asset} & & & \ref{sec:ASSET} \\
            \boro\fr{COMPARISON} & & \rl{compared} & \rl{reference} & & & \ref{sec:COMPARISON} \\
            \fr{CONCESSION} & & \rl{assertion} & \rl{conceded} & & & \ref{sec:CONCESSION} \\
            \boro\fr{EXPLANATION} & & \rl{explained} & \rl{explanation} & & & \ref{sec:EXPLANATION} \\
            \boro\fr{LOCATION} & \rl{initial-location} & \rl{has-location} & \rl{location} & \rl{transitory-location} & \rl{target-location} & \ref{sec:LOCATION} \\
            \fr{WRAPPING-WEARING} & & \rl{worn} & \rl{wearer} & & & \ref{sec:WRAPPING-WEARING} \\
            \fr{ADORNMENT-TARNISHMENT} & \rl{initial-surface} & \rl{ornament} & \rl{surface} & & \rl{target-surface} & \ref{sec:ADORNMENT-TARNISHMENT} \\
            \fr{HITTING} & & \rl{hitting} & \rl{hit} & & & \ref{sec:HITTING} \\
            \fr{INGESTION} & & \rl{ingested} & & \rl{transitory-location} & \rl{ingester} & \ref{sec:INGESTION} \\
            \fr{EXCRETION} & \rl{excreter} & \rl{excreted} & & \rl{transitory-location} & & \ref{sec:EXCRETION} \\
            \fr{UNANCHORED-MOTION} & & \rl{in-motion} & & \rl{transitory-location} & & \ref{sec:UNANCHORED-MOTION} \\
            \boro\fr{MEANS} & & \rl{has-means} & \rl{means} & & & \ref{sec:MEANS} \\
            \boro\fr{MESSAGE} & & \rl{topic} & \rl{content} & & & \ref{sec:MESSAGE} \\
            \boro\fr{PART-WHOLE} & \rl{initial-whole} & \rl{part} & \rl{whole} & & \rl{target-whole} & \ref{sec:PART-WHOLE} \\
            \boro\fr{POSSESSION} & \rl{initial-possessor} & \rl{possessed} & \rl{possessor} & & \rl{target-possessor} & \ref{sec:POSSESSION} \\
            \boro\fr{QUANTITY} & & \rl{has-quantity} & \rl{quantity} & & & \ref{sec:QUANTITY} \\
            \boro\fr{SENDING} & & \rl{sent} & \rl{sender} & & & \ref{sec:SENDING} \\
            \boro\fr{SEQUENCE} & & \rl{follows} & \rl{followed} & & & \ref{sec:SEQUENCE} \\
            \fr{CAUSATION} & & \rl{result} & \rl{causer} & & & \ref{sec:CAUSATION} \\
            \fr{REACTION} & & \rl{reaction} & \rl{trigger} & & & \ref{sec:REACTION} \\
            \fr{RESULTATIVE} & & \rl{has-resultative} & \rl{resultative} & & & \ref{sec:RESULTATIVE} \\
            \fr{CONDITION} & & \rl{has-condition} & \rl{condition} & & & \ref{sec:CONDITION} \\
            \fr{EXCEPTION} & & \rl{has-exception} & \rl{exception} & & & \ref{sec:EXCEPTION} \\
            \boro\fr{SOCIAL-RELATION} & \rl{initial-social-relation} & \rl{has-social-relation} & \rl{social-relation} & & \rl{target-social-relation} & \ref{sec:SOCIAL-RELATION} \\
            \boro\fr{TIME} & & \rl{has-time} & \rl{time} & & & \ref{sec:TIME} \\
            \midrule
            \boro\fr{NONCOMP} & & \rl{has-noncomp} & \rl{noncomp} & & & \ref{sec:NONCOMP} \\
            \bottomrule
        \end{NiceTabular}
    }
    \caption{The superframes and their roles. Top-level superframes are shown in bold. Underneath, some superframes have special cases with partly renamed roles, included to make them more intuitive to apply.}
    \label{tab:superframes}
\end{table}

Superframes is an annotation scheme for semantic roles. Like other such
schemes, it is essentially about pinning down, in a machine-readable form,
``who did what to whom''. It is different from other such schemes, such as
FrameNet \citep{baker-etal-1998-berkeley}, VerbNet
\citep{kipper-schuler-2005-verbnet}, PropBank
\citep{palmer-etal-2005-proposition}, VerbAtlas
\citep{di-fabio-etal-2019-verbatlas}, or WiSER \citep{feng-etal-2022-widely} in
a number of ways. It aims to avoid a number of practical problems in annotating
with those schemes. Here's how Superframes annotation works, in a nutshell:

\begin{enumerate}
    \item Every content word (verb, noun, pronoun, adjective, or adverb) is a
        \emph{predicate}. Every predicate evokes one of a few dozen
        \emph{superframes}, which determines its coarse semantic class and the
        possible role labels for its core arguments.
    \item The syntactic \emph{dependents} of a predicate can be
        \emph{core arguments}, in which case they get one of the role labels
        defined by the superframe of the predicate, or \emph{external
        arguments} or \emph{modifiers}, in which case they are treated as
        evoking their own frame in which the predicate serves as a core argument.
    \item There are only two main core role labels per superframe.
    \item For predicates denoting change (or lack thereof) over time,
        some superframes have \emph{aspectual variants} with role variants that
        allow to distinguish participants before, during, and after an event.
        This avoids having \texttt{Source} and \texttt{Target} as roles in
        their own right, which indicate the time sequence but suppress
        information about the nature of the relation that is changing.
    \item Similarly, Superframes do not have the \texttt{Agent} role, which is
        often in conflict with roles indicating more specifically the agent's
        relation to other participants.
    \item Doubt, ambiguity, and figurativity are systematically treated. If there
        is not one clear solution, the solution is to give two or more
        alternative labels.
\end{enumerate}

Table~\ref{tab:superframes} shows the superframes and their roles.

\subsection{Core Arguments}

The most prototypical predicate is a verb, and the simplest case is a verb with only one argument. It can for example denote a state or an activity:

\ex.\dep{Kim_has-state is *sleeping#STATE*}

\ex.\dep{Kim_is-active is *partying#ACTIVITY*}

With two core arguments, a verb denotes a relation that holds between them:

\ex.\dep{Kim_possessor *owns#POSSESSION* a house_possession}

\ex.\dep{The house_possession *belongs#POSSESSION* to Kim_possessor}

\ex.\dep{Kim_topic *seems#MESSAGE* happy_content}

\subsection{Aspect and Mode}
\label{sec:aspect-mode}

Rather than a static relationship between two entities, many verbs (and other
predicates) denote a change (or absence of change) in such a relationship. We
sort such predicates into a few coarse aspectual classes. For example,
initiation (\frs{-INIT}) means a state is begun or worked towards, deinitiation
(\frs{-DEINIT}) means a state is ended, completed, or its end is worked
towards, change (\frs{-CHANGE}) combines both, where one state is replaced by
another, and continuation (\frs{-CONTINUATION}) means a state persists or is
even intensified.  Accordingly, roles with prefix \rl{target-} mark
participants at or beyond the end of the event, \rl{initial-} marks
participants at the beginning of the event, and \rl{transitory-} marks
participants at some point during the event.

\ex.\dep{Kim_target-possessor *got#POSSESSION-INIT* the house_possession}

\ex.\dep{Kim_initial-possessor *lost#POSSESSION-DEINIT* the house_posesssion}

\ex.\dep{Kim_initial-possessor *sold#POSSESSION-CHANGE* the house_possession to Sandy_target-possessor}

\ex.\dep{Kim_initial-possessor *kept#POSSESSION-CONTINUATION* the house_possession}

\ex.\dep{Kim_has-location *went#LOCATION-CHANGE* from Chicago_initial-location via Pittsburgh_transitory-location to Boston_target-location}

\ex.\label{ex:fall}\dep{The vase_has-location *fell#LOCATION-CHANGE* to the ground_target-location}

\ex.\dep{The vase_has-state *broke#STATE-CHANGE*}

\ex.\dep{Kim_has-social-relation *befriended#SOCIAL-RELATION-INIT* Sandy_target-social-relation}

\ex.\dep{Kim_has-social-relation *married#SOCIAL-RELATION-INIT* Sandy_target-social-relation}

\ex.\dep{Kim_has-social-relation *divorced#SOCIAL-RELATION-DEINIT* Sandy_initial-social-relation}

The \fr{SCENE} superframe is often evoked by ``light'' verbs that contribute an
aspectual or modal meaning. Thus, its aspectual variants are especially common.

\ex.\dep{The concert_target-scene *began#SCENE-INIT*}

\ex.\dep{The concert_initial-scene *continued#SCENE-CONTINUATION*}

\ex.\dep{The concert_initial-scene *finished#SCENE-DEINIT*}

\ex.\dep{The shouting_initial-scene *intensified#SCENE-CONTINUATION*}

\ex.\dep{The shouting_initial-scene *faded#SCENE-DEINIT*}

\ex.\dep{A coup_target-scene was *attempted#SCENE-INIT*}

\ex.\dep{Kim_participant *finished#SCENE-DEINIT* their work_initial-scene}

In addition, we use the modal suffixes \frs{-NECESSITY},  \frs{-POSSIBILITY}.
and \frs{-NEG}. They can combine with aspectual suffixes.

\ex.\dep{Change_scene is *necessary#SCENE-NECESSITY*}

\ex.\dep{Change_scene is *possible#SCENE-POSSIBILITY*}

\ex.\dep{Kim_initial-possessor *owes#POSSESSION-CHANGE-NECESSITY* Sandy_target-possessor money_possessed}

\ex.\dep{Swift action_x-causer *prevented#SCENE-INIT-NEG* an outbreak_target-scene}

\ex.\dep{Kim_participant *refrained#SCENE-INIT-NEG* from going_target-scene}

\ex.\dep{Kim_x-causer *prevented#SCENE-INIT-NEG* Sandy_participant from going_target-scene}

\ex.\dep{Kim_x-causer *saved#SCENE-INIT-NEG* Sandy_participant from the dragon_target-scene}

In the last example, \emph{dragon} is to be understood metonymically as a
scene in which Sandy would have been harmed by the dragon.

\subsection{Non-core Arguments}

Core arguments always get role labels from the superframe the predicate evokes.
But many verbs have more arguments. One common case is a subject that is
presented as the causer of the scene. For example, compare \ref{ex:throw} with
\ref{ex:fall}. The core scene is the same (same superframe, same arguments). We
now assume there is an additional \fr{CAUSATION} scene with \emph{Kim} as the
\rl{causer} and the core scene as the \rl{result}. We denote this by giving
\emph{Kim} the \rl{causer} role label, with an \rl{x-} prefix to mark it as a
non-core role.

\ex.\label{ex:throw}\dep{Kim_x-causer *threw#LOCATION-CHANGE* the vase_has-location to the ground_target-location}

\ex.\dep{Kim_x-causer *broke#STATE-CHANGE* the vase_has-state}

Two other common non-core arguments are the senders and recipients (experiencers) of messages.

\ex.\dep{Kim_x-sender *talked#MESSAGE* to Sandy_x-experiencer about Bali_topic}

Other non-core arguments are usually rather predicate-specific.

\ex.\dep{Kim_x-experiencer *searched#MESSAGE* the woods_x-location for Sandy_topic}

\ex.\dep{Kim_initial-possessor *sold#POSSESSION-CHANGE* Sandy_target-possessor the house_possession for a million dollars_x-asset}

\subsection{Modifiers}

Like non-core arguments, modifiers are assumed to evoke an additional frame,
and labeled with the role they fill in that frame, but with a prefix marking
them as modifiers: \rl{m-}.

\ex.\dep{Kim_excreter is *sweating#EXCRETION* profusely_m-quantity in the sauna_m-location}

\subsection{Nonverbal Predicates}

So far, we have only looked at verbal predicates. But of course, there are
other types of predicates. An ordinary noun like \emph{tree} evokes the
\fr{CLASS} frame, marking the entity it refers to as being a member of a class
(in this case: the class of trees). There are no arguments here because the
predicate itself doubles as a referent. However, the predicate can of course be
modified:

\ex.\dep{a *tree#CLASS* in the garden_m-location}

\ex.\dep{Kim_m-possessor 's *tree#CLASS*}

Event nouns evoke event frames and have arguments:

\ex.\dep{Kim_x-causer 's *breaking#STATE-CHANGE* of the vase_has-state}

Relational nouns evoke relational frames and have arguments:

\ex.\dep{Kim_has-social-relation 's *friend#SOCIAL-RELATION*}

Pronouns and names evoke the \fr{IDENTIFICATION} frame, meaning that they
identify their referent as some entity (via naming or anaphora resolution).

\ex.\dep{*Kim#IDENTIFICATION*}

\ex.\dep{*they#IDENTIFICATION*}

Predicate adjectives most typically denote states or qualities.

\ex.\dep{I_has-quality am *despicable#QUALITY*}

\ex.\dep{the dog_has-state is *tired#STATE*}

With attributive adjectives, the dependency relation is reversed, and the role label is changed accordingly.

\ex.\dep{despicable_m-quality *me#IDENTIFICATION*}

\ex.\dep{the tired_m-state *dog#CLASS*}

Similarly for adverbs denoting, e.g, manner (\rl{quality}) or extent (\rl{quantity}):

\ex.\dep{Kim_has-location *ran#Motion* fast_m-quality}

\ex.\dep{Kim_has-location *ran#Motion* far_m-quantity}

\subsection{Control Relations}
\label{sec:control}

%\todo[inline]{spell out strategies for consistent detection (xcomp, MESSAGE/SCENE frames, special cases...)}

Many constructions systematically introduce semantic predicate-dependent
dependencies that do not correspond to (surface) syntactic dependencies. In such cases, we add those dependency links.

\ex.\dep{Kim_has-location promised Sandy to *come#LOCATION-CHANGE*} (subject control)

\ex.\dep{Kim__x-causer used a hammer to **smash#STATE-CHANGE** the vase__has-state} (subject control)

\ex.\dep{Kim persuaded Sandy_has-location to *come#LOCATION-CHANGE*} (object control)

\ex.\dep{Kim_has-location seemed to *fly#UNANCHORED-MOTION*} (raising)

\ex.\dep{Kim_x-sender entered the room *singing#MESSAGE*} (depictive)

\ex.\dep{You're talking me_has-state *silly#STATE*} (resultative)

\ex.\dep{Kim_has-location has come to *stay#LOCATION-CONTINUATION*} (subjectless adverbial clause)

\ex.\dep{Kim_x-causer left after *trashing#STATE-CHANGE* the room_has-state} (subjectless adverbial clause)

\ex.\dep{Kim_topic is hard to *love#MESSAGE*} (\emph{tough} construction)

\ex.\dep{the song_topic I_x-experiencer *like#MESSAGE*} (relative clause)

\ex.\dep{the question_topic we_x-sender raised without *answering#MESSAGE*} (parasitic gap)

\subsection{Figurativity, Idiomaticity, and Uncertainty}

Difficulties in choosing frames often arise because a predicate literally evokes
one frame, but is used in a way that perhaps fits another frame equally well or
better. In such cases, annotate both the more literal frame and roles, followed
by the \texttt{>}\texttt{>} operator, followed by the more figurative frame and
roles.

\ex.\dep{A hush_in-motion>>scene *passed#UNANCHORED-MOTION>>SCENE* over the group_transitory-location>>participant}

\ex.\dep{Kim_x-sender>>participant *refused#MESSAGE>>SCENE* to eat_topic>>scene}

This mechanism can be used to indicate that a modification may not be fully compositional:

\ex.\dep{primeval_m-time>>m-noncomp *forest#CLASS*}

\ex.\dep{colored_m-quality>>m-noncomp *pencil#CLASS*}

\ex.\dep{to *lay#LOCATION-CHANGE>>MESSAGE-DEINIT* aside_target-location>>x-noncomp my drawings_has-location>>topic}

If you cannot choose between two frames for another reason, use \texttt{||} instead of \texttt{>}\texttt{>}.

\section{Superframes Reference}

\subsection{\fr{SCENE}}
\label{sec:SCENE}

A ``meta'' frame for predicates where the main frame is invoked by \rl{scene},
and the predicate adds some temporal, aspectual, modal, etc., meaning, or just
acts as a light verb. If there is a \rl{participant}, it is assigned a role by
\rl{scene}, which needs an extra dependency link. In the following examples, we
show the annotations for both the matrix predicate and the embedded predicate
in one graph.

\ex.\dep{The **concert#MESSAGE**_target-scene *began#SCENE-INIT*}

\ex.\dep{The **concert#MESSAGE**_initial-scene *continued#SCENE-CONTINUATION*}

\ex.\dep{The **concert#MESSAGE**_initial-scene *finished#SCENE-DEINIT*}

\ex.\dep{The **shouting#MESSAGE**_initial-scene *intensified#SCENE-CONTINUATION*}

\ex.\dep{The **shouting#MESSAGE**_initial-scene *faded#SCENE-DEINIT*}

\ex.\dep{A **coup#EXPERIENCE**_target-scene was *attempted#SCENE-INIT*}

\ex.\dep{Kim_participant__is-active *finished#SCENE-DEINIT* their **work#ACTIVITY**_initial-scene}

\ex.\dep{Swift action_x-causer *prevented#SCENE-INIT-NEG* an **outbreak#SCENE-INIT**_target-scene of ***measles#EXPERIENCE***__target-scene}

\ex.\dep{Kim_participant__has-location *refrained#SCENE-INIT-NEG* from **going#LOCATION-CHANGE**_target-scene}

\ex.\dep{Kim_x-causer *prevented#SCENE-INIT-NEG* Sandy_participant__has-location from **going#LOCATION-CHANGE**_target-scene}

\ex.\dep{Kim_x-causer *saved#SCENE-INIT-NEG* Sandy_participant__x-experiencer from the **dragon#CLASS**_target-scene}

\ex.\dep{Kim_participant__is-active *plays#SCENE* **tennis#ACTIVITY**_scene}

\ex.\dep{Kim_participant__participant___is-active *used#SCENE* to **play#SCENE**_scene ***tennis#ACTIVITY***__scene}

\ex.\dep{Kim_participant_x-causer *gave#SCENE* Sandy_participant__hit a **kick#HITTING**_scene}

The modifier relation \rl{m-scene} is used when a syntactic dependeny points
from an argument to a predicate, as, e.g., with relative clauses or evaluatives.

\ex.\dep{the *clown#CLASS*__topic I__x-experiencer **saw#MESSAGE**_m-scene smiled}

\ex.\dep{**Fortunately#EXPERIENCE**_m-scene for Sandy__experiencer , Kim_has-location is *here#LOCATION*__experienced}

\subsection{\fr{IDENTIFICATION}}
\label{sec:IDENTIFICATION}

\rl{identifier} identifies \rl{identified}.

Evoked by pronouns, names, and other identifiers, as well as predicates
denoting naming relationships.

\ex.\dep{*I#IDENTIFICATION* saw a picture}

\ex.\dep{I can distinguish *China#IDENTIFICATION* from Arizona}

\ex.\dep{a book_identified *called#IDENTIFICATION* True Stories_identifier from Nature}

\ex.\dep{This_identified is *Kim#IDENTIFICATION*}

Predicates that evoke other frames can still use \rl{x-identified} to mark the
copula subject as identified:

\ex.\dep{This_x-identified is the *book#MESSAGE* I like_m-scene}

\subsection{\fr{RANK}}
\label{sec:RANK}

\rl{rank} indicates the order that \rl{has-rank} has in some sequence.

\ex.\dep{*Chapter#MESSAGE* 1_m-rank}

\ex.\dep{my_m-sender first_m-rank *drawing#MESSAGE*}

\subsection{\fr{CLASS}}
\label{sec:CLASS}

\rl{class} indicates the class of entity that \rl{has-class} represents.

Most prototypically evoked by common nouns with no arguments.

\ex.\dep{swallowing an animal#CLASS}

\subsection{\fr{EXISTENCE}}
\label{sec:EXISTENCE}

\rl{exists} exists. Use this only for non-scene entities; for scenes, use the \fr{SCENE} frame.

\ex.\dep{I_exists *exist#EXISTENCE*}

\ex.\dep{There_x-noncomp *is#EXISTENCE* a hill_exists}

\ex.\dep{There_x-noncomp *is#SCENE* a hubbub_scene}

\subsection{\fr{TRANSFORMATION-CREATION}}
\label{sec:TRANSFORMATION-CREATION}

Special case of \fr{EXISTENCE-INIT} where \rl{created} (aka \rl{target-exists})
is newly created from \rl{material}, or \rl{material} is transformed to become
\rl{created}.

\ex.\dep{I_x-causer succeeded in *making#TRANSFORMATION-CREATION* my first drawing_created}

\ex.\dep{Kim_x-causer *built#TRANSFORMATION-CREATION* a castle_created out of sand_material}

\ex.\dep{Kim_x-causer *turned#TRANSFORMATION-CREATION* straw_material into gold_created}

\subsection{\fr{REPRODUCTION}}
\label{sec:REPRODUCTION}

Special case of \fr{EXISTENCE-INIT} where \rl{original} continues to exist, and
a (modified) \rl{copy} (aka \rl{target-exists}) comes into existence.

\ex.\dep{Here is a *copy#REPRODUCTION* of the drawing_original}

\ex.\dep{This_copy is a *translation#REPRODUCTION* of the pamphlet_original into English_x-quality}

\subsection{\fr{QUALITY}}
\label{sec:QUALITY}

\rl{quality} indicates a (permanent) quality/property/manner of \rl{has-quality}.

\ex.\dep{a magnificent_m-quality *picture#MESSAGE*}

\ex.\dep{I_x-experiencer *pondered#MESSAGE* deeply_m-quality over the adventures_topic of the jungle}

\ex.\dep{a skilled_m-quality *surgeon#CLASS*}

\subsection{\fr{STATE}}
\label{sec:STATE}

\rl{state} indicates a (temporary) state of \rl{has-state}.

\ex.\dep{when I_has-state was six years_x-quantity *old#STATE*}

\ex.\dep{Boa constrictors swallow their prey_has-state *whole#STATE*}

\ex.\dep{they_has-state *sleep#STATE*}

\ex.\dep{they_x-causer swallow their prey whole without *chewing#STATE-CHANGE* it_has-state}

\ex.\dep{the six months that they_x-causer need for *digestion#STATE-CHANGE*}

\ex.\dep{And that_x-causer hasn't much *improved#STATE-CHANGE* my opinion_has-state of them}

\subsection{\fr{DESTRUCTION}}
\label{sec:DESTRUCTION}

Special case of \fr{STATE-CHANGE} where \rl{destroyed} (aka \rl{has-state}) goes out of existence.

\ex.\dep{Sam_destroyed 's *death#DESTRUCTION*}

\ex.\dep{Sam_x-causer 's *destruction#DESTRUCTION* of the city_destroyed}

\subsection{\fr{EXPERIENCE}}
\label{sec:EXPERIENCE}

\rl{experienced} indicates an experience that \rl{experiencer} undergoes.

Used for dynamic scenes where the \rl{experiencer} is not necessarily active,
and that cannot well be framed as a state change. Also used for sensory and
mental perception, addressees in communication, beneficiaries, and for ``bystander'' roles.

\ex.\dep{Kim_experiencer 's *adventures#EXPERIENCE* in the jungle_m-location}

\ex.\dep{Kim_x-causer *attacked#EXPERIENCE* Sandy_experiencer}

\ex.\dep{I_x-experiencer *saw#MESSAGE* a magnificent picture_topic}

\ex.\dep{I_x-experiencer *pondered#MESSAGE* deeply_m-quality}

\ex.\dep{Kim_x-sender *talked#MESSAGE* to Sandy_x-experiencer}

\ex.\dep{Kim_participant *did#SCENE* something_scene nice for Sandy_m-experiencer}

\ex.\dep{Kim_x-experiencer cooked a meal only to *have#SCENE* Sandy_participant spurn_scene it}

\ex.\dep{Kim_experiencer *managed#EXPERIENCE* with dealing_experienced the cards}

\ex.\dep{Die Piroggen_participant waren Maria_x-experiencer zu dunkel_target-scene *geraten#SCENE-INIT*}

\ex.\dep{Das_experienced hat_x-noncomp mir_experiencer gerade_x-noncomp noch_x-noncomp *gefehlt#EXPERIENCE*}

\ex.\dep{they_experiencer *need#EXPERIENCE-NECESSITY* six months_experienced for digestion_x-explanation}

For more uses, see the examples for \fr{MESSAGE} in Section~\ref{sec:MESSAGE}.

\subsection{\fr{ACTIVITY}}
\label{sec:ACTIVITY}

\rl{is-active} actively participates in \rl{activity}.

Used for dynamic scenes where \rl{is-active} has agency and that cannot well be
framed as a state change.

\ex.\dep{Kim_is-active *worked#ACTIVITY*}

\ex.\dep{Kim_is-active *partied#ACTIVITY*}

\ex.\dep{Kim_is-active had *sex#ACTIVITY*}

\ex.\dep{after some *work#ACTIVITY* with a colored pencil_m-means}

\ex.\dep{I_is-active devoted myself to *geography#ACTIVITY*}

\subsection{\fr{MODE}}
\label{sec:MODE}

Used for adverbial modifiers that have no arguments other than the phrase they
modify, and that, rouhgly speaking, indicate the modal strength of what is
expressed and/or its relation to the discourse.

\ex.\dep{Even_m-mode *Kim#IDENTIFICATION* did n't know that}

\ex.\dep{They_x-causer only_m-mode *rinsed#ADORNMENT-TARNISHMENT-DEINIT* the dishes_initial-surface}

\ex.\dep{*Passt#COMPARISON* das_compared eh_m-mode ?}

\ex.\dep{Kim_x-experiencer probably_m-mode *knows#MESSAGE* that_content}

\ex.\dep{That_has-quality 's really_m-mode *great#QUALITY*}

\ex.\dep{Kim_has-location is not_m-mode *here#LOCATION*}

\subsection{\fr{ACCOMPANIMENT}}
\label{sec:ACCOMPANIMENT}

\rl{accompanier} accompanies \rl{accompanied}, meaning that it occurs together
with it or participates equally in the same scene.

\ex.\dep{*veggies#CLASS* with rice_m-accompanier}

\ex.\dep{The veggies_accompanied *come#ACCOMPANIMENT* with rice_accompanier}

\ex.\dep{Kim_x-causer *added#ACCOMPANIMENT-INIT* rice_target-accompanier to the veggies_accompanied}

\ex.\dep{Rolling thunder_accompanier *accompanies#ACCOMPANIMENT* the rain_accompanied}

Often, the accompanier denotes not the accompanying scene but an entity
participating in it, and must be metonymically understood as the scene.

\ex.\dep{Kim_has-location *cycled#LOCATION-CHANGE* to Rome_target-location with Sandy_m-accompanier}

\ex.\dep{Kim_is-active *danced#ACTIVITY* with Sandy_x-accompanier}

\ex.\dep{Kim_participant *had#SCENE* sex_scene with Sandy_x-accompanier}

\ex.\dep{Kim_x-accompanier *chased#UNANCHORED-MOTION* Sandy_in-motion around the block_transitory-location}

\ex.\dep{Kim_x-accompanier *accompanied#ACCOMPANIMENT* Sandy_accompanied}

\ex.\dep{Kim_x-accompanier *accompanied#ACCOMPANIMENT* Sandy_accompanied on the piano_x-means}

\subsection{\fr{DEPICTIVE}}
\label{sec:DEPICTIVE}

Special case of \fr{ACCOMPANIMENT} where \rl{depictive} (aka \rl{accompanier})
assigns a participant of \rl{has-depictive} (aka \rl{accompanied}) a role (cf.
Sec.~\ref{sec:control}).

\ex.\dep{Kim_has-location__x-sender *entered#LOCATION-INIT* the room_target-location **singing#MESSAGE**_m-depictive}

\subsection{\fr{ATTRIBUTE}}
\label{sec:ATTRIBUTE}

In a scene \rl{has-attribute}, \rl{attribute} is the part or attribute of one
or more participants that is most directly involved in the scene. Add a
dependency link between the participant and its attribute to indicate wich
participant(s) have the attribute.

\ex.\dep{Kim_compared__has-quality *exceeds#COMPARISON* Sandy_reference__has-quality in **height#QUALITY**_x-attribute}

\ex.\dep{That_has-quality__has-quality is *great#QUALITY* in terms of **ROI#QUALITY**_m-attribute}

\ex.\dep{Kim_hitting__x-whole ist auf den **Kopf#CLASS**_x-attribute *gefallen#HITTING*}

\ex.\dep{Kim_x-causer *hit#HITTING* Sandy_hit__x-whole on the **head#CLASS**_x-attribute with a stick_hitting}

\subsection{\fr{ASSET}}
\label{sec:ASSET}

In a scene \rl{has-asset}, \rl{asset} is given or offered in an exchange or wager.

\ex.\dep{Kim_target-possessor *bought#POSSESSION-CHANGE* the house_possession for a million dollars_x-asset}

\ex.\dep{Kim_x-sender *offered#MESSAGE* Sandy_x-experiencer a million dollars_content for the house_x-asset}

\ex.\dep{I_x-sender *bet#MESSAGE* you_x-experiencer 30 bucks_x-asset to an apple_x-reference he will win_content}

\subsection{\fr{COMPARISON}}
\label{sec:COMPARISON}

\rl{compared} is characterized with respect to \rl{reference}.

Examples of comparing scenes:

\ex.\dep{Compared_m-reference to Sandy, Kim_has-quality is *tall#QUALITY*}

\ex.\dep{Sandy_has-quality is *short#QUALITY* whereas Kim is tall_m-comparison}

\ex.\dep{They_x-sender *demonize#MESSAGE* the left_topic while doing_m-reference nothing about the right}

Examples of comparing non-scene entities:

\ex.\dep{Kim_compared *outranks#COMPARISON* Sandy_reference}

\ex.\dep{Kim_compared *exceeds#COMPARISON* Sandy_reference in height_x-attribute}

\ex.\dep{The Polish restaurant_compared *compared#COMPARISON* favorably_x-quality to the Spanish one_reference}

\ex.\dep{Kim_x-experiencer *compared#COMPARISON* Coke_compared to Pepsi_reference}

The \rl{reference} need not be an entity similar to the \rl{compared}, it can also be an abstract constraint:

\ex.\dep{The program_compared *conforms#COMPARISON* to the spec_reference}

\ex.\dep{Kim_compared *ran#COMPARISON-DEINIT* afoul_m-noncomp of Fielding 's constraints_reference}

\subsection{\fr{CONCESSION}}
\label{sec:CONCESSION}

Special case of \fr{COMPARISON}, where \rl{compared} is what's \rl{asserted} and \rl{reference} is what's \rl{conceded}.

\ex.\dep{Kim_has-location *went#LOCATION-CHANGE* out_target-location despite the rain_m-conceded}

\ex.\dep{It_x-noncomp *rained#STATE*, but Kim went_m-asserted out}

\ex.\dep{Kim_sender *sent#SENDING* Sandy_x-experiencer a letter_sent , but it never arrived_m-asserted}

\ex.\dep{Kim_has-location *came#LOCATION-INIT* although Sandy had told_m-conceded them not to}

\subsection{\fr{EXPLANATION}}
\label{sec:EXPLANATION}

\rl{explanation} explains \rl{explained}, but is not a cause, but, e.g., a purpose.

\ex.\dep{I_x-sender am *stressing#MESSAGE* this_topic because it is important_m-explanation}

\ex.\dep{Kim_has-location__target-possessor *went#LOCATION-CHANGE* to town_target-location to **buy#POSSESSION-CHANGE**_m-explanation food__possession}

\subsection{\fr{LOCATION}}
\label{sec:LOCATION}

Describes \rl{has-location} as located or moving wrt. respect to \rl{location}.

\ex.\dep{the *hat#CLASS* in the box_m-location}

\ex.\dep{Kim_has-location *lives#LOCATION* in Boston_location}

\ex.\dep{Kim_has-location *went#LOCATION-CHANGE* from the living room_initial-location through the door_transitory-location into the kitchen_target-location}

\ex.\dep{Kim_x-causer *placed#LOCATION-CHANGE* the hat_has-location on the table_target-location}

\subsection{\fr{WRAPPING-WEARING}}
\label{sec:WRAPPING-WEARING}

Special case of \fr{LOCATION} where \rl{wearer} (aka \rl{location}) wears or is
wrapped in \rl{wrapper} (aka \rl{has-location}).

\ex.\dep{Kim_wearer is *wearing#WRAPPING-WEARING* a shirt_wrapper}

\ex.\dep{Kim_wearer is *wearing#WRAPPING-WEARING* glasses_wrapper}

\ex.\dep{The shroud_wrapper *wraps#WRAPPING-WEARING* the scepter_wearer}

\ex.\dep{Kim_target-wearer *put#WRAPPING-WEARING-INIT* on a sweater_wrapper}

\ex.\dep{Kim_initial-wearer *took#WRAPPING-WEARING-DEINIT* off their glasses_wrapper}

\subsection{\fr{ADORNMENT-TARNISHMENT}}
\label{sec:ADORNMENT-TARNISHMENT}

Special case of \fr{LOCATION} where \rl{ornament} (aka \rl{has-location}) sits on \rl{surface} (aka \rl{location}).

\dep{Kim_x-causer *decorated#ADORNMENT-TARNISHMENT* the balcony_surface with fairy lights_ornament}

\dep{Kim_x-causer *splashed#ADORNMENT-TARNISHMENT-INIT* Sandy_surface with water_ornament}

\dep{Kim_x-causer *washed#ADORNMENT-TARNISHMENT-DEINIT* the dirt_ornament off Sandy_initial-surface}

\dep{Kim_x-causer *washed#ADORNMENT-TARNISHMENT-DEINIT* Sandy_initial-surface}

\subsection{\fr{HITTING}}
\label{sec:HITTING}

Special case of \fr{LOCATION-INIT} where \rl{hitting} (aka \rl{has-location})
comes into contact with \rl{hit} (aka \rl{target-location}).

\ex.\dep{Kim_x-causer *hit#HITTING* Sandy_hit}

\ex.\dep{Kim_x-causer *hit#HITTING* Sandy_hit with a stick_hitting}

\ex.\dep{The stick_hitting *hit#HITTING* Sandy_hit}

\ex.\dep{Kim_x-causer *hit#HITTING* Sandy_hit__x-whole on the **head#CLASS**_x-attribute with a pool noodle_hitting}

\ex.\dep{Kim_x-causer *kicked#HITTING* Sandy_hit}

\subsection{\fr{INGESTION}}
\label{sec:INGESTION}

Special case of \fr{LOCATION-INIT} where \rl{ingester} (aka
\rl{target-location}) ingests \rl{ingested} (aka \rl{has-location}).

\ex.\dep{Kim_ingester *ate#INGESTION* an apple_ingested}

\ex.\dep{Kim_ingester *nibbled#INGESTION* on the pretzel_ingested}

\subsection{\fr{EXCRETION}}
\label{sec:EXCRETION}

Special case of \fr{LOCATION-DEINIT} where \rl{excreter} (aka
\rl{initial-location}) excretes \rl{excreted} (aka \rl{has-location}).

\ex.\dep{Kim_excreter *threw#EXCRETION* up the pretzel_excreted}

\subsection{\fr{UNANCHORED-MOTION}}
\label{sec:UNANCHORED-MOTION}

Special case of \rl{LOCATION-CHANGE} where no initial or target location is indicated.

\ex.\dep{Kim_in-motion is *running#UNANCHORED-MOTION* along the river_transitory-location}

\ex.\dep{I_x-causer learned to *pilot#UNANCHORED-MOTION* airplanes_in-motion}

\ex.\dep{Kim_in-motion is *dancing#UNANCHORED-MOTION* around the room_transitory-location with Sandy_m-accompanier}

\ex.\dep{Kim_in-motion is an avid_m-quality *unicyclist#UNANCHORED-MOTION*}

%\todo[inline]{define clearly when dancing etc. is UNANCHORED-MOTION and when it is ACTIVITY}

\subsection{\fr{MEANS}}
\label{sec:MEANS}

\rl{has-means} is a scene caused by something via an intermediary \rl{means}.

\ex.\dep{Kim_x-causer *cut#STATE-CHANGE* the cake_has-state with a knife_m-means}

\ex.\dep{Kim_x-causer *painted#ADORNMENT-TARNISHMENT* the room_surface by exploding_m-means a paint bomb}

\ex.\dep{Kim_x-causer__x-causer *used#MEANS* a pen_means to **get#LOCATION-DEINIT**_has-means the lid__has-location off__initial-location}

\ex.\dep{You_x-causer *used#MEANS* me_means !}

\subsection{\fr{MESSAGE}}
\label{sec:MESSAGE}

A message about \rl{topic} with content \rl{content} is expressed or received
or just exists in recorded form. When \rl{content} and \rl{topic} are both realized, \rl{content} must assign a role to \rl{topic}.

\subsubsection{Expression}

\ex.\dep{Kim_x-sender *yelped#MESSAGE*}

\ex.\dep{Kim_x-sender *said#MESSAGE* : it 's fine_content}

\ex.\dep{Kim_x-sender *said#MESSAGE* it was fine_content}

\ex.\dep{Kim_x-sender *called#MESSAGE* Sandy_topic__x-sender a **liar#MESSAGE**_content}

\ex.\dep{Kim_x-sender *told#MESSAGE* Sandy_x-experiencer a secret_content}

\ex.\dep{Kim_x-sender *talked#MESSAGE* about Sandy_topic}

\ex.\dep{Kim_x-sender *talked#MESSAGE* **shit#MESSAGE**_content about Sandy_topic__topic}

\ex.\dep{Kim_x-sender and Sandy_x-sender *conversed#MESSAGE*}

\ex.\dep{Kim_x-sender *conversed#MESSAGE* with Sandy_x-accompanier}

\subsubsection{Gesture}

\ex.\dep{Kim_x-sender *curtseyed#MESSAGE* to the Queen_x-experiencer}

\ex.\dep{Kim_x-causer>>x-sender *shook#UNANCHORED-MOTION>>MESSAGE* their head_in-motion>>x-noncomp no_x-sent>>content}

\subsubsection{Performance}

Performance of a work of art is framed as \fr{MESSAGE} where the work of art is the \rl{topic}.

\ex.\dep{Kim_x-sender *played#MESSAGE* a little tune_topic on their tuba_x-means}

\ex.\dep{They_x-sender *performed#MESSAGE* the play_topic}

\ex.\dep{Kim_x-sender *sang#MESSAGE* a song_topic}

\subsubsection{Depiction}

\ex.\dep{Kim_x-sender *drew#MESSAGE* a heron_topic}

\ex.\dep{a *picture#MESSAGE* of the heron_topic}

\subsubsection{Recording}

\ex.\dep{Kim_x-sender *drew#MESSAGE* a picture_x-created}

\ex.\dep{Kim_x-sender *wrote#MESSAGE* Sandy_x-experiencer a letter_content}

\ex.\dep{Kim_x-sender *wrote#MESSAGE* the message_content__has-quality onto a piece_x-target-location of paper with a pen_m-means in big red **letters#QUALITY**_x-depictive}

\ex.\dep{The concert_topic was *recorded#MESSAGE* on tape_x-target-location}

\ex.\dep{The band_x-sender *recorded#MESSAGE* an album_x-created}

\subsubsection{Perception}

We also frame perception as \fr{MESSAGE}, including mental and volitional perception.

\ex.\dep{Kim_x-experiencer *saw#MESSAGE* a flower_topic}

\ex.\dep{Kim_x-experiencer *found#MESSAGE* the flower_topic__has-quality **beautiful#QUALITY**_content}

\ex.\dep{Kim_x-experiencer *thinks#MESSAGE* Sandy is a liar_content}

\ex.\dep{Kim_x-experiencer *thinks#MESSAGE* Sandy_topic__x-sender a **liar#MESSAGE**_content}

\ex.\dep{Kim_x-experiencer *saw#MESSAGE* Sandy_topic__in-motion **swim#UNANCHORED-MOTION**_content}

\ex.\dep{Kim_x-experiencer__in-motion *wants#MESSAGE* to **swim#UNANCHORED-MOTION**_content}

\ex.\dep{Kim_x-experiencer *wants#MESSAGE* Sandy_topic__in-motion to **swim#UNANCHORED-MOTION**_content}

\ex.\dep{Kim_topic__x-experiencer *seems#MESSAGE* **happy#MESSAGE**_content}

\ex.\dep{Kim_topic__x-experiencer *seems#MESSAGE* **happy#MESSAGE**_content to Sandy_x-experiencer}

\ex.\dep{The Thought Police_x-experiencer *observed#MESSAGE* Winston_topic}

\ex.\dep{Kim_x-experiencer *studies#MESSAGE* linguistics_topic}

\ex.\dep{Sandy_x-experiencer is a *professor#MESSAGE* of linguistics_topic}

\ex.\dep{The jury_x-experiencer *found#MESSAGE* Kim_topic__participant___is-active **guilty#SCENE**_content of the ***crime#ACTIVITY***__scene}

Use \fr{MESSAGE-INIT} (\fr{MESSAGE-DEINIT}, \fr{MESSAGE-INIT-NEG}) for the coming
about (ending, failing to come about) of knowledge and awareness.

\ex.\dep{Kim_x-experiencer *noticed#MESSAGE-INIT* the bird_topic}

\ex.\dep{Kim_x-sender *taught#MESSAGE-INIT* Sandy_x-experiencer Spanish_topic}

\ex.\dep{Kim_x-experiencer *measured#MESSAGE-INIT* the elasticity_topic}

\ex.\dep{Kim_x-initial-experiencer *forgot#MESSAGE-DEINIT* everything they knew_initial-content}

\ex.\dep{Kim_x-initial-experiencer *forgot#MESSAGE-DEINIT* about the cake_topic}

\ex.\dep{Kim_x-experiencer *forgot#MESSAGE-INIT-NEG* to take_target-content the trash out}

\subsection{\fr{PART-WHOLE}}
\label{sec:PART-WHOLE}

\rl{part} is part of \rl{whole}.

\ex.\dep{Kim_m-whole 's *leg#CLASS*}

\ex.\dep{a *man#CLASS* with a mustache_m-part}

\ex.\dep{*part#PART-WHOLE* of the year_whole}

\ex.\dep{wheat_whole *contains#PART-WHOLE* gluten_part}

\subsection{\fr{POSSESSION}}
\label{sec:POSSESSION}

\rl{possessor} possesses or controls the \rl{possessed}.

\ex.\dep{Kim_m-possessor 's *house#CLASS*}

\ex.\dep{Kim_possessor *owns#POSSESSION* a house_possessed}

\ex.\dep{The house_possessed *belongs#POSSESSION* to Kim_possessor}

\ex.\dep{the *owner#POSSESSION* of the house_possessed}

\ex.\dep{Kim_possessor *has#POSSESSION* Sandy 's phone_possessed}

\ex.\dep{Kim_target-possessor *bought#POSSESSION-CHANGE* a house_possessed from Sandy_initial-possessor}

\ex.\dep{Sandy_initial-possessor *sold#POSSESSION-CHANGE* Kim_target-possessor the house_possessed}

\ex.\dep{Kim_initial-possessor *kept#POSSESSION-CONTINUATION* the house_possessed}

\ex.\dep{Kim_initial-possessor *lost#POSSESSION-DEINIT* the house_possessed}

\ex.\dep{Caesar_target-possessor *conquered#POSSESSION-INIT* Gaul_possessed}

\ex.\dep{Caesar_target-possessor 's *conquest#POSSESSION-INIT* of Gaul_possessed}

\ex.\dep{Kim_initial-possessor *owes#POSSESSION-CHANGE-NECESSITY* Sandy_target-possessor money_possessed}

\subsection{\fr{QUANTITY}}
\label{sec:QUANTITY}

\rl{quantity} is the quantity, degree, or extent of \rl{has-quantity}.

\ex.\dep{three_m-quantity *burgers#CLASS*}

\ex.\dep{three_m-quantity *liters#QUANTITY* of coke_has-quantity}

\ex.\dep{We_x-sender *discourage#MESSAGE* this_topic emphatically_m-quantity}

\subsection{\fr{SENDING}}
\label{sec:SENDING}

\rl{sender} originates a message, \rl{sent}, that can be experienced.

\ex.\dep{According to Kim_m-sender , it_x-noncomp is *raining#STATE*}

For more uses, see \fr{MESSAGE} (Section~\ref{sec:MESSAGE}).

\subsection{\fr{SEQUENCE}}
\label{sec:SEQUENCE}

\rl{follows} follows \rl{followed}, e.g., temporally, logically, by rank, as heir, etc.

\ex.\dep{Form_follows *follows#SEQUENCE* function_followed}

\ex.\dep{Cook_follows is Jobs_followed 's *successor#SEQUENCE*}

\ex.\dep{Das_follows *fußt#SEQUENCE* auf einer falschen Vorstellung_followed}

\ex.\dep{Kim_x-experiencer *deduced#SEQUENCE* the truth_follows from the clues_followed}

\ex.\dep{Given that I 'm tired_m-followed , I_has-location wo n't be *there#LOCATION*}

\subsection{\fr{CAUSATION}}
\label{sec:CAUSATION}

Special case of \fr{SEQUENCE} where \rl{causer} (aka \rl{followed}) causes \rl{result} (aka \rl{follows}).

\ex.\dep{Kim_x-causer *broke#STATE-CHANGE* the glass_has-state}

\ex.\dep{The knife_x-causer *cut#STATE-CHANGE* the bread_has-state}

\ex.\dep{Kim_x-causer *cut#STATE-CHANGE* the bread_has-state with a knife_m-means}

\ex.\dep{The war_causer *caused#CAUSATION* a famine_result}

\ex.\dep{There_x-noncomp *was#SCENE* a famine_scene because of the war_m-causer}

\ex.\dep{Der Wasserdruck_has-quantity *stieg#QUANTITY-CHANGE*, wodurch der Brunnen überfloss_m-result}

\ex.\dep{Die Qualität_result ist der Motivation_causer *geschuldet#CAUSATION*}

\ex.\dep{Kim_has-location *went#LOCATION-CHANGE* to town_target-location because they wanted_m-causer to buy food}

Note how the last example expresses a purpose, but expresses it as a cause, so
\rl{m-causer} lis the right label to use. Compare this to construal as a
purpose:

\ex.\dep{Kim_has-location *went#LOCATION-CHANGE* to town_target-location to buy_m-explanation food}

\subsection{\fr{REACTION}}
\label{sec:REACTION}

Special case of \fr{CAUSATION} where \rl{trigger} (aka \rl{causer}) triggers a
\rl{reaction} (aka \rl{result}) in the \rl{x-causer}.

\ex.\dep{Kim_x-causer__x-sender *reacted#SEQUENCE* to the allegations_trigger with a **denial#MESSAGE**_reaction}

\subsection{\fr{RESULTATIVE}}
\label{sec:RESULTATIVE}

Special case of \fr{CAUSATION} where \rl{resultative} (aka \rl{result}) assigns
an argument of \rl{has-resultative} (aka \rl{causer}) a role. We treat the
English resultative construction as a valency-changing operation that adds one
or two arguments to the matrix predicate, so we use \rl{x-resultative} rather
than \rl{m-resultative}.

\ex.\dep{Kim_x-causer *hammered#HITTING* the metal_hit__has-state **flat#STATE**_x-resultative}

\ex.\dep{Kim_x-causer *painted#ADORNMENT-TARNISHMENT* the room_surface__has-quality **red#QUALITY**_x-resultative}

\ex.\dep{Kim_experiencer *sneezed#EXPERIENCE* the napkin_x-resultative__x-initial-location:has-location off the **table#CLASS**_x-resultative}

In the last example, we use \rl{x-initial-location:has-location} to specify not
only the role of the napkin in the resulting event (\rl{has-location}) but also
that of the table (\rl{initial-location}). Using \rl{x-has-location} would be
imprecise because we would then assume that the table has \rl{location}.

\subsection{\fr{CONDITION}}
\label{sec:CONDITION}

Special case of \fr{SEQUENCE} where \rl{condition} (aka \rl{followed}) is a
condition to \rl{has-condition} (aka \rl{follows}).

\ex.\dep{I_has-social-relation will *join#SOCIAL-RELATION-INIT* the club if they ask_m-condition me}

\ex.\dep{The start date_has-condition is *contingent#CONDITION* on their approval_condition}

\ex.\dep{Eine Aussöhung_has-condition *bedingt#SEQUENCE* eine Entschuldigung_condition}

\subsection{\fr{EXCEPTION}}
\label{sec:EXCEPTION}

Special case of \fr{SEQUENCE} where \rl{exception} (aka \rl{followed}) is an
exception (a negative condition, if you will) to \rl{has-exception} (aka \rl{follows}).

\ex.\dep{Except for Kim_m-exception , everybody_has-social-relation *joined#SOCIAL-RELATION-INIT*}

\subsection{\fr{SOCIAL-RELATION}}
\label{sec:SOCIAL-RELATION}

\rl{has-social-relation} is an individual that is in some socially constructed
relationship with \rl{social-relation}. \rl{social-relation} might, e.g., be a
relative, a friend, an organization, a responsibility, or a judicial sentence.

\ex.\dep{Kim_has-social-relation 's *friend#SOCIAL-RELATION*}

\ex.\dep{Kim_social-relation is my_has-social-relation *cousin#SOCIAL-RELATION*}

\ex.\dep{Kim_social-relation and Sandy are *friends#SOCIAL-RELATION*}

\ex.\dep{Kim_has-social-relation is *friends#SOCIAL-RELATION* with Sandy_social-relation}

\ex.\dep{Kim_has-social-relation *works#SOCIAL-RELATION* at Google_social-relation}

\ex.\dep{Kim_has-social-relation *works#SOCIAL-RELATION* for Sandy_social-relation}

\ex.\dep{Kim_has-social-relation *emcees#SOCIAL-RELATION*}

\ex.\dep{Kim_has-social-relation is *hosting#SOCIAL-RELATION* the party_social-relation}

\ex.\dep{Kim_has-social-relation is under house *arrest#SOCIAL-RELATION*}

\ex.\dep{Kim_has-social-relation 's *sentence#SOCIAL-RELATION* was suspended}

\ex.\dep{Kim_has-social-relation *married#SOCIAL-RELATION-INIT* Sandy_target-social-relation}

\ex.\dep{The official_x-causer *married#SOCIAL-RELATION-INIT* Kim_has-social-relation to Sandy_target-social-relation}

\ex.\dep{The official_x-causer *married#SOCIAL-RELATION-INIT* Kim_has-social-relation and Sandy}

\ex.\dep{Kim_has-social-relation *divorced#SOCIAL-RELATION-DEINIT* Sandy_initial-social-relation}

\ex.\dep{Kim_has-social-relation *befriended#SOCIAL-RELATION-INIT* Sandy_target-social-relation}

\ex.\dep{Kim_has-social-relation *took#SOCIAL-RELATION-INIT* the job_target-social-relation}

\ex.\dep{Kim_has-social-relation *joined#SOCIAL-RELATION-INIT* Google_target-social-relation}

\ex.\dep{Kim_has-social-relation *joined#SOCIAL-RELATION-INIT* a union_target-social-relation}

\ex.\dep{Sandy_x-causer *fired#SOCIAL-RELATION-DEINIT* Kim_has-social-relation from their job_initial-social-relation}

\ex.\dep{Kim_has-social-relation *left#SOCIAL-RELATION-DEINIT* Google_initial-social-relation}

\ex.\dep{Kim_has-social-relation *assumed#SOCIAL-RELATION-INIT* office_target-social-relation}

\ex.\dep{The judge_x-causer *sentenced#SOCIAL-RELATION-INIT* Kim_has-social-relation to three days_target-social-relation in prison}

\ex.\dep{Kim_has-social-relation was *pardoned#SOCIAL-RELATION-DEINIT*}

\subsection{\fr{TIME}}
\label{sec:TIME}

\rl{time} indicates when, how often, or for how long \rl{has-time} takes place. Also evoked by time expressions without arguments.

\ex.\dep{Kim_in-motion *swims#UNANCHORED-MOTION* on Monday_m-time}

\ex.\dep{Kim_experiencer *sneezed#EXPERIENCE* twice_m-time}

\ex.\dep{Kim_in-motion *swam#UNANCHORED-MOTION* for an hour_m-time}

\ex.\dep{Kim_x-sender *says#MESSAGE* hello_content whenever I meet_m-time them}

\ex.\dep{*Once#TIME* when I was six years old}

\ex.\dep{the six *months#TIME*__scene they__x-experiencer **need#SCENE-NECESSITY**_m-scene for digestion__participant_has-time}

\subsection{\fr{NONCOMP}}
\label{sec:NONCOMP}

Used to mark syntactic arguments that are thought of as part of the predicate, as in verbal idioms, weather verbs, inherently reflexive verbs, or existential \emph{there}.

\ex.\dep{Kim_destroyed *kicked#DESTRUCTION* the bucket_x-noncomp}

\ex.\dep{It_x-noncomp is *raining#STATE*}

\ex.\dep{I_x-sender *address#MESSAGE* myself_x-noncomp to you_x-experiencer}

\ex.\dep{There_x-noncomp *was#SCENE* a famine_scene}

Light verbs, on the other hand, are treated with \fr{SCENE}, see Section~\ref{sec:SCENE}.

\section{Memos}

\subsection{Prefer Core over Non-core Arguments}

When an argument fills both a core and a non-core role, it is more important to
annotate the former.

\ex.\dep{Kim_has-location *drove#LOCATION-CHANGE* to Boston_target-location}

\ex.\dep{Kim_x-causer *drove#LOCATION-CHANGE* the car_has-location to Boston_target-location}

\ex.\dep{They_target-possessor *plundered#POSSESSION-CHANGE* Rome_initial-possessor}

\ex.\dep{Kim_initial-wearer *undressed#WRAPPING-WEARING-DEINIT*}

\subsection{Arguments Determine Frames}

The most important criterion in choosing a frame for a predicate is that there
should be suitable roles for the predicate's arguments, even if they are
unrealized (implicit) in the annotated instance. For example, while
\emph{drawing} denotes a \fr{CLASS} of things, it can occur with a
prepositional argument denoting a \rl{topic}, so \fr{MESSAGE} is a better
choice.

\ex.\dep{my_m-sender first_m-rank *drawing#MESSAGE*}

\ex.\dep{my_m-sender first_m-rank *drawing#MESSAGE* of a snake_topic}

\ex.\dep{Kim_x-causer *helped#SCENE-INIT* Sandy_participant}

\ex.\dep{Kim_x-causer *helped#SCENE-INIT* Sandy_participant clean_target-scene the dishes}

This logic extends to \emph{shadow arguments} and \emph{default arguments}
\citep{pustejovsky-1995-generative,di-fabio-etal-2019-verbatlas}, i.e., arguments
that do not appear in the syntactic argument structure because they are
incorporated into the predicate or logically implied, like the bones in
\ref{ex:debone}, mucus and air in \ref{ex:sneeze}, or groceries in
\ref{ex:deliver}.

\ex.\label{ex:debone}\dep{Kim_x-causer *deboned#PART-WHOLE-DEINIT* the fish_initial-whole}

\ex.\label{ex:sneeze}\dep{Kim_excreter *sneezed#EXCRETION*}

\ex.\label{ex:deliver}\dep{Our local supermarket_x-causer *delivers#LOCATION-INIT*}

\subsection{A Participant whose Syntactic Argument Position is Occupied Should Not Be Treated like an Implicit Argument}

For example, consider \ref{ex:cut1}, Here, \emph{The knife} occupies the subject position and should be treated as the causer of the cutting. We could add the person handling the knife as the causer, and treat the knife as an instrument. However, to add the former to the sentence, we would not merely have to add another realized argument, but also change the syntactic argument structure so that the the subject position goes to that causer, as in \ref{ex:cut2}. Thus, we treat this as a different framing with a different causer, rather than a more explicit version of the same framing. Likewise, \ref{ex:high1} and \ref{ex:high2} are two different framings, one with \emph{price} as \rl{has-state}, and one with \emph{butter}.

\ex.\label{ex:cut1}\dep{The knife_x-causer *cut#STATE-CHANGE* the butter_has-state}

\ex.\label{ex:cut2}\dep{Kim_x-causer *cut#STATE-CHANGE* the butter_has-state with the knife_x-means}

\ex.\label{ex:high1}\dep{The price_has-quantity is *high#QUANTITY*}

\ex.\label{ex:high2}\dep{The butter_has-quantity__has-quality is *high#QUANTITY* in **price#QUALITY**_x-attribute}

\subsection{When in Doubt, Treat Different Syntactic Frames of the Same Predicate Consistently}

For example, in \ref{ex:chase1}, \emph{chase} could be framed as caused motion
with Kim as \rl{x-causer} or as accompanied motion with Kim as
\rl{x-accompanier}. Because the latter works for other syntactic frames of
\emph{chase} as well, as in \ref{ex:chase2}, prefer it.

\ex.\a.\label{ex:chase1}\dep{Kim_x-accompanier *chased#UNACHORED-MOTION* Sandy_in-motion around the block_transitory-location}
    \b.\label{ex:chase2}\dep{Kim_x-accompanier *chased#UNACHORED-MOTION* after Sandy_in-motion}

\subsection{Participant Nouns}

Some nouns denote a person who participates in a specific type of scene in a
specific role. In such cases, use the most appropriate frame for that
scene. For example, in a narrative where the narrator has just been criticized
by a stranger, you could annotate as follows:

\ex.\dep{With that, my_topic *critic#MESSAGE* sat down again}

In other cases, such nouns rather denote a person's profession or expertise or
their role in a social context:

\ex.\dep{He_has-class is a *teacher#CLASS*}

\ex.\dep{He_social-relation is our_has-social-relation *teacher#SOCIAL-RELATION*}

\ex.\dep{She_has-social-relation is the *president#SOCIAL-RELATION* of our club_social-relation}

\subsection{Particle Verbs}

We follow the PARSEME classification of particle verbs into spatial,
semi-non-compositional, and fully non-compositional ones
\citep{savary-etal-2017-parseme,ramisch-etal-2018-edition,ramisch-etal-2020-edition,savary-etal-2023-parseme}.

In UD, particle verbs are connected to their particle via the
\texttt{compound:prt} relation. If the meaning is spatial, this dependency is
labeled with \rl{initial-location} or \rl{target-location}.

\ex.\dep{*get#LOCATION-DEINIT* the lid_has-location off_initial-location}

\ex.\dep{You_has-location may *go#LOCATION-INIT* in_target-location now_m-time}

In semi-non-compositional particle verbs, where the particle adds a partially predictable
but nonspatial meaning to the verb, use an appropriate role.

\ex.\dep{*eat#INGESTION* up_m-quantity the cookies_ingested} (implies \emph{eat} the cookies)

In fully non-compositional particle verbs, where the meaning is not
predictable, use \rl{m-noncomp}.

\ex.\dep{*do#EXPERIENCE* somebody_experiencer in_m-noncomp} (does not imply \emph{do} somebody)

\section{TODO}

The butter is high in price: high has SCENE-like arguments (participant butter
and price scene), but also expresses a QUANTITY. SCENE-QUANTITY?

A whole section on sentence adverbs: lieber (MESSAGE), sowieso (CONDITION),
ungeachtet (CONCESSION), erstmals (TIME), unvermindert (QUANTITY-CONTINUATION)

Speaker-oriented adverbs: MESSAGE? erstaunlicherweise, geheimnisvollerweise,
glücklicherweise, möglicherweise, notwendigerweise, tragischerweise,
unglaublicherweise (MESSAGE-INIT-NEG?), unglücklicherweise, zweckmäßigerweise?

codify the general principle somewhere: if superframe and ARG1 have the same
name (quasi-unary relations), we can just use m-rel. Otherwise, use m-scene.

\bibliographystyle{apalike}
\bibliography{anthology,custom}

\end{document}

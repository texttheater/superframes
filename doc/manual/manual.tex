\documentclass[a4paper]{article}

\usepackage[utf8]{inputenc}

\usepackage{tgpagella}
\usepackage[T1]{fontenc}

%opening
\title{Superframes v0 Manual}
\author{Kilian Evang}

\usepackage{hyperref}

\usepackage{booktabs}
\usepackage{linguex}
\usepackage{natbib}
\usepackage{relsize}
\usepackage{tocbibind}
\usepackage{tikz-dependency}

\hyphenation{Frame-Net}

\begin{document}

\maketitle

\tableofcontents

\clearpage
\section{Introduction}

Superframes is an annotation scheme for semantic roles. It has the following goals:

\begin{enumerate}
    \item \textbf{Lexicon-free annotation:} With fine-grained schemes like FrameNet \citep{baker-etal-1998-berkeley-framenet} or PropBank \citep{palmer-etal-2005-proposition}, annotators have to constantly look up which frames exist and which roles are defined for them. Lexicons are also perennially incomplete, and the process of extending them is complicated. Superframes defines only a small number of coarse-grained frames with the aim of making annotation quick and easy across all languages and domains.
    \item \textbf{Obvious role choices through framing:} In semantic role inventories in the VerbNet tradition \citep{kipper-schuler-2005-verbnet}, roles are semantically defined only vaguely and ambiguously. For example, the subject of the English verb \emph{watch} can be described as an Agent as well as an Experiencer. Prior approaches to resolving this ambiguity involve the creation of a lexicon (see above) or giving up on the idea of categorial role labels altogether \citep{white-etal-2016-universal}. The Superframes approach is to proceed in two steps: first pick the most fitting superframe for each predicate, then the core roles are clearly defined. Additional argument roles are handled via mixin roles (see below).
    \item \textbf{Comprehensive annotation with a unified vocabulary}: Superframes is a comprehensive and unified inventory of coarse semantic roles applicable to all types of morphosyntactic dependencies between content words, including modifier relations, compound relations, state argument relations, process argument relations, event argument relations, discourse relations, etc. It is designed to be annotated on top of existing morphosyntactic dependency graphs \citep[e.g., Universal Dependencies; ][]{de-marneffe-etal-2021-universal}. This has the advantage that the markables are pre-identified and that an explicit annotation of the morphosyntax-semantics interface results.
    \item \textbf{Ambiguity tolerance:} not all ambiguities in choosing a superframe can be resolved. Superframes embraces data perspectivism \citep{basile-2020-end} and encourages annotators to annotate multiple possibilities, in particular in the case of metaphorical language.
\end{enumerate}

\clearpage
\section{Types of Roles}

\begin{table}
    \begin{tabular}{llll}
        \toprule
        \textbf{relation} & \textbf{description} & \textbf{domain} & \textbf{range} \\
        \midrule
        \multicolumn{4}{l}{\emph{entity-entity}} \\
        cmp & comparison & compared & reference \\
        idn & identifier & entity & identifier \\
        loc & location & located & location \\
        msg & message & topic & comment \\
        pss & possession/control & possessee & possessor \\
        qnt & quantity & of what & how much \\
        scn & scene & participant & scene \\
        soc & social & somebody & relative/org/task... \\
        suc & succession & follows & predecessor/base/background \\
        whl & part-whole & part & whole \\
        \midrule
        \multicolumn{4}{l}{\emph{scene-entity}} \\
        ast & asset & scene & asset \\
        ben & beneficiary & scene & beneficiary \\
        cau & causer & scene & cause(r) \\
        cnc & concession & scene & despite what \\
        cnd & condition & scene & condition \\
        mnr & manner & scene & manner \\
        mns & means & scene & means \\
        prp & purpose & scene & purpose \\
        rcp & recipient & scene & recipient \\
        snd & sender & scene & sender \\
        tmp & temporal & scene & time/frequency/... \\
        \midrule
        \multicolumn{4}{l}{\emph{constructional}} \\
        anc & ancillary & predicate & ancillary entity \\
        att & attribute & predicate & attribute \\
        cpd & complex predicate & predicate & same predicate \\
        dis & discourse & token & discourse function \\
        dpc & depictive & predicate & secondary predicate\\
        exp & expletive & predicate & expletive \\
        rsd & resultative & predicate & affected entity \\
        rsr & resultative & predicate & result \\ 
        \bottomrule
    \end{tabular}
    \caption{The inventory of superframes}
    \label{tab:inventory}
\end{table}

The Superframes annotation scheme is based on the binary relations shown in Table~\ref{tab:inventory}. Borrowing terminology from AMR, we call the first relate the ``domain'' and the second relate the ``range''. This inventory is then used to annotate bilexical dependencies with roles. We distinguish a number of types of roles, as explained in the following.

\subsection{Type I: Modifier Roles}

Modifier dependencies are annotated with a plain binary relation or its inverse (denoted by \textsf{-of}, as in AMR). This applies equally to verb, noun, and other modifiers.

\ex. \a. \dep{*partied* at home_loc}
     \b. \dep{a *man* with a mustache_whl-of}

\subsection{Type II: Core Argument Roles}

Predicates such as verbs (but also adjectives, state/process/event nouns, relational nouns, etc.) evoke their own superframe. The core arguments are those that correspond to the domain and the range, respectively. We denote them by the suffixes \textsf{d} and \textsf{r}, respectively.

\ex. \a. \dep{Kim_locd *went* to Boston_locr}
     \b. \dep{Kim_pssr *owns* a house_pssd}

The first example illustrates that Superframes abstract away from aktionsart: it does not matter for the choice of superframe or roles whether a state (Kim is in Boston), an event bringing that state about (Kim went to Boston), or a process (Kim is walking) is described. Borrowing terminology from UCCA \citep{abend-rappoport-2013-universal}, we collectively call states, processes, and events ``scenes''.

\subsection{Type IIa: Initial and Intermediate Range Roles}

However, some predicates denote the dissolution of a relation between a domain and an initial range, and the establishment of the domain and a new range. To distinguish the initial from the final range, we use the prefix \textsf{ir} instead of \textsf{r} for it. Likewise, we use \textsf{mr} for intermediate ranges.

\ex. \a. \dep{Kim_locd *went* from Chicago_locir via Pittsburg_locmr to Boston_locr}\\
     \b. \dep{Kim_pssir *kept* the house_pssd}
     \b. \dep{Kim_pssir *lost* the house_pssd}

\subsection{Type III: Non-core Argument Roles}

Some predicates have more than just the domain and range arguments. For such arguments, annotators should choose the binary relation that describes the relation between the scene and the argument best, and prefix it with \textsf{x} to distinguish it from a modifier. Particularly frequent non-core roles are \textsf{xcau}, \textsf{xsnd}, and \textsf{xrcp}.

\ex. \a. \dep{Sandy_xcau *brought* Kim_locd to Boston_locr}
     \b. \dep{Kim_xsnd *talked* about Sandy_msgd}\\
     \b. \dep{Kim_xrcp *saw* Sandy_msgd swim_msgr}
     \b. \dep{Kim_xrcp *searched* the woods_xloc for Sandy_msgd}

\subsection{How to Frame Scenes}

Part of the annotation task is to choose a superframe for each instance of a predicate.

Implicit, shadow, and default arguments \citep{di-fabio-etal-2019-verbatlas} should
be treated the same as regular arguments for framing. For example,
\emph{debone} denotes the removal of bones (denoted by a shadow argument) from
a body, so prefer \textsf{whl} over \textsf{scn}.

\ex. \a. \dep{The cook_xcau *deboned* the fish_whlir}
     \b. \dep{Kim_locir *sneezed*}

%TBD: is this really a good idea? Where do you draw the line? Cherrypicking, babysitting? Difference between xcau and scnd.

Prefer core roles over non-core roles. For example, if the subject is both the causer and the final possessor in a scene, choose \textsf{pssr} over \textsf{xcau}.

\ex. \a. \dep{They_pssr *plundered* Rome_pssir}
     \b. \dep{Kim_locir *undressed*}

When in doubt, prefer roles that works for multiple syntactic frames of the
same predicate. For example, in \ref{ex:chase-caused} \emph{chase} could be
framed as caused motion (with an \textsf{xcau}) or as accompanied motion (with
an \textsf{xanc}). Because the latter works for other syntactic frames of
\emph{chase} as well, as in \ref{ex:chase-accompanied}, prefer it.

\ex. \a. \label{ex:chase-caused} \dep{Kim_xanc *chased* Sandy_locd around the block_locmr}
     \b. \label{ex:chase-accompanied} \dep{Kim_xanc *chased* after Sandy_locd}

\subsection{Dual Framing}

For predicates that seem to fit two superframes equally well, annotate both. Likewise, if both a literal and a metaphorical meaning are accessible to you, annotate both.

\ex.
\a. \begin{dependency}
    \begin{deptext}
        Kim \& refused \& to \& eat \\
    \end{deptext}
    \depedge[edge height=\baselineskip]{2}{1}{xsnd}
    \depedge[edge height=\baselineskip]{2}{4}{msgd}
    \depedge[edge below,edge height=\baselineskip]{2}{1}{scnd:locr}
    \depedge[edge below,edge height=\baselineskip]{2}{4}{scnr}
\end{dependency}
\b. \begin{dependency}
    \begin{deptext}
        A \& hush \& passed \& over \& the \& group \\
    \end{deptext}
    \depedge[edge height=\baselineskip]{3}{2}{locd}
    \depedge[edge height=\baselineskip]{3}{6}{locmr}
    \depedge[edge below,edge height=\baselineskip]{3}{2}{scnr}
    \depedge[edge below,edge height=\baselineskip]{3}{6}{scnd:xsnd}
\end{dependency}

%TBD: defend scn/msg; deduce msg/suc

\clearpage
\section{Core Relations}

% Template:
% "definition"
% modifier examples
% static argument examples
% dynamic argument examples
% subframes

\subsection{Comparison (\textsf{cmp})}

\subsubsection{Modifier Examples}

\ex. \a. \dep{Compared_cmp to Sandy, Kim_scnd is *tall*}

\subsubsection{Argument Examples}

\ex.
\a. \dep{Kim_cmpd *outranks* Sandy_cmpr}
\b. \dep{Kim_cmpd *exceeds* Sandy_cmpr in height_xatt}
\b. \dep{Kim_scnd is *taller* than Sandy_xcmp}
\b. \dep{The Polish restaurant_cmpd *compared* favorably_mnr to the Spanish one_cmpr}
\b. \dep{Kim_xrcp *compared* Coke_scnd to Pepsi_scnr}

\clearpage
\subsection{Location (\textsf{loc})}
\label{sec:loc}

Describes the location (or change of location, i.e., motion) of the \textsf{locd}.

\subsubsection{Modifier Examples}

\ex. \a. \dep{The *hat* in the box_loc}

\subsubsection{Static Argument Examples}

\ex. \a. \dep{Kim_locd *lives* in Boston_locr}

\subsubsection{Dynamic Argument Examples}

\ex. \a. \dep{Kim_locd *went* from the living room_locir through the door_locmr into the kitchen_locr}
     \b. \dep{Kim_xcau *placed* the hat_locd on the table_locr}
     \b. \dep{Kim_locd is *running*}
     \b. \dep{Kim_locd is *dancing* around the room_locmr with Sandy_xanc}

\subsubsection{Wrapping and Wearing}

\ex. \a. \dep{Kim_locr is *wearing* a shirt_locd}
     \b. \dep{Kim_locr is *wearing* glasses_locd}
     \b. \dep{The shroud_locd *wraps* the scepter_locr}
     \b. \dep{Kim_locr *put* on a sweater_locd}
     \b. \dep{Kim_locir *took* off their glasses_locd}

\subsubsection{Ingestion and Excretion}

\ex. \a. \dep{Kim_locr *ate* an apple_locd}
     \b. \dep{Kim_locir *threw* up the apple_locd}
     \b. \dep{Kim_locir *sneezed*}

\subsubsection{Embellishment and Tarnishment}

\ex. \a. \dep{Kim_xcau *decorated* the balcony_locr with fairy lights_locd}
     \b. \dep{Kim_xcau *splashed* Sandy_locr with water_locd}
     \b. \dep{Kim_xcau *cleaned* the dirt_locd off Sandy_locir}

\clearpage
\subsection{Message (\textsf{msg})}
\label{sec:msg}

A message is expressed or received, where \textsf{msgd} is what the message is
about, and \textsf{msgr} is the message itself, or its contents. Often used
together with \textsf{xsnd} (sender) and \textsf{xrcp} (recipient).

\subsubsection{Expression}

\ex. \a. \dep{Kim_xsnd *yelped*}
     \b. \dep{Kim_xsnd *said* : it 's fine_msgr}
     \b. \dep{Kim_xsnd *said* it was fine_msgr}
     \b. \dep{Kim_xsnd *called* Sandy_msgd a liar_msgr}
     \b. \dep{Kim_xsnd *told* Sandy_xrcp a secret_msgr}
     \b. \dep{Kim_xsnd *talked* about Sandy_msgd}
     \b. \dep{Kim_xsnd *talked* shit_msgr about Sandy_msgd}
     \b. \dep{Kim_xsnd and Sandy *conversed*}
     \b. \dep{Kim_xsnd *conversed* with Sandy_xanc}

\subsubsection{Gesture}

\ex. \a. \dep{Kim_xsnd *curtseyed* to the Queen_xrcp}
     \b. \dep{Kim_xsnd *shook* their head_cpd no_msgr}

\subsubsection{Performance}

Performing a work of art is framed as expression where the work of art is the
\textsf{msgd}.

\ex. \a. \dep{Kim_xsnd *played* a little tune_msgd on their tuba_xmns}
     \b. \dep{They_xsnd *performed* the play_msgd}
     \b. \dep{Kim_xsnd *sang* a song_msgd}

\subsubsection{Depiction}

\ex. \a. \dep{Kim_xsnd *drew* a heron_msgd}
     \b. \dep{Kim_xcau *drew* a picture_scnd}
     \b. \dep{a *picture* of the heron_msgd}

\subsubsection{Recording}

\ex. \a. \dep{Kim_xsnd *wrote* Sandy_xrcp a letter_msgr}
     \b. \dep{Kim_xsnd *wrote* the message_msgr onto a piece of paper_xloc with a pen_mns in big red letters_xdpc}
     \b. \dep{The concert_msgd was *recorded* on tape_xloc}

\subsubsection{Perception}

Perception, including mental and volitional perception.

\ex. \a. \dep{Kim_xrcp *saw* a flower_msgd}
     \b. \dep{Kim_xrcp *found* the flower_msgd beautiful_msgr}
     \b. \dep{Kim_xrcp *thinks* Sandy is a liar_msgr}
     \b. \dep{Kim_xrcp *thinks* Sandy_msgd a liar_msgr}
     \b. \dep{Kim_xrcp *saw* Sandy_msgd swim_msgr}
     \b. \dep{Kim_xrcp *wants* to swim_msgr}
     \b. \dep{Kim_xrcp *wants* Sandy_msgd to swim_msgr}
     \b. \dep{Kim_msgd *seems* happy_msgr}
     \b. \dep{Kim_msgd *seems* happy_msgr to Sandy_xrcp}
     \b. \dep{The Thought Police_xrcp *observed* Winston_msgd}
     \b. \dep{Kim_xrcp *studies* linguistics_msgd}
     \b. \dep{Sandy is a *professor* of linguistics_msgd}
     \b. \dep{Kim_xrcp *measured* the elasticity_msgd}
     \b. \dep{Kim_xrcp *deduced* the truth_msgr}
     \b. \dep{The jury_xrcp *found* Kim_msgd guilty_msgr}

\subsubsection{Reaction}

Predicates that evoke an underspecified scene in reaction to something are
framed as perception, too.

\ex. \a. \dep{Kim_xrcp *reacted* badly_mnr to the misfortune_msgd}
     \b. \dep{Kim_xrcp *responded* to my letter_msgd}
     \b. \dep{Kim_xrcp *managed* with dealing_msgd the cards}

\subsubsection{Additional Markables}

TBD

\clearpage
\subsection{Possession/Control (\textsf{pss})}
\label{sec:pss}

\subsubsection{Modifiers}

\ex. \a. \dep{Kim_pss 's *house*}

\subsubsection{Static Arguments}

\ex. \a. \dep{Kim_pssr *owns* a house_pssd}
     \b. \dep{The house_pssd *belongs* to Kim_pssr}
     \b. \dep{the *owner* of the house_pssd}
     \b. \dep{Kim_pssr *has* Sandy 's phone_pssd}

\subsubsection{Dynamic Arguments}
\ex. \a. \dep{Kim_pssr *bought* a house_pssd from Sandy_pssir}
     \b. \dep{Kim_pssir *kept* the house_pssd}
     \b. \dep{Kim_pssir *lost* the house_pssd}
     \b. \dep{Caesar_pssr *conquered* Gaul_pssd}
     \b. \dep{Caesar_pssr 's *conquest* of Gaul_pssd}

\clearpage
\subsection{Scene (\textsf{scn})}
\label{sec:scn}

\textsf{scnd} is the single core participant of a scene specified by the
predicate, or a participant in a scene specified by the \textsf{scnr}.

\subsubsection{States and Properties}

\ex. \a. \dep{Kim_scnd is *tall*}
     \b. \dep{Kim_scnd is a *person*}
     \b. \dep{The painting_scnd *improved*}
     \b. \dep{Kim_xcau *improved* the painting_scnd}

\subsubsection{Activities}

``Activities'' are events with a single core participant that can be
characterized as actively participating in the event.

\ex. \a. \dep{Kim_scnd *partied*}
     \b. \dep{Kim_scnd *partied* with Sandy_anc}
     \b. \dep{Kim_scnd *had* sex_cpd with Sandy_xanc}

\subsubsection{Experiences}

``Experiences'' ar events with a single core participant that can be
characterized as undergoing the event.

\ex. \a. \dep{Kim_xcau *attacked* Sandy_scnd}
     \b. \dep{Kim_xcau *used* a pen_scnd to get_xprp the lid off}
     \c. \dep{Kim_scnd *needs* Sandy_scnr}

In the last example, understand \emph{you} to metonymically stand for a scene,
e.g. one of Sandy helping Kim.

\subsubsection{Transformation and Creation}

\textsf{scnd} denotes the entity undergoing the transformation, and
\textsf{scnr} denotes its new state. Creation is framed as transformation of
some material (\textsf{scnd}) into the newly created entity (\textsf{scnr}).

\ex. \a. \dep{The ice_scnd *turned* into water_scnr}
     \b. \dep{God_xcau *made* people_scnr out of clay_scnd}
     \b. \dep{A rock_scnr *formed*}
     \b. \dep{Kim_xcau *created* a work_scnr of art}

\subsubsection{Reproduction}

\ex. \a. \dep{Kim_xcau *copied* the book_scnd}
     \b. \dep{Kim_xcau *translated* the book_scnd to German_scnr}

\subsubsection{Destruction}

\ex. \a. \dep{The vase_scnd *broke*}
     \b. \dep{Kim_xcau *broke* the vase_scnd}
     \b. \dep{Kim_xcau *killed* Sandy_scnd}
     \b. \dep{Sandy_scnd *died*}

%\subsubsection{Auxilary Verbs and Copulas}
%
%Auxiliary verbs, when annotating on top of SUD rather than UD, also fall under
%\textsf{scn}.
%
%\ex. \dep{We_scnd *will* see_scnr}
%     \dep{*Have* you_scnd eaten_scnr ?}
%     \dep{Kim_scnd *must* go_scnr}
%     \dep{Kim_scnd *did* not go_scnr}
%     \dep{Kim_scnd *is* a champion_scnr}
%     \dep{Kim_scnd *is* tall_scnr}

\subsubsection{Phase}

\ex. \a. \dep{The wound_scnd *began* to heal_scnr}
     \b. \dep{A commotion_scnr *started*}
     \b. \dep{Kim_xcau *started* a commotion_scnr}
     \b. \dep{The concert_scnir *ended*}
     \b. \dep{The concert_scnir *continued*}
     \b. \dep{Kim_xcau *interrupted* the session_scnir}
     \b. \dep{The storm_scnir *ebbed*}
     \b. \dep{Kim_xcau *calmed* the commotion_scnir}

\subsubsection{Causation}

\ex. \a. \dep{Kim_xcau *let* Sandy_scnd join_scnr}
     \b. \dep{Kim_xcau *made* Sandy_scnd join_scnr}
     \b. \dep{Kim_xcau *allowed* Sandy_scnd to join_scnr}

\subsubsection{Prevention}

\ex. \a. \dep{Kim_xcau *kept* Sandy_scnd from joining_scnr}
     \b. \dep{Swift action_xcau *prevented* an outbreak_scnr}
     \b. \dep{Kim_scnd *refrained* from going_scnr}
     \b. \dep{Kim_xcau *saved* Sandy_scnd from the dragon_scnr}

In the last example, understand \emph{dragon} metonymically as a scene in which
the dragon causes harm to Sandy.

\subsubsection{Relative Clauses}

For relative clauses, the \textsf{scn} label is used as a modifier relation:

\ex. \a. \dep{a *package* that was too heavy_scn}

\subsubsection{Additional Markables}

TBD

\clearpage
\subsection{Social (\textsf{soc})}
\label{sec:soc}

The domain is an individual that is in some socially constructed relationship with the range. The range might e.g. be a relative, a friend, an organization, a responsibility, or a judicial sentence.

\subsubsection{State Examples}

\ex. \a. \dep{Kim_socr is my_socd *cousin*}
     \b. \dep{Kim_socd and Sandy are *friends*}
     \b. \dep{Kim_socd is *friends* with Sandy_socr}
     \b. \dep{Kim_socd *works* at Google_socr}
     \b. \dep{Kim_socd *works* for Sandy_socr}
     \b. \dep{Kim_socd *emcees*}
     \b. \dep{Kim_socd is *hosting* the party_socr}
     \b. \dep{Kim_socd is under house *arrest*}

\subsubsection{Event Examples}

\ex. \a. \dep{Kim_socd *married* Sandy_socr}
     \b. \dep{The official_xcau *married* Kim_socd to Sandy_socr}
     \b. \dep{The official_xcau *married* Kim_socd and Sandy}
     \b. \dep{Kim_socd *divorced* Sandy_socir}
     \b. \dep{Kim_socd *befriended* Sandy_socr}
     \b. \dep{Kim_socd *took* the job_socr}
     \b. \dep{Kim_socd *joined* a Google_socr}
     \b. \dep{Kim_socd *joined* a union_socr}
     \b. \dep{Sandy_xcau *fired* Kim_socd from their job_socir}
     \b. \dep{Kim_socd *left* Google_socir}
     \b. \dep{Kim_socd *assumed* office_socr}
     \b. \dep{The judge_xcau *sentenced* Kim_socd to three days_socr in prison}
     \b. \dep{Kim_socd was *pardoned*}

\clearpage
\section{Peripheral Relations}

\clearpage
\section{Constructional Relations}

\subsection{Complex Predicate (\textsf{cpd})}

TBD criteria: inner modification (heave a deep sigh, give a loud shout), possessive participants (lose one's cool, pull so's leg), cf. Croft; outer modification (have sex with)

\clearpage
\section{Difficult Cases}

Functor head constructions

\ex. \a. \dep{You_cmpd are *more* of a dancer_xnuc?? than she is_cmpr}

Neither subject nor object have a core role:

\ex. \a. \dep{It_exp *rains*} (scn)
     \b. \dep{Kim_xsnd *plays* the flute_mns} (msg)
     \c. \dep{Kim_xcau *paints*} (scn?)

Should we force some metonymy to force an overt core role? E.g., have \emph{flute} stand for the msgr?

I need you.

He responded to the provocation with violence.

He began the party with a speech.

Spend money/time on something

Deserve

Socially constructed properties like names, prices: name the ship, it costs 1M, our story on that is not yet clear

How does our terminology relate to Croft's?

reference = n/a, because references don't have roles

modification = modifier roles

predication = argument roles

action = subdistinguished by relation

property/object = a distinction we do not make

But where do binary states like possession fit into Croft's scheme?

Explicit distinctions we stay away from for now, many additional complexities
trying to pin them down:

\begin{itemize}
    \item state vs. process vs. event, or really any aspectual information that
        is not strictly required because of source/path/goal-like arguments.
        Thus: \dep{Kim_xcau *paints*}.
    \item state vs. property vs. object predication
\end{itemize}

\clearpage

\bibliographystyle{apalike}
\bibliography{anthology,custom}

\end{document}

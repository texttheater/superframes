\documentclass[a4paper]{article}

\usepackage[utf8]{inputenc}

\usepackage{tgpagella}
\usepackage[T1]{fontenc}

%opening
\title{Superframes v0 Manual}
\author{Kilian Evang}
\date{Last updated: \today}

\usepackage{hyperref}

\usepackage{booktabs}
\usepackage{linguex}
\usepackage{natbib}
\usepackage{relsize}
\usepackage{tocbibind}
\usepackage{tikz-dependency}

\hyphenation{Frame-Net}

\begin{document}

\maketitle

\tableofcontents

%%%%%%%%%%%%%%%%%%%%%%%%%%%%%%%%%%%%%%%%%%%%%%%%%%%%%%%%%%%%%%%%%%%%%%%%%%%%%%%

\clearpage
\section{Introduction}

Superframes is an annotation scheme for semantic roles. It has the following goals:

\begin{enumerate}
    \item \textbf{Lexicon-free annotation:} With fine-grained schemes like FrameNet \citep{baker-etal-1998-berkeley-framenet} or PropBank \citep{palmer-etal-2005-proposition}, annotators have to constantly look up which frames exist and which roles are defined for them. Lexicons are also perennially incomplete, and the process of extending them is complicated. Superframes defines only a small number of coarse-grained frames with the aim of making annotation quick and easy across all languages and domains, and to minimize role divergence even in the absence of a lexicon.
    \item \textbf{Obvious role choices through framing:} In semantic role inventories in the VerbNet tradition \citep{kipper-schuler-2005-verbnet}, roles are semantically defined only vaguely and ambiguously. For example, the subject of the English verb \emph{watch} can be described as an Agent as well as an Experiencer. Prior approaches to resolving this ambiguity involve the creation of a lexicon (see above) or giving up on the idea of categorial role labels altogether \citep{white-etal-2016-universal}. The Superframes approach is to proceed in two steps: first pick the most fitting superframe for each predicate, then the core roles are clearly defined. Additional argument roles are handled via mixin roles (see below).
    \item \textbf{Comprehensive annotation with a unified vocabulary}: Superframes is a comprehensive and unified inventory of coarse semantic roles applicable to all types of morphosyntactic dependencies between content words, including modifier relations, compound relations, state argument relations, process argument relations, event argument relations, discourse relations, etc. It is designed to be annotated on top of existing morphosyntactic dependency graphs \citep[e.g., Universal Dependencies; ][]{de-marneffe-etal-2021-universal}. This has the advantage that the markables are pre-identified and that an explicit annotation of the morphosyntax-semantics interface results.
    \item \textbf{Ambiguity tolerance:} not all ambiguities in choosing a superframe can be resolved. Superframes embraces data perspectivism \citep{basile-2020-end} and encourages annotators to annotate multiple possibilities, in particular in the case of metaphorical language.
    \item \textbf{Fine-grained comparisons:} By virtue of being annotated atop syntactic dependencies, Superframes target the syntax-semantics interface and enable cross-linguistic studies that go beyond comparisons of meaning representations like AMR.
\end{enumerate}

%%%%%%%%%%%%%%%%%%%%%%%%%%%%%%%%%%%%%%%%%%%%%%%%%%%%%%%%%%%%%%%%%%%%%%%%%%%%%%%

\clearpage
\section{Types of Roles}

\begin{table}
    \begin{tabular}{lllll}
        \toprule
        \textbf{relation} & \textbf{description} & \textbf{domain} & \textbf{range} & \textbf{sec.} \\
        \midrule
        \multicolumn{4}{l}{\emph{superframe relations}}\\
        acc & accompaniment & accompanied & accompanier & \ref{sec:acc} \\
        ast & asset & scene & asset & \ref{sec:ast} \\
        cau & causer & scene & cause(r) & \ref{sec:cau} \\
        cmp & comparison & compared & reference & \ref{sec:cmp} \\
        loc & location & located & location & \ref{sec:loc} \\
        mnr & manner & scene & manner & \ref{sec:mnr} \\
        mns & means & scene & means & \ref{sec:mns} \\
        msg & message & topic & comment & \ref{sec:msg} \\
        pss & possession/control & possessee & possessor & \ref{sec:pss} \\
        qnt & quantity & of what & how much & \ref{sec:qnt} \\
        scn & scene & participant & scene & \ref{sec:scn} \\
        snd & sender & scene & sender & \ref{sec:snd} \\
        soc & social & somebody & relative/org/task... & \ref{sec:soc} \\
        suc & succession & follows & background & \ref{sec:suc} \\
        tmp & temporal & scene & time/frequency/... & \ref{sec:tmp} \\
        whl & part-whole & part & whole & \ref{sec:whl} \\
        xpl & explanation & scene & explanation & \ref{sec:xpl} \\
        xpr & experiencer & scene & experiencer & \ref{sec:xpr} \\
        \midrule
        \multicolumn{4}{l}{\emph{constructional relations}} \\
        att & attribute & predicate & attribute & \ref{sec:att} \\
        cpd & complex predicate & predicate & same predicate & \ref{sec:cpd} \\
        dim & discourse marker & token & discourse marker & \ref{sec:dim} \\
        dpc & depictive & predicate & secondary predicate & \ref{sec:dpc}\\
        exp & expletive & predicate & expletive & \ref{sec:exp} \\
        rsd & resultative & predicate & affected entity & \ref{sec:rsd} \\
        rsr & resultative & predicate & result & \ref{sec:rsr} \\
        \bottomrule
    \end{tabular}
    \caption{The inventory of superframes, with links to the manual sections in which they are described.}
    \label{tab:inventory}
\end{table}

The Superframes annotation scheme is based on the binary relations shown in Table~\ref{tab:inventory}. Borrowing terminology from AMR, we call the first relate the ``domain'' and the second relate the ``range''. This inventory is then used to annotate bilexical dependencies with roles. We distinguish a number of types of roles, as explained in the following.

%%%%%%%%%%%%%%%%%%%%%%%%%%%%%%%%%%%%%%%%%%%%%%%%%%%%%%%%%%%%%%%%%%%%%%%%%%%%%%%

\subsection{Type I: Modifier Roles}

Modifier dependencies are annotated with a plain binary relation or its inverse (denoted by \textsf{-of}, as in AMR). This applies equally to verb, noun, and other modifiers.

\ex. \a. \dep{*partied* at home_loc}
     \b. \dep{a *man* with a mustache_whl-of}

%%%%%%%%%%%%%%%%%%%%%%%%%%%%%%%%%%%%%%%%%%%%%%%%%%%%%%%%%%%%%%%%%%%%%%%%%%%%%%%

\subsection{Type II: Core Argument Roles}

Predicates such as verbs (but also adjectives, state/process/event nouns, relational nouns, etc.) evoke their own superframe. The core arguments are those that correspond to the domain and the range, respectively. We denote them by the suffixes \textsf{d} and \textsf{r}, respectively.

\ex. \a. \dep{Kim_locd *went* to Boston_locr}
     \b. \dep{Kim_pssr *owns* a house_pssd}

The first example illustrates that Superframes abstract away from aktionsart: it does not matter for the choice of superframe or roles whether a state (Kim is in Boston), an event bringing that state about (Kim went to Boston), or a process (Kim is walking) is described. Borrowing terminology from UCCA \citep{abend-rappoport-2013-universal}, we collectively call states, processes, and events ``scenes''.

%%%%%%%%%%%%%%%%%%%%%%%%%%%%%%%%%%%%%%%%%%%%%%%%%%%%%%%%%%%%%%%%%%%%%%%%%%%%%%%

\subsection{Type IIa: Initial and Intermediate Range Roles}

However, some predicates denote the dissolution of a relation between a domain and an initial range, and the establishment of the domain and a new range. To distinguish the initial from the final range, we use the prefix \textsf{ir} instead of \textsf{r} for it. Likewise, we use \textsf{mr} for intermediate ranges.

\ex. \a. \dep{Kim_locd *went* from Chicago_locir via Pittsburg_locmr to Boston_locr}\\
     \b. \dep{Kim_pssir *kept* the house_pssd}
     \b. \dep{Kim_pssir *lost* the house_pssd}

%%%%%%%%%%%%%%%%%%%%%%%%%%%%%%%%%%%%%%%%%%%%%%%%%%%%%%%%%%%%%%%%%%%%%%%%%%%%%%%

\subsection{Type III: Non-core Argument Roles}

Some predicates have more than just the domain and range arguments. For such arguments, annotators should choose the binary relation that describes the relation between the scene and the argument best, and prefix it with \textsf{x} to distinguish it from a modifier. Particularly frequent non-core roles are \textsf{xcau}, \textsf{xsnd}, and \textsf{xxpr}.

\ex. \a. \dep{Sandy_xcau *brought* Kim_locd to Boston_locr}
     \b. \dep{Kim_xsnd *talked* about Sandy_msgd}\\
     \b. \dep{Kim_xxpr *saw* Sandy_msgd swim_msgr}
     \b. \dep{Kim_xxpr *searched* the woods_xloc for Sandy_msgd}

%%%%%%%%%%%%%%%%%%%%%%%%%%%%%%%%%%%%%%%%%%%%%%%%%%%%%%%%%%%%%%%%%%%%%%%%%%%%%%%

\subsection{How to Frame Scenes}

Part of the annotation task is to choose a superframe for each instance of a predicate.

Implicit, shadow, and default arguments \citep{di-fabio-etal-2019-verbatlas} should
be treated the same as regular arguments for framing. For example,
\emph{debone} denotes the removal of bones (denoted by a shadow argument) from
a body, so prefer \textsf{whl} over \textsf{scn}.

\ex. \a. \dep{The cook_xcau *deboned* the fish_whlir}
     \b. \dep{Kim_locir *sneezed*}

However, if realizing a shadow argument would necessitate that another realized
argument be omitted or change its syntactic argument position, do not treat it
this way. For example:

\ex. \a. \dep{The knife_xcau *cut* the butter_scnd}
     \b. \dep{Kim_xcau *cut* the butter_scnd with the knife_ins}

\ex. \a. \dep{The price_scnd is *high*}
     \b. \dep{The butter_scnd is *high* in price_att}

Prefer core roles over non-core roles. For example, if the subject is both the
causer and the final possessor in a scene, choose \textsf{pssr} over
\textsf{xcau}.

\ex. \a. \dep{They_pssr *plundered* Rome_pssir}
     \b. \dep{Kim_locir *undressed*}

For control verbs, choose a role for the controlled argument that works
regardless of the predicate of the open complement.

\ex.
\a. \dep{Kim_xxpr *wants* to swim_msgr}
\b. \dep{Kim_xcau *used* a hammer_mnsr to smash_mnsd the vase}
\b. \dep{Kim_xcau *used* binoculars_mnsr to watch_msnd Sandy}
\b. \dep{Kim_xcau *reacted* to the news_caud with sadness_caur}
\b. \label{ex:help-ternary} \dep{Kim_xcau *helped* Sandy_scnd clean_scnr the dishes}
\b. \dep{Kim_xcau *did* it_xprd for Sandy_xprr}

When in doubt, prefer roles that works for multiple syntactic frames of the
same predicate. For example, in \ref{ex:chase-caused} \emph{chase} could be
framed as caused motion (with an \textsf{xcau}) or as accompanied motion (with
an \textsf{xacc}). Because the latter works for other syntactic frames of
\emph{chase} as well, as in \ref{ex:chase-accompanied}, prefer it. Likewise,
\ref{ex:help-binary} remains consistent with \ref{ex:help-binary}.

\ex.
\a. \label{ex:chase-caused} \dep{Kim_xacc *chased* Sandy_locd around the block_locmr}
\b. \label{ex:chase-accompanied} \dep{Kim_xacc *chased* after Sandy_locd}
\b. \label{ex:help-binary} \dep{Kim_xcau *helped* Sandy_scnd}

%%%%%%%%%%%%%%%%%%%%%%%%%%%%%%%%%%%%%%%%%%%%%%%%%%%%%%%%%%%%%%%%%%%%%%%%%%%%%%%

\subsection{Dual Framing}

For predicates that seem to fit two superframes equally well, annotate both. Likewise, if both a literal and a metaphorical meaning are accessible to you, annotate both.

\ex.
\a. \begin{dependency}
    \begin{deptext}
        Kim \& refused \& to \& eat \\
    \end{deptext}
    \depedge[edge height=\baselineskip]{2}{1}{xsnd}
    \depedge[edge height=\baselineskip]{2}{4}{msgd}
    \depedge[edge below,edge height=\baselineskip]{2}{1}{scnd:locr}
    \depedge[edge below,edge height=\baselineskip]{2}{4}{scnr}
\end{dependency}
\b. \begin{dependency}
    \begin{deptext}
        A \& hush \& passed \& over \& the \& group \\
    \end{deptext}
    \depedge[edge height=\baselineskip]{3}{2}{locd}
    \depedge[edge height=\baselineskip]{3}{6}{locmr}
    \depedge[edge below,edge height=\baselineskip]{3}{2}{scnr}
    \depedge[edge below,edge height=\baselineskip]{3}{6}{scnd:xsnd}
\end{dependency}

%TBD: defend scn/msg; deduce msg/suc

%%%%%%%%%%%%%%%%%%%%%%%%%%%%%%%%%%%%%%%%%%%%%%%%%%%%%%%%%%%%%%%%%%%%%%%%%%%%%%%
%%%%%%%%%%%%%%%%%%%%%%%%%%%%%%%%%%%%%%%%%%%%%%%%%%%%%%%%%%%%%%%%%%%%%%%%%%%%%%%

\clearpage
\section{Superframe Relations}

\subsection{Accompaniment (\textsf{acc})}
\label{sec:acc}

The \textsf{accd} and the \textsf{accr} appear together, more or less symmetrically.

\ex.
\a. \dep{*chicken* with rice_acc}
\b. \dep{The chicken_accd *comes* with rice_accr}
\b. \dep{Kim_xcau *added* rice_accr to the chicken_accd}

Can also be used for scenes unfolding in close connection.

\ex.
\a. \dep{Rolling thunder_accr *accompanies* the rain_accd}
\b. \dep{Man_locd durfte das Gelände_locr nur_dim unter Aufsicht_acc *betreten*}

Often, the accompanier denotes not a scene but an entity participating in it,
and must be metonymically understood as the scene.

\ex.
\a. \dep{Kim_locd *cycled* to Rome_locr with Sandy_acc}
\b. \dep{Kim_locd *danced* with Sandy_xacc}
\b. \dep{Kim_xacc *chased* Sandy_locd around the block_locmr}
\b. \dep{Kim_accr *accompanied* Sandy_accd}
\b. \dep{Kim_accr *accompanied* Sandy_accd on the piano_xmns}

%%%%%%%%%%%%%%%%%%%%%%%%%%%%%%%%%%%%%%%%%%%%%%%%%%%%%%%%%%%%%%%%%%%%%%%%%%%%%%%

\subsection{Asset (\textsf{ast})}
\label{sec:ast}

Entity given or offered in an exchange or wager.

\ex.
\a. \dep{Kim_pssr *bought* the house_pssd for a million dollars_xast}
\b. \dep{Kim_xsnd *offered* Sandy_xxpr a million dollars_msgr for the house_xast}
\b. \dep{I_xsnd *bet* you_xxpr 30 bucks_xast he will win_msgr}

TODO: can this be merged into qnt?

%%%%%%%%%%%%%%%%%%%%%%%%%%%%%%%%%%%%%%%%%%%%%%%%%%%%%%%%%%%%%%%%%%%%%%%%%%%%%%%

\clearpage
\subsection{Causer (\textsf{cau})}
\label{sec:cau}

Entity causing a scene.

\ex.
\a. \dep{Kim_xcau *broke* the glass_scnd}
\b. \dep{The knife_xcau *cut* the bread_scnd}
\b. \dep{Kim_xcau *cut* the bread_scnd with a knife_mns}
\b. \dep{The war_xcau *caused* a famine_scnr}
\b. \dep{There was a *famine* because of the war_cau}
\b. \dep{Der Wasserdruck_scnd *stieg* , wodurch der Brunnen überfloss_cau-of}
\b. \dep{Kim_locd *went* to town_locr because they wanted_cau to buy food}

Note how the last example expresses a purpose, but construes it as a cause, so
\textsf{cau} is the right relation to use. Compare this to construal as a
purpose:

\ex.
\a. \dep{Kim_locd *went* to town_locr to buy_prp food}

%%%%%%%%%%%%%%%%%%%%%%%%%%%%%%%%%%%%%%%%%%%%%%%%%%%%%%%%%%%%%%%%%%%%%%%%%%%%%%%

\clearpage
\subsection{Comparison (\textsf{cmp})}
\label{sec:cmp}

\subsubsection{Modifier Examples}

\ex.
\a. \dep{Compared_cmp to Sandy, Kim_scnd is *tall*}
\b. \dep{Sandy_scnd is *short* whereas Kim is tall_cmp}
\b. \dep{They_xsnd *demonize* the left_msgd while doing_cmp nothing about the right}
\b. \dep{Kim_locd *went* out despite the rain_cmp}
\b. \dep{Kim_locd *came* although Sandy had told_cmp them not to}
\b. \dep{Kim_xsnd *sent* Sandy_xxpr a letter_msgr , but it never arrived_cmp}

\subsubsection{Argument Examples}

\ex.
\a. \dep{Kim_cmpd *outranks* Sandy_cmpr}
\b. \dep{Kim_cmpd *exceeds* Sandy_cmpr in height_xatt}
\b. \dep{The Polish restaurant_cmpd *compared* favorably_xmnr to the Spanish one_cmpr}
\b. \dep{Kim_xxpr *compared* Coke_scnd to Pepsi_scnr}

\subsubsection{Comparatives}

The comparative of adjectives is treated as a valency-changing operation that
gives the adjective an additional \textsf{xcmp} argument:

\ex.
\a. \dep{Kim_scnd is *taller* than Sandy_xcmp}

%%%%%%%%%%%%%%%%%%%%%%%%%%%%%%%%%%%%%%%%%%%%%%%%%%%%%%%%%%%%%%%%%%%%%%%%%%%%%%%

\clearpage
\subsection{Location (\textsf{loc})}
\label{sec:loc}

Describes the location (or change of location, i.e., motion) of the \textsf{locd}.

\subsubsection{Modifier Examples}

\ex. \a. \dep{The *hat* in the box_loc}

\subsubsection{Static Argument Examples}

\ex. \a. \dep{Kim_locd *lives* in Boston_locr}

\subsubsection{Dynamic Argument Examples}

\ex. \a. \dep{Kim_locd *went* from the living room_locir through the door_locmr into the kitchen_locr}
     \b. \dep{Kim_xcau *placed* the hat_locd on the table_locr}
     \b. \dep{Kim_locd is *running*}
     \b. \dep{Kim_locd is *dancing* around the room_locmr with Sandy_xacc}

\subsubsection{Wrapping and Wearing}

\ex. \a. \dep{Kim_locr is *wearing* a shirt_locd}
     \b. \dep{Kim_locr is *wearing* glasses_locd}
     \b. \dep{The shroud_locd *wraps* the scepter_locr}
     \b. \dep{Kim_locr *put* on a sweater_locd}
     \b. \dep{Kim_locir *took* off their glasses_locd}

\subsubsection{Ingestion and Excretion}

\ex. \a. \dep{Kim_locr *ate* an apple_locd}
     \b. \dep{Kim_locr *nibbled* on the pretzel_locd}
     \b. \dep{Kim_locir *threw* up the apple_locd}
     \b. \dep{Kim_locir *sneezed*}

\subsubsection{Embellishment and Tarnishment}

\ex.
\a. \dep{Kim_xcau *decorated* the balcony_locr with fairy lights_locd}
\b. \dep{Kim_xcau *splashed* Sandy_locr with water_locd}
\b. \dep{Kim_xcau *cleaned* the dirt_locd off Sandy_locir}
\b. \dep{Kim_xcau *cleaned* Sandy_locir}
\b. \dep{*chicken* with rice_loc-of}
\b. \dep{The chicken_locr *comes* with rice_loc}
\b. \dep{Kim_xcau *added* rice_locd to the chicken_locr}

\subsubsection{Hitting}

\ex.
\a. \dep{Kim_xcau *hit* Sandy_locr}
\b. \dep{Kim_xcau *hit* Sandy_locr with a stick_locd}
\b. \dep{The stick_locd *hit* Sandy_locr}
\b. \dep{Kim_xcau *hit* Sandy_locr on the head_xatt with a pool noodle_locd}
\b. \dep{Kim_xcau *kicked* Sandy_locr}

%%%%%%%%%%%%%%%%%%%%%%%%%%%%%%%%%%%%%%%%%%%%%%%%%%%%%%%%%%%%%%%%%%%%%%%%%%%%%%%

\clearpage
\subsection{Manner (\textsf{mnr})}
\label{sec:mnr}

Manner in which a scene unfolds.

\ex.
\a. \dep{Kim_xsnd *sang* softly_mnr}
\b. \dep{Kim_xcau *placed* the note_locd surreptitiously_mnr on the counter_locr}

%%%%%%%%%%%%%%%%%%%%%%%%%%%%%%%%%%%%%%%%%%%%%%%%%%%%%%%%%%%%%%%%%%%%%%%%%%%%%%%

\clearpage
\subsection{Means (\textsf{mns})}
\label{sec:mns}

Intermediary between the causer and a scene; instrument or second scene through
which the causer causes the first scene.

\ex.
\a. \dep{Kim_xcau *cut* the cake_scnd with a knife_mns}
\b. \dep{Kim_xcau *painted* the room_locr by exploding_mns a paint bomb}
\b. \dep{Kim_xcau *used* a pen_mnsr to get_mnsd the lid off}
\b. \dep{You_xcau *used* me_mnsd !}

%%%%%%%%%%%%%%%%%%%%%%%%%%%%%%%%%%%%%%%%%%%%%%%%%%%%%%%%%%%%%%%%%%%%%%%%%%%%%%%

\clearpage
\subsection{Message (\textsf{msg})}
\label{sec:msg}

A message is expressed or received, where \textsf{msgd} is what the message is
about, and \textsf{msgr} is the message itself, or its contents. Often used
together with \textsf{xsnd} (sender) and \textsf{xxpr} (experiencer).

\subsubsection{Expression}

\ex. \a. \dep{Kim_xsnd *yelped*}
     \b. \dep{Kim_xsnd *said* : it 's fine_msgr}
     \b. \dep{Kim_xsnd *said* it was fine_msgr}
     \b. \dep{Kim_xsnd *called* Sandy_msgd a liar_msgr}
     \b. \dep{Kim_xsnd *told* Sandy_xxpr a secret_msgr}
     \b. \dep{Kim_xsnd *talked* about Sandy_msgd}
     \b. \dep{Kim_xsnd *talked* shit_msgr about Sandy_msgd}
     \b. \dep{Kim_xsnd and Sandy *conversed*}
     \b. \dep{Kim_xsnd *conversed* with Sandy_xacc}

\subsubsection{Gesture}

\ex. \a. \dep{Kim_xsnd *curtseyed* to the Queen_xxpr}
     \b. \dep{Kim_xsnd *shook* their head_cpd no_msgr}

\subsubsection{Performance}

Performing a work of art is framed as expression where the work of art is the
\textsf{msgd}.

\ex. \a. \dep{Kim_xsnd *played* a little tune_msgd on their tuba_xmns}
     \b. \dep{They_xsnd *performed* the play_msgd}
     \b. \dep{Kim_xsnd *sang* a song_msgd}

\subsubsection{Depiction}

\ex. \a. \dep{Kim_xsnd *drew* a heron_msgd}
     \b. \dep{a *picture* of the heron_msgd}

When the object is the resulting work of art, frame it as creation (\textsf{scn}) instead:

\ex.
\a. \dep{Kim_xcau *drew* a picture_scnd}

\subsubsection{Recording}

\ex. \a. \dep{Kim_xsnd *wrote* Sandy_xxpr a letter_msgr}
     \b. \dep{Kim_xsnd *wrote* the message_msgr onto a piece of paper_xloc with a pen_mns in big red letters_xdpc}
     \b. \dep{The concert_msgd was *recorded* on tape_xloc}

\subsubsection{Perception}

Perception, including mental and volitional perception.

\ex. \a. \dep{Kim_xxpr *saw* a flower_msgd}
     \b. \dep{Kim_xxpr *found* the flower_msgd beautiful_msgr}
     \b. \dep{Kim_xxpr *thinks* Sandy is a liar_msgr}
     \b. \dep{Kim_xxpr *thinks* Sandy_msgd a liar_msgr}
     \b. \dep{Kim_xxpr *saw* Sandy_msgd swim_msgr}
     \b. \dep{Kim_xxpr *wants* to swim_msgr}
     \b. \dep{Kim_xxpr *wants* Sandy_msgd to swim_msgr}
     \b. \dep{Kim_msgd *seems* happy_msgr}
     \b. \dep{Kim_msgd *seems* happy_msgr to Sandy_xxpr}
     \b. \dep{The Thought Police_xxpr *observed* Winston_msgd}
     \b. \dep{Kim_xxpr *studies* linguistics_msgd}
     \b. \dep{Kim_xsnd *taught* Sandy_xxpr Spanish_msgd}
     \b. \dep{Sandy is a *professor* of linguistics_msgd}
     \b. \dep{Kim_xxpr *measured* the elasticity_msgd}
     \b. \dep{Kim_xxpr *deduced* the truth_msgr}
     \b. \dep{The jury_xxpr *found* Kim_msgd guilty_msgr}

\subsubsection{Additional Markables}

TBD

%%%%%%%%%%%%%%%%%%%%%%%%%%%%%%%%%%%%%%%%%%%%%%%%%%%%%%%%%%%%%%%%%%%%%%%%%%%%%%%

\clearpage
\subsection{Possession/Control (\textsf{pss})}
\label{sec:pss}

\subsubsection{Modifier Examples}

\ex. \a. \dep{Kim_pss 's *house*}

\subsubsection{Static Argument Examples}

\ex. \a. \dep{Kim_pssr *owns* a house_pssd}
     \b. \dep{The house_pssd *belongs* to Kim_pssr}
     \b. \dep{the *owner* of the house_pssd}
     \b. \dep{Kim_pssr *has* Sandy 's phone_pssd}

\subsubsection{Dynamic Argument Examples}
\ex.
\a. \dep{Kim_pssr *bought* a house_pssd from Sandy_pssir}
\b. \dep{Sandy_pssir *sold* Kim_pssr the house_pssd}
\b. \dep{Kim_pssir *kept* the house_pssd}
\b. \dep{Kim_pssir *lost* the house_pssd}
\b. \dep{Caesar_pssr *conquered* Gaul_pssd}
\b. \dep{Caesar_pssr 's *conquest* of Gaul_pssd}

%%%%%%%%%%%%%%%%%%%%%%%%%%%%%%%%%%%%%%%%%%%%%%%%%%%%%%%%%%%%%%%%%%%%%%%%%%%%%%%

\clearpage
\subsection{Quantity (\textsf{qnt})}
\label{sec:qnt}

\textsf{qntr} is the quantity, degree, or extent of \textsf{qntd}.

\subsubsection{Modifier Examples}

\ex.
\a. \dep{three_qnt *burgers*}
\b. \dep{three_qnt *liters* of coke_qnt-of}
\b. \dep{We_xsnd *discourage* this_msgd emphatically_qnt}
\b. \dep{ten *dollars* per item_qnt}

\subsubsection{Argument Examples}

\ex.
\a. \dep{The birds_qntd *number* in the thousands_qntr}

%%%%%%%%%%%%%%%%%%%%%%%%%%%%%%%%%%%%%%%%%%%%%%%%%%%%%%%%%%%%%%%%%%%%%%%%%%%%%%%

\clearpage
\subsection{Scene (\textsf{scn})}
\label{sec:scn}

\textsf{scnd} is the single core participant of a scene specified by the
predicate, or a participant in a scene specified by the \textsf{scnr}.

\subsubsection{States and Properties}

\ex. \a. \dep{Kim_scnd is *tall*}
     \b. \dep{The painting_scnd *improved*}
     \b. \dep{Kim_xcau *improved* the painting_scnd}

\subsubsection{Kinds}

\ex. \a. \dep{Kim_scnd is a *painter*}
     \b. \dep{Thag dog_scnd is a *labrador*}

\subsubsection{Activities}

``Activities'' are events with a single core participant that can be
characterized as actively participating in the event.

\ex. \a. \dep{Kim_scnd *partied*}
     \b. \dep{Kim_scnd *partied* with Sandy_acc}
     \b. \dep{Kim_scnd *had* sex_cpd with Sandy_xacc}

\subsubsection{Experiences}

``Experiences'' are events with a single core participant that can be
characterized as undergoing the event.

\ex. \a. \dep{Kim_xcau *attacked* Sandy_scnd}
     \b. \dep{They_xcau *played* you_scnd like a fool_cmp}
     \b. \dep{Kim_scnd *needs* Sandy_scnr}

In the last example, understand \emph{Sandy} to metonymically stand for a scene,
e.g. one of Sandy helping Kim.

\subsubsection{Transformation and Creation}

\textsf{scnd} denotes the entity undergoing the transformation, and
\textsf{scnr} denotes its new state. Creation is framed as transformation of
some material (\textsf{scnd}, often unexpressed) into the newly created entity
(\textsf{scnr}).

\ex. \a. \dep{The ice_scnd *turned* into water_scnr}
     \b. \dep{God_xcau *made* people_scnr out of clay_scnd}
     \b. \dep{A rock_scnr *formed*}
     \b. \dep{Kim_xcau *created* a work_scnr of art}

\subsubsection{Reproduction}

\ex. \a. \dep{Kim_xcau *copied* the book_scnd}
     \b. \dep{Kim_xcau *translated* the book_scnd to German_scnr}

\subsubsection{Destruction}

\ex. \a. \dep{The vase_scnd *broke*}
     \b. \dep{Kim_xcau *broke* the vase_scnd}
     \b. \dep{Kim_xcau *killed* Sandy_scnd}
     \b. \dep{Sandy_scnd *died*}

%\subsubsection{Auxilary Verbs and Copulas}
%
%Auxiliary verbs, when annotating on top of SUD rather than UD, also fall under
%\textsf{scn}.
%
%\ex. \dep{We_scnd *will* see_scnr}
%     \dep{*Have* you_scnd eaten_scnr ?}
%     \dep{Kim_scnd *must* go_scnr}
%     \dep{Kim_scnd *did* not go_scnr}
%     \dep{Kim_scnd *is* a champion_scnr}
%     \dep{Kim_scnd *is* tall_scnr}

\subsubsection{Phase}

The scene starting, continuing, ending, etc., is always the range. Only use
\textsf{scnd} if the predicate has a separate argument for a participant in the
scene.

\ex. \a. \dep{The wound_scnd *began* to heal_scnr}
     \b. \dep{A commotion_scnr *started*}
     \b. \dep{Kim_xcau *started* a commotion_scnr}
     \b. \dep{The concert_scnir *ended*}
     \b. \dep{The concert_scnir *continued*}
     \b. \dep{Kim_xcau *interrupted* the session_scnir}
     \b. \dep{The storm_scnir *ebbed*}
     \b. \dep{Kim_xcau *calmed* the commotion_scnir}

\subsubsection{Causation}

\ex. \a. \dep{Kim_xcau *let* Sandy_scnd join_scnr}
     \b. \dep{Kim_xcau *made* Sandy_scnd join_scnr}
     \b. \dep{Kim_xcau *allowed* Sandy_scnd to join_scnr}

\subsubsection{Prevention}

\ex. \a. \dep{Kim_xcau *kept* Sandy_scnd from joining_scnr}
     \b. \dep{Swift action_xcau *prevented* an outbreak_scnr}
     \b. \dep{Kim_scnd *refrained* from going_scnr}
     \b. \dep{Kim_xcau *saved* Sandy_scnd from the dragon_scnr}

In the last example, understand \emph{dragon} metonymically as a scene in which
the dragon causes harm to Sandy.

\subsubsection{Relative Clauses}

For relative clauses, the \textsf{scn} label is used as a modifier relation:

\ex. \a. \dep{a *package* that was too heavy_scn}

\subsubsection{Assignment}

The scene relation is also used for the assignment of names, identifiers,
tasks, or abstract assignment of other entities to entities.

\ex.
\a. \dep{The baby_scnd was *named* Kim_scnr}
\b. \dep{*Hangar* \#3_scn is empty}
\b. \dep{The teacher_xcau *assigned* each student_scnd a task_scnr}
\b. \dep{Is that Peter_scn-of 's *chair* ?}

\subsubsection{Additional Markables}

TBD

%%%%%%%%%%%%%%%%%%%%%%%%%%%%%%%%%%%%%%%%%%%%%%%%%%%%%%%%%%%%%%%%%%%%%%%%%%%%%%%

\clearpage
\subsection{Sender (\textsf{snd})}
\label{sec:snd}

Entity that sends a signal that can be perceived by an experiencer.

\ex.
\a. \dep{Kim_xsnd *told* Sandy_xxpr about the virus_msgd}
\b. \dep{Kim_xsnd *called* Sandy_msgd a liar_msgr}
\b. \dep{Kim_xsnd *hummed* a little melody_msgd , but nobody heard_cnc-of it}
\b. \dep{Kim_xsnd *exploded* at Sandy_xxpr}
\b. \dep{Kim_xsnd *bowed* to Sandy_xxpr}

%%%%%%%%%%%%%%%%%%%%%%%%%%%%%%%%%%%%%%%%%%%%%%%%%%%%%%%%%%%%%%%%%%%%%%%%%%%%%%%

\clearpage
\subsection{Social (\textsf{soc})}
\label{sec:soc}

The \textsf{socd} is an individual that is in some socially constructed
relationship with the \textsf{socr}. The \textsf{socr} might e.g. be a
relative, a friend, an organization, a responsibility, or a judicial sentence.

\subsubsection{Static Argument Examples}

\ex. \a. \dep{Kim_socr is my_socd *cousin*}
     \b. \dep{Kim_socd and Sandy are *friends*}
     \b. \dep{Kim_socd is *friends* with Sandy_socr}
     \b. \dep{Kim_socd *works* at Google_socr}
     \b. \dep{Kim_socd *works* for Sandy_socr}
     \b. \dep{Kim_socd *emcees*}
     \b. \dep{Kim_socd is *hosting* the party_socr}
     \b. \dep{Kim_socd is under house *arrest*}
     \b. \dep{Kim_socd 's *sentence* was suspended}

\subsubsection{Dynamic Argument Examples}

\ex. \a. \dep{Kim_socd *married* Sandy_socr}
     \b. \dep{The official_xcau *married* Kim_socd to Sandy_socr}
     \b. \dep{The official_xcau *married* Kim_socd and Sandy}
     \b. \dep{Kim_socd *divorced* Sandy_socir}
     \b. \dep{Kim_socd *befriended* Sandy_socr}
     \b. \dep{Kim_socd *took* the job_socr}
     \b. \dep{Kim_socd *joined* Google_socr}
     \b. \dep{Kim_socd *joined* a union_socr}
     \b. \dep{Sandy_xcau *fired* Kim_socd from their job_socir}
     \b. \dep{Kim_socd *left* Google_socir}
     \b. \dep{Kim_socd *assumed* office_socr}
     \b. \dep{The judge_xcau *sentenced* Kim_socd to three days_socr in prison}
     \b. \dep{Kim_socd was *pardoned*}

%%%%%%%%%%%%%%%%%%%%%%%%%%%%%%%%%%%%%%%%%%%%%%%%%%%%%%%%%%%%%%%%%%%%%%%%%%%%%%%

\clearpage
\subsection{Succession (\textsf{suc})}
\label{sec:suc}

The \textsf{sucd} follows the \textsf{sucr}, e.g., temporally, logically, by
rank, as a heir, etc. Also used for ordinal numbers.

\subsubsection{Modifier Examples}

\ex.
\a. \dep{the third_sucd-of *man*}

\subsubsection{Argument Examples}

\ex.
\a. \dep{Form_sucd *follows* function_sucr}
\b. \dep{Cook_sucd is Jobs_sucr ' *successor*}
\b. \dep{Das_sucd *fußt* auf einer falschen Vorstellung_sucr}
\b. \dep{I_socd will *join* the club_socr if they ask_suc me}
\b. \dep{The start date_sucd is *contingent* on their approval_sucr}
\b. \dep{Eine Aussöhnung_sucd *bedingt* eine Entschuldigung_sucr}

%%%%%%%%%%%%%%%%%%%%%%%%%%%%%%%%%%%%%%%%%%%%%%%%%%%%%%%%%%%%%%%%%%%%%%%%%%%%%%%

\clearpage
\subsection{Temporal (\textsf{tmp})}
\label{sec:tmp}

When, how often, or for how long an event takes place.

\ex.
\a. \dep{Kim_locd *swims* on Monday_tmp}
\b. \dep{Kim_locir *sneezed* twice_tmp}
\b. \dep{Kim_locd *swam* for an hour_tmp}
\b. \dep{Kim_xsnd *says* hello_msgr whenever I meet_tmp them}

%%%%%%%%%%%%%%%%%%%%%%%%%%%%%%%%%%%%%%%%%%%%%%%%%%%%%%%%%%%%%%%%%%%%%%%%%%%%%%%

\clearpage
\subsection{Part-whole (\textsf{whl})}
\label{sec:whl}

The \textsf{whld} is part of the \textsf{whlr}.

\subsubsection{Modifier Examples}

\ex.
\a. \dep{Kim_whl 's *leg*}
\b. \dep{a *man* with a mustache_whl-of}
\b. \dep{the *beginning* of the party_whl}

\subsubsection{Argument Examples}

\ex.
\a. \dep{Wheat_whlr *contains* gluten_whld}

In scene-entity relations, the domain is always a scene, and the range is a
non-core participant, such as a causer, sender, receiver, or beneficiary.

%%%%%%%%%%%%%%%%%%%%%%%%%%%%%%%%%%%%%%%%%%%%%%%%%%%%%%%%%%%%%%%%%%%%%%%%%%%%%%%

\clearpage
\subsection{Explanation (\textsf{xpl})}
\label{sec:xpl}

Explanation that is not a cause, such as a purpose.

\ex.
\a. \dep{Kim_locd *went* to town_locr to buy_xpl food}
\b. \dep{I_xsnd am *stressing* this_sncd because it's important_xpl}

TODO: additional markables

%%%%%%%%%%%%%%%%%%%%%%%%%%%%%%%%%%%%%%%%%%%%%%%%%%%%%%%%%%%%%%%%%%%%%%%%%%%%%%%

\clearpage
\subsection{Experiencer (\textsf{xpr})}
\label{sec:xpr}

Entity that perceives or experiences a scene, or that a message is addressed
to. Also used for beneficiaries and maleficiaries.

\ex.
\a. \dep{Sandy_xxpr *heard* about the virus_msgd}
\b. \dep{Kim_xsnd *told* Sandy_xxpr about the virus_msgd}
\b. \dep{Kim_xxpr *saw* a flower_msgd}
\b. \dep{Kim_xxpr *saw* Sandy_msgd swim_msgr}
\b. \dep{Kim_xxpr *found* the flower_msgd beautiful_msgr}
\b. \dep{The virus_msgd *seems* dangerous_msgr to Kim_xxpr}
\b. \dep{Kim_xsnd *sent* Sandy_xxpr a letter_msgr , but it never arrived_cnc-of}
%\b. \dep{Kim_xxpr *reacted* badly_mnr to the misfortune_msgd}
%\b. \dep{Kim_xxpr *responded* to my letter_msgd}
%\b. \dep{Kim_xxpr *responded* to the attack_msgd with a drone strike_xmns}
\b. \dep{Kim_xxpr *managed* with dealing_msgd the cards}
\b. \dep{Die Piroggen_scnd waren Maria_xxpr zu dunkel_scnr *geraten*}
\b. \dep{Emily_xsnd *delivered* Carlos 's message_msgr to Nianwen to Kilian_xxpr}
\b. \dep{Carlos_xsnd 's *message* to Nianwen_xxpr}
\b. \dep{Kim_locd *danced* for Sandy_xpr}
\b. \dep{Kim_xcau *did* it_xprd for Sandy_xprr}

\subsubsection{Mental State}

\ex.
\a. \dep{Kim_xxpr is *tired*}
\b. \dep{Kim_xxpr is *crazy*}

%%%%%%%%%%%%%%%%%%%%%%%%%%%%%%%%%%%%%%%%%%%%%%%%%%%%%%%%%%%%%%%%%%%%%%%%%%%%%%%
%%%%%%%%%%%%%%%%%%%%%%%%%%%%%%%%%%%%%%%%%%%%%%%%%%%%%%%%%%%%%%%%%%%%%%%%%%%%%%%

\clearpage
\section{Constructional Relations}

Constructional relations do not correspond to superframes, but mark various
types of mismatches between syntactic and semantic predicate-argument
structure.

%%%%%%%%%%%%%%%%%%%%%%%%%%%%%%%%%%%%%%%%%%%%%%%%%%%%%%%%%%%%%%%%%%%%%%%%%%%%%%%

\clearpage
\subsection{Attribute (\textsf{att})}
\label{sec:att}

Used for arguments and adjuncts that denote that part or attribute of another
participant which is most directly involved in the scene.

\ex.
\a. \dep{Kim_cmpd *exceeds* Sandy_cmpr in length_xatt}
\b. \dep{That_scnd is *great* in terms_att of ROI}
\b. \dep{Kim_locd ist auf den Kopf_xatt *gefallen*}
\b. \dep{Kim_xcau *hit* Sandy_locr on the head_xatt with a stick_locd}

TODO: additional markables

%%%%%%%%%%%%%%%%%%%%%%%%%%%%%%%%%%%%%%%%%%%%%%%%%%%%%%%%%%%%%%%%%%%%%%%%%%%%%%%

\clearpage
\subsection{Complex Predicate (\textsf{cpd})}
\label{sec:cpd}

If a predicate together with one or more of its syntactic arguments forms a
complex predicate, those syntactic dependencies are labeled \textsf{cpd}.

\subsubsection{Light Verbs}

\ex.
\a. \dep{Kim_xsnd *gave* a lecture_cpd}
\b. \dep{Kim_scnd *had* sex_cpd with Sandy_xacc}

TODO: causative light verbs

\subsubsection{Verbal Idioms}

\ex.
\a. \dep{Kim_xxpr *went* bananas_cpd}

\subsubsection{Weather pronouns}

\ex.
\a. \dep{It_cpd is *raining*}

\subsubsection{Expletive \emph{there}}

\ex.
\a. \dep{There_cpd *was* a famine_scnr}

\subsubsection{Inherently Reflexive Verbs}

\ex.
\a. \dep{I_xsnd *address* myself_cpd to you_xxpr}

% TBD criteria: inner modification (heave a deep sigh, give a loud shout), possessive participants (lose one's cool, pull so's leg), cf. Croft; outer modification (have sex with)
% outer arguments

%%%%%%%%%%%%%%%%%%%%%%%%%%%%%%%%%%%%%%%%%%%%%%%%%%%%%%%%%%%%%%%%%%%%%%%%%%%%%%%

\clearpage
\subsection{Discourse Marker (\textsf{dim})}
\label{sec:dim}

Used for adverbials etc. that mark a clause for its relation to preceding/following discourse, speaker attitude, etc.

\ex.
\a. \dep{However_dim , nothing_scnr *happened*}
\b. \dep{Even_dim *Kim* didn't see it}
\b. \dep{Der_xxpr *will* nur_dim spielen_msgd}

%%%%%%%%%%%%%%%%%%%%%%%%%%%%%%%%%%%%%%%%%%%%%%%%%%%%%%%%%%%%%%%%%%%%%%%%%%%%%%%

\clearpage
\subsection{Depictive (\textsf{dpc})}
\label{sec:dpc}

Used for secondary predicates that describe the state of one of the participants during the scene.

\ex.
\a. \dep{Kim_scnd *slept* naked_dpc}
\b. \dep{Pythons_locr *swallow* their prey_locd whole_dpc}

TODO: additional markables

%%%%%%%%%%%%%%%%%%%%%%%%%%%%%%%%%%%%%%%%%%%%%%%%%%%%%%%%%%%%%%%%%%%%%%%%%%%%%%%

\clearpage
\subsection{Expletive (\textsf{exp})}
\label{sec:exp}

Used for expletive pronouns and pronominal adverbs that fill in for a
dislocated clause. Not used for weather pronouns and expletive \emph{there},
which are instead treated as parts of complex predicates
(Section~\ref{sec:cpd}).

\ex.
\a. \dep{Kim_xxpr *hates* it_exp when people yell_msgd}
\b. \dep{Wie_cau *kommst* du_xxpr darauf_exp , ich hätte_msgr etwas damit zu tun?}

TODO: additional markables

%%%%%%%%%%%%%%%%%%%%%%%%%%%%%%%%%%%%%%%%%%%%%%%%%%%%%%%%%%%%%%%%%%%%%%%%%%%%%%%

\clearpage
\subsection{Resultative (\textsf{rsd}, \textsf{rsr})}
\label{sec:rsd}\label{sec:rsr}

Secondary predicates that denote the end state of a participant are marked
\textsf{rsr}. If that participant is not assigned a role by the primary
predicate, use \textsf{rsd} instead.

\ex.
\a. \dep{Kim_xcau *hammered* the metal_scnd flat_rsr}
\b. \dep{You_xsnd are *talking* me_rsd silly_rsr}

TODO: additional markables

\appendix

%%%%%%%%%%%%%%%%%%%%%%%%%%%%%%%%%%%%%%%%%%%%%%%%%%%%%%%%%%%%%%%%%%%%%%%%%%%%%%%

\clearpage
\section{Difficult Cases}

Verbs with scene arguments, often control constructions

Guideline: keep number of x-roles small; focus on non-subjects

\ex.
\a. \dep{Kim_xcau *used* a hammer_mnsr to smash_mnsd the vase}
\b. \dep{Kim_xcau *reacted* to the attack_caur with a drone strike_caud}
\b. \dep{Kim_xxpr *managed* with dealing_sncr the cards}
\b. \dep{Kim_accr *accompanied* Sandy_accd on the piano_xmns}
\b. \dep{Kim_xxpr *compared* Sandy_cmpd to a Summer 's day_cmpr}
\b. \dep{Kim_xcau *helped* Sandy_scnd clean_scnr}
\b. \dep{Kim_xxpr *needs* Sandy_scnd to take_scnr a stand}
\b. \dep{Kim_xxpr *needs* Sandy_scnr}
\b. \dep{Kim_scnd *deserves* to be treated_scnr with respect}
\b. \dep{Kim_xcau *began* the party_whlr with a speech_whld}

Functor head constructions

\ex. \a. \dep{You_cmpd are *more* of a dancer_xnuc?? than she is_cmpr}

Neither subject nor object have a core role:

\ex. \a. \dep{It_exp *rains*} (scn)
     \b. \dep{Kim_xsnd *plays* the flute_mns} (msg)
     \c. \dep{Kim_xcau *paints*} (scn?)

Should we force some metonymy to force an overt core role? E.g., have \emph{flute} stand for the msgr?

I need you.

He responded to the provocation with violence.

He began the party with a speech.

Spend money/time on something

Deserve

Socially constructed properties like names, prices: name the ship, it costs 1M, our story on that is not yet clear

How does our terminology relate to Croft's?

reference = n/a, because references don't have roles

modification = modifier roles

predication = argument roles

action = subdistinguished by relation

property/object = a distinction we do not make

But where do binary states like possession fit into Croft's scheme?

Explicit distinctions we stay away from for now, many additional complexities
trying to pin them down:

\begin{itemize}
    \item state vs. process vs. event, or really any aspectual information that
        is not strictly required because of source/path/goal-like arguments.
        Thus: \dep{Kim_xcau *paints*}.
    \item state vs. property vs. object predication
\end{itemize}

Tamm: adnominal possession (not restricted to possession strictu sensu) where
the head's referent is typically identified via the possessor's referent,
e.g.,: Peter's chair (location/temporary possession/assignment)/brother
(social)/book (creator)/house (location/possession)/finger (part) vs.
non-anchoring relations (typification): golden ring (material)/coffee cup
(container-contained/purpose)/autumn flower (temporal).

We need to say something about relational nouns:

\dep{my_scnr *creator*}

Subevents/context:

Das Betreten ist nur in Begleitung gestattet

Frida begleitete Hans auf dem Klavier

%%%%%%%%%%%%%%%%%%%%%%%%%%%%%%%%%%%%%%%%%%%%%%%%%%%%%%%%%%%%%%%%%%%%%%%%%%%%%%%

\section{Motivation}

Why Superframes?

See motivation introduction.

AMR: try to abstract away from superficial differences (unsuccessfully, see work on divergences). E.g., focus is not necessarily placed at predicate associated with syntactic root of sentence.

Superframes: use dependency tree as a scaffold, annotate syntax-semantic interface rather than ``the'' semantics directly.

\clearpage

\bibliographystyle{apalike}
\bibliography{anthology,custom}

\end{document}

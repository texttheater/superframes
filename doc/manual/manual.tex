\documentclass[a4paper]{article}

\usepackage{natbib}
\usepackage{tgpagella}

\usepackage{amstext}
\usepackage{booktabs}
\usepackage{emoji}
\usepackage{fontspec}
\usepackage{hyperref}
\usepackage{linguex}
\usepackage{mathtext}
\usepackage{nicematrix}
\usepackage{relsize}
\usepackage{tikz-dependency}
\usepackage[backgroundcolor=blue!20!white]{todonotes}

\title{Superframes Manual}
\author{Kilian Evang}
\date{Last updated: \today}

% frame and role names
\newcommand{\fr}[1]{\textsf{#1}}
\newcommand{\frs}[1]{\mbox{\textsf{#1}}} % frame suffixes start with hyphen, prevent line break
\newcommand{\rl}[1]{\textsf{#1}}

% box drawing characters
\newfontfamily{\noto}{Noto Sans Mono}
\DeclareTextFontCommand{\textnoto}{\noto}
\newcommand{\Sub}{\textnoto{└} }
\newcommand{\SubSub}{\textcolor{white}{\Sub}\Sub}
\newcommand{\SubSubSub}{\textcolor{white}{\Sub}\SubSub}

\begin{document}

% less white space in examples
\setlength{\Exindent}{0pt}
\setlength{\Exlabelsep}{0pt}
\setlength{\SubExleftmargin}{6pt}
\setlength{\SubSubExleftmargin}{6pt}

\maketitle

%\begin{abstract}
%\end{abstract}


\newpage\tableofcontents


\newpage\section{Introduction}

\begin{table}
    \resizebox{\textwidth}{!}{
        \begin{NiceTabular}{llllllr}
            \toprule
            \RowStyle{\bfseries}\fr{SUPERFRAME} & \rl{initial-arg2} & \rl{arg1} & \rl{arg2} & \rl{transitory-arg2} & \rl{target-arg2} & Sec. \\
            \midrule

            \emoji{link} \fr{SITUATION} & \rl{initial-situator} & \rl{theme} & \rl{situator} & \rl{transitory-situator} & \rl{target-situator} & \ref{sec:SITUATION} \\

            \Sub\emoji{fork-and-knife} \fr{ACCOMPANIMENT} & \rl{initial-accompanier} & \rl{accompanied} & \rl{accompanier} & & \rl{target-accompanier} & \ref{sec:ACCOMPANIMENT} \\
            \SubSub\emoji{paintbrush} \fr{DEPICTIVE} & & \rl{has-depictive} & \rl{depictive} & & & \ref{sec:DEPICTIVE} \\

            \Sub\emoji{money-bag} \fr{ASSET} & & \rl{has-asset} & \rl{asset} & & & \ref{sec:ASSET} \\

            \Sub\emoji{keycap-asterisk} \fr{ATTRIBUTE} & & \rl{has-attribute} & \rl{attribute} & & & \ref{sec:ATTRIBUTE} \\

            \Sub\emoji{balance-scale} \fr{COMPARISON} & & \rl{compared} & \rl{reference} & & & \ref{sec:COMPARISON} \\
            \SubSub\emoji{ok-hand} \fr{CONCESSION} & & \rl{assertion} & \rl{conceded} & & & \ref{sec:CONCESSION} \\

            \Sub\emoji{crossed-swords}\ \fr{EVENT} & & \rl{undergoer} & \rl{event} & & & \ref{sec:EVENT} \\
            \SubSub\emoji{woman-dancing} \fr{ACTIVITY} & & \rl{is-active} & \rl{activity} & & & \ref{sec:ACTIVITY} \\

            \Sub\emoji{sparkles} \fr{EXISTENCE} & \rl{initial-exists} & \rl{material} & \rl{exists} & & \rl{target-exists} & \ref{sec:EXISTENCE} \\
            \SubSub\emoji{memo} \fr{REPRODUCTION} & & \rl{original} & & & \rl{copy} & \ref{sec:REPRODUCTION} \\
            \SubSub\emoji{magic-wand} \fr{TRANSFORMATION-CREATION} & & \rl{material} & & & \rl{created} & \ref{sec:TRANSFORMATION-CREATION} \\

            \Sub\emoji{eye} \fr{EXPERIENCE} & & \rl{experiencer} & \rl{experienced} &  &  & \ref{sec:EXPERIENCE} \\

            \Sub\emoji{red-exclamation-mark} \fr{EXPLANATION} & & \rl{explained} & \rl{explanation} & & & \ref{sec:EXPLANATION} \\
            \SubSub\emoji{bullseye} \fr{PURPOSE} & & \rl{has-purpoe} & \rl{purpose} & & & \ref{sec:PURPOSE} \\

            \Sub\emoji{passport-control} \fr{IDENTIFICATION} & \rl{initial-identifier} & \rl{identified} & \rl{identifier} & & \rl{target-identifier} & \ref{sec:IDENTIFICATION} \\

            \Sub\emoji{round-pushpin} \fr{LOCATION} & \rl{initial-location} & \rl{has-location} & \rl{location} & \rl{transitory-location} & \rl{target-location} & \ref{sec:LOCATION} \\
            \SubSub\emoji{gem-stone} \fr{ADORNMENT-TARNISHMENT} & \rl{initial-surface} & \rl{ornament} & \rl{surface} & & \rl{target-surface} & \ref{sec:ADORNMENT-TARNISHMENT} \\
            \SubSub\emoji{sweat-droplets} \fr{EXCRETION} & \rl{excreter} & \rl{excreted} & & \rl{transitory-location} & \rl{target-location} & \ref{sec:EXCRETION} \\
            \SubSub\emoji{cricket-game} \fr{HITTING} & & \rl{hitting} & \rl{hit} & & & \ref{sec:HITTING} \\
            \SubSub\emoji{bowl-with-spoon} \fr{INGESTION} & & \rl{ingested} & & \rl{transitory-location} & \rl{ingester} & \ref{sec:INGESTION} \\
            \SubSub\emoji{leaf-fluttering-in-wind} \fr{UNANCHORED-MOTION} & & \rl{in-motion} & & \rl{transitory-location} & & \ref{sec:UNANCHORED-MOTION} \\
            \SubSub\emoji{t-shirt} \fr{WRAPPING-WEARING} & \rl{initial-wearer} & \rl{wrapper} & \rl{wearer} & & \rl{target-wearer} & \ref{sec:WRAPPING-WEARING} \\

            \Sub\emoji{hammer} \fr{MEANS} & & \rl{has-means} & \rl{means} & & & \ref{sec:MEANS} \\

            \Sub\emoji{left-speech-bubble} \fr{MESSAGE} & \rl{initial-content} & \rl{topic} & \rl{content} & & \rl{target-content} & \ref{sec:MESSAGE} \\

            \Sub\emoji{red-question-mark} \fr{MODE} & & \rl{has-mode} & \rl{mode} & & & \ref{sec:MODE} \\

            \Sub\emoji{spider-web} \fr{NONCOMP} & & \rl{has-noncomp} & \rl{noncomp} & & & \ref{sec:NONCOMP} \\

            \Sub\emoji{puzzle-piece} \fr{PART-WHOLE} & \rl{initial-whole} & \rl{part} & \rl{whole} & & \rl{target-whole} & \ref{sec:PART-WHOLE} \\
            \SubSub\emoji{light-bulb} \fr{EXAMPLE} & \rl{} & \rl{example} & \rl{exemplified} & & \rl{} & \ref{sec:EXAMPLE} \\

            \Sub\emoji{service-dog} \fr{POSSESSION} & \rl{initial-possessor} & \rl{possessed} & \rl{possessor} & & \rl{target-possessor} & \ref{sec:POSSESSION} \\

            \Sub\emoji{abacus} \fr{QUANTITY} & \rl{initial-quantity} & \rl{has-quantity} & \rl{quantity} & & \rl{target-quantity} & \ref{sec:QUANTITY} \\

            \Sub\emoji{1st-place-medal} \fr{RANK} & \rl{initial-rank} & \rl{has-rank} & \rl{rank} & & \rl{target-rank} & \ref{sec:RANK} \\

            \Sub\emoji{performing-arts} \fr{SCENE} & \rl{initial-scene} & \rl{participant} & \rl{scene} & \rl{transitory-scene} & \rl{target-scene} & \ref{sec:SCENE} \\

            \Sub\emoji{zzz} \fr{STATE} & \rl{initial-state} & \rl{has-state} & \rl{state} & & \rl{target-state} & \ref{sec:STATE} \\
            \SubSub\emoji{red-apple} \fr{QUALITY} & \rl{initial-quality}& \rl{has-quality} & \rl{quality} & & \rl{target-quality} & \ref{sec:QUALITY} \\
            \SubSubSub\emoji{deciduous-tree} \fr{CLASS} & \rl{initial-class} & \rl{has-class} & \rl{class} & & \rl{target-class} & \ref{sec:CLASS} \\
            \SubSub\emoji{skull} \fr{DESTRUCTION} & & \rl{destroyed} & & & & \ref{sec:DESTRUCTION} \\

            \Sub\emoji{megaphone} \fr{SENDING} & & \rl{sent} & \rl{sender} & & & \ref{sec:SENDING} \\

            \Sub\emoji{right-arrow} \fr{SEQUENCE} & & \rl{follows} & \rl{followed} & & & \ref{sec:SEQUENCE} \\
            \SubSub\emoji{joystick} \fr{CAUSATION} & & \rl{result} & \rl{causer} & & & \ref{sec:CAUSATION} \\
            \SubSub\emoji{scroll} \fr{CONDITION} & & \rl{has-condition} & \rl{condition} & & & \ref{sec:CONDITION} \\
            \SubSub\emoji{prohibited} \fr{EXCEPTION} & & \rl{has-exception} & \rl{exception} & & & \ref{sec:EXCEPTION} \\
            \SubSub\emoji{collision} \fr{REACTION} & & \rl{reaction} & \rl{trigger} & & & \ref{sec:REACTION} \\
            \SubSub\emoji{sneezing-face} \fr{RESULTATIVE} & & \rl{has-resultative} & \rl{resultative} & & & \ref{sec:RESULTATIVE} \\

            \Sub\emoji{handshake} \fr{SOCIAL-RELATION} & \rl{initial-social-relation} & \rl{has-social-relation} & \rl{social-relation} & & \rl{target-social-relation} & \ref{sec:SOCIAL-RELATION} \\

            \Sub\emoji{alarm-clock} \fr{TIME} & \rl{initial-time} & \rl{has-time} & \rl{time} & & \rl{target-time} & \ref{sec:TIME} \\

            \bottomrule
        \end{NiceTabular}
    }
    \caption{Hierarchy of Superframes and their Roles}
    \label{tab:superframes}
\end{table}

Superframes is an annotation scheme for semantic roles. Like other such
schemes, it is essentially about pinning down, in a machine-readable form,
``who did what to whom''. It is different from other such schemes, such as
FrameNet \citep{baker-etal-1998-berkeley}, VerbNet
\citep{kipper-schuler-2005-verbnet}, PropBank
\citep{palmer-etal-2005-proposition}, VerbAtlas
\citep{di-fabio-etal-2019-verbatlas}, or WiSER \citep{feng-etal-2022-widely} in
a number of ways. It aims to avoid a number of practical problems in annotating
with those schemes. Here's how Superframes annotation works, in a nutshell:

\begin{enumerate}
    \item Every content word (verb, noun, pronoun, adjective, or adverb) is a
        \emph{predicate}. Every predicate evokes one of a few dozen
        \emph{superframes}, which determines its coarse semantic class and the
        possible role labels for its core arguments.
    \item The syntactic \emph{dependents} of a predicate can be
        \emph{core arguments}, in which case they get one of the role labels
        defined by the superframe of the predicate, or \emph{external
        arguments} or \emph{modifiers}, in which case they are treated as
        evoking their own frame in which the predicate serves as a core argument.
    \item There are only two main core role labels per superframe.
    \item For predicates denoting change (or lack thereof) over time,
        some superframes have \emph{aspectual variants} with role variants that
        allow to distinguish participants before, during, and after an event.
        This avoids having \texttt{Source} and \texttt{Target} as roles in
        their own right, which indicate the time sequence but suppress
        information about the nature of the relation that is changing.
    \item Similarly, Superframes do not have the \texttt{Agent} role, which is
        often in conflict with roles indicating more specifically the agent's
        relation to other participants.
    \item Doubt, ambiguity, and figurativity are systematically treated. If there
        is not one clear solution, the solution is to give two or more
        alternative labels.
\end{enumerate}

Table~\ref{tab:superframes} shows the superframes and their roles, sorted into
a rough hierarchy. At the top is \fr{SITUATION}. All the main superframes are
direct children of \fr{SITUATION}. Some of them have one or more subtypes
intended to make the annotation of certain special cases more intuitive and
unambiguous.


\newpage\subsection{Core Arguments}

The most prototypical predicate is a verb, and the simplest case is a verb with only one argument. It can for example denote a state or an activity:

\ex.\dep{Kim_has-state is *sleeping#STATE*}

\ex.\dep{Kim_is-active is *partying#ACTIVITY*}

With two core arguments, a verb denotes a relation that holds between them:

\ex.\dep{Kim_possessor *owns#POSSESSION* a house_possessed}

\ex.\dep{The house_possessed *belongs#POSSESSION* to Kim_possessor}

\ex.\dep{Kim_topic *seems#MESSAGE* happy_content}


\newpage\subsection{Aspect, Mode, and Polarity}
\label{sec:aspect-mode-polarity}

Rather than a static relationship between two entities, many verbs (and other
predicates) denote a change (or absence of change) in such a relationship. We
sort such predicates into a few coarse aspectual classes. For example,
initiation (\frs{-INIT}) means a state is begun or worked towards, deinitiation
(\frs{-DEINIT}) means a state is ended, completed, or its end is worked
towards, change (\frs{-CHANGE}) combines both, where one state is replaced by
another, continuation (\frs{-CONTINUATION}) means a state persists or is even
intensified, and (\frs{-PREVENTION}) means it fails to come about.
Accordingly, roles with prefix \rl{target-} mark participants at or beyond the
end of the event, \rl{initial-} marks participants at the beginning of the
event, and \rl{transitory-} marks participants at some point during the event.

\ex.\dep{Kim_target-possessor *got#POSSESSION-INIT* the house_possessed}

\ex.\dep{Kim_initial-possessor *lost#POSSESSION-DEINIT* the house_possessed}

\ex.\dep{Kim_initial-possessor *sold#POSSESSION-CHANGE* the house_possessed to Sandy_target-possessor}

\ex.\dep{Kim_initial-possessor *kept#POSSESSION-CONTINUATION* the house_possessed}

\ex.\dep{Kim_has-location *went#LOCATION-CHANGE* from Chicago_initial-location via Pittsburgh_transitory-location to Boston_target-location}

\ex.\label{ex:fall}\dep{The vase_has-location *fell#LOCATION-CHANGE* to the ground_target-location}

\ex.\dep{The vase_has-state *broke#STATE-CHANGE*}

\ex.\dep{Kim_has-social-relation *befriended#SOCIAL-RELATION-INIT* Sandy_target-social-relation}

\ex.\dep{Kim_has-social-relation *married#SOCIAL-RELATION-INIT* Sandy_target-social-relation}

\ex.\dep{Kim_has-social-relation *divorced#SOCIAL-RELATION-DEINIT* Sandy_initial-social-relation}

\ex.\dep{Kim_x-causer *saved#SITUATION-PREVENTION* Sandy_theme from the dragon_target-situator}

The \fr{SCENE} superframe is often evoked by ``light'' verbs that contribute an
aspectual or modal meaning. Thus, its aspectual variants are especially common.

\ex.\dep{The concert_target-scene *began#SCENE-INIT*}

\ex.\dep{The concert_initial-scene *continued#SCENE-CONTINUATION*}

\ex.\dep{The concert_initial-scene *finished#SCENE-DEINIT*}

\ex.\dep{The shouting_initial-scene *intensified#SCENE-CONTINUATION*}

\ex.\dep{The shouting_initial-scene *faded#SCENE-DEINIT*}

\ex.\dep{A coup_target-scene was *attempted#SCENE-INIT*}

\ex.\dep{Kim_participant *finished#SCENE-DEINIT* their work_initial-scene}

\ex.\dep{Swift action_x-causer *prevented#SCENE-PREVENTION* an outbreak_target-scene}

\ex.\dep{Kim_participant *refrained#SCENE-PREVENTION* from going_target-scene}

\ex.\dep{Kim_x-causer *prevented#SCENE-PREVENTION* Sandy_participant from going_target-scene}

In addition, we use the modal suffixes \frs{-NECESSITY} and \frs{-POSSIBILITY}.
They can combine with aspectual suffixes.

\ex.\dep{Change_scene is *necessary#SCENE-NECESSITY*}

\ex.\dep{Change_scene is *possible#SCENE-POSSIBILITY*}

\ex.\dep{Kim_initial-possessor *owes#POSSESSION-CHANGE-NECESSITY* Sandy_target-possessor money_possessed}

Finally, we can use the polarity suffix \frs{-NEG}. It can combine with aspectual and modal suffixes.

\ex.\dep{*absence#EXISTENCE-NEG* of evidence_exists}

\ex.\dep{That_scene is *impossible#SCENE-POSSIBILITY-NEG*}

\ex.\dep{They *never#TIME-NEG* understand}


\newpage\subsection{Non-core Arguments}

Core arguments always get role labels from the superframe the predicate evokes.
But many verbs have more arguments. One common case is a subject that is
presented as the causer of the scene. For example, compare \ref{ex:throw} with
\ref{ex:fall}. The core scene is the same (same superframe, same arguments). We
now assume there is an additional \fr{CAUSATION} scene with \emph{Kim} as the
\rl{causer} and the core scene as the \rl{result}. We denote this by giving
\emph{Kim} the \rl{causer} role label, with an \rl{x-} prefix to mark it as a
non-core role.

\ex.\label{ex:throw}\dep{Kim_x-causer *threw#LOCATION-CHANGE* the vase_has-location to the ground_target-location}

\ex.\dep{Kim_x-causer *broke#STATE-CHANGE* the vase_has-state}

Two other common non-core arguments are the senders and recipients (experiencers) of messages.

\ex.\dep{Kim_x-sender *talked#MESSAGE-INIT* to Sandy_x-experiencer about Bali_topic}

Other non-core arguments are usually rather predicate-specific.

\ex.\dep{Kim_x-experiencer *searched#MESSAGE-INIT* the woods_x-location for Sandy_topic}

\ex.\dep{Kim_initial-possessor *sold#POSSESSION-CHANGE* Sandy_target-possessor the house_possessed for a million dollars_x-asset}


\newpage\subsection{Modifiers}

Like non-core arguments, modifiers are assumed to evoke an additional frame,
and labeled with the role they fill in that frame, but with a prefix marking
them as modifiers: \rl{m-}.

\ex.\dep{Kim_excreter is *sweating#EXCRETION* profusely_m-quantity in the sauna_m-location}

Adjectival and adverbial modification is characterized by the syntactic modifier acting as a predicate, with the syntactic modifiee as an argument.
We label such modifier dependencies \rl{m-scene} (cf. Section~\ref{sec:SCENE}) and add a reverse dependency with the corresponding role label.

\ex.\dep{Ich *spiele#ACTIVITY*__topic **lieber#MESSAGE**_m-scene Schach}

\ex.\dep{Der ist **sowieso#CONDITION**_m-scene *kaputt#STATE*__has-condition}

\ex.\dep{Sie *sprangen#LOCATION-INIT*__asserted des Regens__conceded **ungeachtet#CONCESSION**_m-scene nach draußen}

\ex.\dep{Kim war **unvermindert#QUANTITY-CONTINUATION**_m-scene *fröhlich#MESSAGE*__has-quantity}

If arg2 has the same name as the frame, this structure can be abbreviated to
just use that as a modifier role instead of \rl{m-scene} and a backlink. For
example, the following pairs are equivalent:

\ex.
\a.\dep{a **red#QUALITY**_m-scene *bucket#CLASS*__has-quality}
\b.\dep{a **red#QUALITY**_m-quality *bucket#CLASS*}

\ex.
\a.\dep{The water is **very#QUANTITY**_m-scene *cold#STATE*__has-quantity}
\b.\dep{The water is **very#QUANTITY**_m-quantity *cold#STATE*}

\ex.
\a.\dep{Kim *kommt#LOCATION-INIT*__has-time **erstmals#TIME**_m-scene mit Sandy}
\b.\dep{Kim *kommt#LOCATION-INIT* **erstmals#TIME**_m-time mit Sandy}


\newpage\subsection{Nonverbal Predicates}

So far, we have only looked at verbal predicates. But of course, there are
other types of predicates. An ordinary noun like \emph{tree} evokes the
\fr{CLASS} frame, marking the entity it refers to as being a member of a class
(in this case: the class of trees). There are no arguments here because the
predicate itself doubles as a referent. However, the predicate can of course be
modified:

\ex.\dep{a *tree#CLASS* in the garden_m-location}

\ex.\dep{Kim_m-possessor 's *tree#CLASS*}

Event nouns evoke event frames and have arguments:

\ex.\dep{Kim_x-causer 's *breaking#STATE-CHANGE* of the vase_has-state}

Relational nouns evoke relational frames and have arguments:

\ex.\dep{Kim_has-social-relation 's *friend#SOCIAL-RELATION*}

Pronouns and names evoke the \fr{IDENTIFICATION} frame, meaning that they
identify their referent as some entity (via naming or anaphora resolution).

\ex.\dep{*Kim#IDENTIFICATION*}

\ex.\dep{*they#IDENTIFICATION*}

Predicate adjectives most typically denote states or qualities.

\ex.\dep{I_has-quality am *despicable#QUALITY*}

\ex.\dep{the dog_has-state is *tired#STATE*}

With attributive adjectives, the dependency relation is reversed, and the role label is changed accordingly.

\ex.\dep{despicable_m-quality *me#IDENTIFICATION*}

\ex.\dep{the tired_m-state *dog#CLASS*}

Similarly for adverbs denoting, e.g, manner (\rl{quality}) or extent (\rl{quantity}):

\ex.\dep{Kim_has-location *ran#Motion* fast_m-quality}

\ex.\dep{Kim_has-location *ran#Motion* far_m-quantity}


\newpage\subsection{Nonlocal Dependencies}
\label{sec:control}

%\todo[inline]{spell out strategies for consistent detection (xcomp, MESSAGE/SCENE frames, special cases...)}

Many constructions systematically introduce semantic predicate-dependent
dependencies that do not correspond to (surface) syntactic dependencies. In
such cases, we add those dependency links.

\ex.\dep{Kim_has-location promised Sandy to *come#LOCATION-CHANGE*} (subject control)

\ex.\dep{Kim__x-causer used a hammer to **smash#STATE-CHANGE** the vase__has-state} (subject control)

\ex.\dep{Kim persuaded Sandy_has-location to *come#LOCATION-CHANGE*} (object control)

\ex.\dep{Kim_has-location seemed to *fly#UNANCHORED-MOTION*} (raising)

\ex.\dep{Kim_x-sender entered the room *singing#MESSAGE-INIT*} (depictive)

\ex.\dep{You're talking me_has-state *silly#STATE*} (resultative)

\ex.\dep{Kim_has-location has come to *stay#LOCATION-CONTINUATION*} (subjectless adverbial clause)

\ex.\dep{Kim_x-causer left after *trashing#STATE-CHANGE* the room_has-state} (subjectless adverbial clause)

\ex.\dep{Kim_topic is hard to *love#MESSAGE*} (\emph{tough} construction)

\ex.\dep{the song_topic that_topic I_x-experiencer *like#MESSAGE*} (relative clause)

\ex.\dep{the question_topic we_x-sender raised without *answering#MESSAGE-INIT*} (parasitic gap)

\ex.\dep{ein *sogenannter#IDENTIFICATION-INIT*__m-scene___m-scene **Televisor#CLASS**_identified oder ***Hörsehschirm#CLASS***_identified} (coordination)


\newpage\subsection{Figurativity, Idiomaticity, and Uncertainty}

Difficulties in choosing frames often arise because a predicate literally evokes
one frame, but is used in a way that perhaps fits another frame equally well or
better. In such cases, annotate both the more literal frame and roles, followed
by the \texttt{>}\texttt{>} operator, followed by the more figurative frame and
roles.

\ex.\dep{A hush_in-motion>>scene *passed#UNANCHORED-MOTION>>SCENE* over the group_transitory-location>>participant}

\ex.\dep{Kim_x-sender>>participant *refused#MESSAGE-INIT>>SCENE* to eat_topic>>scene}

This mechanism can be used to indicate that an expression has become fixed and
not fully compositional:

\ex.\dep{primeval_m-time>>m-noncomp *forest#CLASS*}

\ex.\dep{colored_m-quality>>m-noncomp *pencil#CLASS*}

\ex.\dep{to *lay#LOCATION-CHANGE>>MESSAGE-DEINIT* aside_target-location>>x-noncomp my drawings_has-location>>topic}

If you cannot choose between two frames for another reason, use \texttt{||} instead of \texttt{>}\texttt{>}.


\newpage\section{Superframes Reference}

\subsection{\emoji{link} \fr{SITUATION}}
\label{sec:SITUATION}

This is the most generic superframe: something (\rl{theme}) is related to
something (\rl{situator}). Prototypically, the former is the less central, more
mobile element. It is situated in some conceptual space with respect to the
situator, or put differently: it undergoes something in connection with the
situator. When in doubt, the syntactically less oblique argument is the
theme. In more specific superframes, the theme:situator relation takes
the shape of e.g., compared:reference, has-location:location,
possessed:possessor, part:whole, follows:followed,
has-social-relation:social-relation. It can take more abstract shapes as well,
e.g. has-quality:quality, where the situator is a predicate that is true of the
theme.

This generic superframe is useful in cases where the type of relation is not
specified further.

\ex.\dep{*Yes#SITUATION*}

\ex.\dep{*No#SITUATION-NEG*}

\ex.\dep{*transition#SITUATION-CHANGE* of the account_theme to a new government_target-situator}

\ex.\dep{they_theme *need#SITUATION-NECESSITY* six months_situator for digestion_x-purpose}


\newpage\subsection{\emoji{fork-and-knife} \fr{ACCOMPANIMENT}}
\label{sec:ACCOMPANIMENT}

\rl{accompanier} accompanies \rl{accompanied}, meaning that it occurs together
with it or participates equally in the same scene.

\ex.\dep{*veggies#CLASS* with rice_m-accompanier}

\ex.\dep{The veggies_accompanied *come#ACCOMPANIMENT* with rice_accompanier}

\ex.\dep{Kim_x-causer *added#ACCOMPANIMENT-INIT* rice_target-accompanier to the veggies_accompanied}

\ex.\dep{Rolling thunder_accompanier *accompanies#ACCOMPANIMENT* the rain_accompanied}

\ex.\dep{boy_m-accompanier *king#SOCIAL-RELATION*}

Often, the accompanier denotes not the accompanying scene but an entity
participating in it, and must be metonymically understood as the scene.

\ex.\dep{Kim_has-location *cycled#LOCATION-CHANGE* to Rome_target-location with Sandy_m-accompanier}

\ex.\dep{Kim_is-active *danced#ACTIVITY* with Sandy_x-accompanier}

\ex.\dep{Kim_participant *had#SCENE* sex_scene with Sandy_x-accompanier}

\ex.\dep{Kim_x-accompanier *chased#UNANCHORED-MOTION* Sandy_in-motion around the block_transitory-location}

\ex.\dep{Kim_x-accompanier *accompanied#ACCOMPANIMENT* Sandy_accompanied}

\ex.\dep{Kim_x-accompanier *accompanied#ACCOMPANIMENT* Sandy_accompanied on the piano_x-means}


\newpage\subsection{\emoji{paintbrush} \fr{DEPICTIVE}}
\label{sec:DEPICTIVE}

Special case of \fr{ACCOMPANIMENT} where \rl{depictive} (aka \rl{accompanier})
assigns a participant of \rl{has-depictive} (aka \rl{accompanied}) a role (cf.
Sec.~\ref{sec:control}).

\ex.\dep{Kim_has-location__x-sender *entered#LOCATION-INIT* the room_target-location **singing#MESSAGE-INIT**_m-depictive}


\newpage\subsection{\emoji{money-bag} \fr{ASSET}}
\label{sec:ASSET}

In a scene \rl{has-asset}, \rl{asset} is given or offered in an exchange or wager.

\ex.\dep{Kim_target-possessor *bought#POSSESSION-CHANGE* the house_possessed for a million dollars_x-asset}

\ex.\dep{Kim_x-sender *offered#MESSAGE-INIT* Sandy_x-experiencer a million dollars_target-content for the house_x-asset}

\ex.\dep{I_x-sender *bet#MESSAGE-INIT* you_x-experiencer 30 bucks_x-asset to an apple_x-reference he will win_target-content}


\newpage\subsection{\emoji{keycap-asterisk} \fr{ATTRIBUTE}}
\label{sec:ATTRIBUTE}

In a scene \rl{has-attribute}, \rl{attribute} is the part or attribute of one
or more participants that is most directly involved in the scene. Add a
dependency link between the participant and its attribute to indicate wich
participant(s) have the attribute.

\ex.\dep{Kim_compared__has-quality *exceeds#COMPARISON* Sandy_reference__has-quality in **height#QUALITY**_x-attribute}

\ex.\dep{That_has-quality__has-quality is *great#QUALITY* in terms of **ROI#QUALITY**_m-attribute}

\ex.\dep{Kim_hitting__x-whole ist auf den **Kopf#CLASS**_x-attribute *gefallen#HITTING*}

\ex.\dep{Kim_x-causer *hit#HITTING* Sandy_hit__x-whole on the **head#CLASS**_x-attribute with a stick_hitting}


\newpage\subsection{\emoji{balance-scale} \fr{COMPARISON}}
\label{sec:COMPARISON}

\rl{compared} is characterized with respect to \rl{reference}.

Examples of comparing scenes:

\ex.\dep{Compared_m-reference to Sandy, Kim_has-quality is *tall#QUALITY*}

\ex.\dep{Sandy_has-quality is *short#QUALITY* whereas Kim is tall_m-comparison}

\ex.\dep{They_x-sender *demonize#MESSAGE-INIT* the left_topic while doing_m-reference nothing about the right}

Examples of comparing non-scene entities:

\ex.\dep{Kim_compared *outranks#COMPARISON* Sandy_reference}

\ex.\dep{Kim_compared *exceeds#COMPARISON* Sandy_reference in height_x-attribute}

\ex.\dep{The Polish restaurant_compared *compared#COMPARISON* favorably_x-quality to the Spanish one_reference}

\ex.\dep{Kim_x-experiencer *compared#COMPARISON* Coke_compared to Pepsi_reference}

\ex.\dep{kidney_m-reference *bean#CLASS*}

The \rl{reference} need not be an entity similar to the \rl{compared}, it can also be an abstract constraint:

\ex.\dep{The program_compared *conforms#COMPARISON* to the spec_reference}

\ex.\dep{Kim_compared *ran#COMPARISON-DEINIT* afoul_m-noncomp of Fielding 's constraints_reference}

We analyze gradation of adjectives as a valency-changing derivation that adds
an \rl{x-reference} argument.

\ex.\dep{more_m-quantity *isolated#SOCIAL-RELATION* than a shipwrecked sailor_x-reference}

\ex.\dep{Kim_has-quality is *taller#QUALITY* than Sandy_x-reference}


\newpage\subsection{\emoji{ok-hand} \fr{CONCESSION}}
\label{sec:CONCESSION}

Special case of \fr{COMPARISON}, where \rl{compared} is what's \rl{asserted} and \rl{reference} is what's \rl{conceded}.

\ex.\dep{Kim_has-location *went#LOCATION-CHANGE* out_target-location despite the rain_m-conceded}

\ex.\dep{It_x-noncomp *rained#STATE*, but Kim went_m-asserted out}

\ex.\dep{Kim_sender *sent#SENDING* Sandy_x-experiencer a letter_sent , but it never arrived_m-asserted}

\ex.\dep{Kim_has-location *came#LOCATION-INIT* although Sandy had told_m-conceded them not to}


\newpage\subsection{\emoji{crossed-swords} \fr{EVENT}}
\label{sec:EVENT}

Used for predicates that are inherently dynamic and cannot be framed as
\fr{-CHANGE}/\fr{-INIT}/\fr{-DEINIT}, so usually activities in terms of Vendler.

\ex.\dep{Kim_undergoer 's *adventures#EVENT* in the jungle_m-location}

\ex.\dep{Kim_x-causer *attacked#EVENT* Sandy_undergoer}

\ex.\dep{career_m-event *girl#CLASS*}

Note that many predicates that denote events in terms of Vendler can be framed differently (as changes):

\ex.\dep{Kim_excreter *sneezed#EXCRETION*}

\ex.\dep{The ambassador_has-location *arrived#LOCATION-INIT* in Moscow_target-location}


\newpage\subsection{\emoji{woman-dancing} \fr{ACTIVITY}}
\label{sec:ACTIVITY}

Special case of \fr{EVENT} where the \rl{undergoer} is active.

\ex.\dep{Kim_is-active *partied#ACTIVITY*}

\ex.\dep{Kim_is-active had *sex#ACTIVITY*}


\newpage\subsection{\emoji{sparkles} \fr{EXISTENCE}}
\label{sec:EXISTENCE}

\rl{exists} exists. Use this only for non-scene entities; for scenes, use the \fr{SCENE} frame.

\ex.\dep{I_exists *exist#EXISTENCE*}

\ex.\dep{There_x-noncomp *is#EXISTENCE* a hill_exists}

\ex.\dep{There_x-noncomp *is#SCENE* a hubbub_scene}


\newpage\subsection{\emoji{memo} \fr{REPRODUCTION}}
\label{sec:REPRODUCTION}

Special case of \fr{EXISTENCE-INIT} where \rl{original} continues to exist, and
a (modified) \rl{copy} (aka \rl{target-exists}) comes into existence.

\ex.\dep{Here is a *copy#REPRODUCTION* of the drawing_original}

\ex.\dep{This_copy is a *translation#REPRODUCTION* of the pamphlet_original into English_x-quality}


\newpage\subsection{\emoji{magic-wand} \fr{TRANSFORMATION-CREATION}}
\label{sec:TRANSFORMATION-CREATION}

Special case of \fr{EXISTENCE-INIT} where \rl{created} (aka \rl{target-exists})
is newly created from \rl{material}, or \rl{material} is transformed to become
\rl{created}.

\ex.\dep{I_x-causer succeeded in *making#TRANSFORMATION-CREATION* my first drawing_created}

\ex.\dep{Kim_x-causer *built#TRANSFORMATION-CREATION* a castle_created out of sand_material}

\ex.\dep{Kim_x-causer *turned#TRANSFORMATION-CREATION* straw_material into gold_created}


\newpage\subsection{\emoji{eye} \fr{EXPERIENCE}}
\label{sec:EXPERIENCE}

\rl{experiencer} experiences \rl{experienced}.

In connection with a \fr{MESSAGE} frame in the \rl{experienced} role, used for
sensory and mental perception as well as addressees in communication. Also use
for beneficiaries, and for ``bystander'' roles.

\ex.\dep{I_x-experiencer *saw#MESSAGE* a magnificent picture_topic}

\ex.\dep{I_x-experiencer *pondered#MESSAGE-INIT* deeply_m-quality}

\ex.\dep{Kim_x-sender *talked#MESSAGE-INIT* to Sandy_x-experiencer}

\ex.\dep{Kim_participant *did#SCENE* something_scene nice for Sandy_m-experiencer}

\ex.\dep{Kim_x-experiencer cooked a meal only to *have#SCENE* Sandy_participant spurn_scene it}

\ex.\dep{Die Piroggen_participant waren Maria_x-experiencer zu dunkel_target-scene *geraten#SCENE-INIT*}

\ex.\dep{Das_experienced hat_x-noncomp mir_experiencer gerade_x-noncomp noch_x-noncomp *gefehlt#EXPERIENCE*}

For more uses, see the examples for \fr{MESSAGE} in Section~\ref{sec:MESSAGE}.


\newpage\subsection{\emoji{red-exclamation-mark} \fr{EXPLANATION}}
\label{sec:EXPLANATION}

\rl{explanation} explains \rl{explained}, but is not a cause.

\ex.\dep{I_x-sender am *stressing#MESSAGE-INIT* this_topic because it is important_m-explanation}


\newpage\subsection{\emoji{bullseye} \fr{PURPOSE}}
\label{sec:PURPOSE}

Special case of \fr{EXPLANATION} where \rl{explanation} is a \rl{purpose}.

\ex.\dep{Kim_has-location__target-possessor *went#LOCATION-CHANGE* to town_target-location to **buy#POSSESSION-CHANGE**_m-explanation food__possessed}

\ex.\dep{**drinking#INGESTION**_m-purpose *water#CLASS*__ingested}

\ex.\dep{lamp_m-purpose *oil#CLASS*}

\ex.\dep{train_m-purpose *station#CLASS*}

\ex.\dep{buffer_m-purpose *state#STATE*}

\ex.\dep{animal_m-purpose *doctor#CLASS*}


\newpage\subsection{\emoji{passport-control} \fr{IDENTIFICATION}}
\label{sec:IDENTIFICATION}

\rl{identifier} identifies \rl{identified}.

Evoked by definite pronouns, names, and other identifiers, as well as
predicates denoting naming relationships.

\ex.\dep{*I#IDENTIFICATION* saw a picture}

\ex.\dep{I can distinguish *China#IDENTIFICATION* from Arizona}

\ex.\dep{a book_identified *called#IDENTIFICATION* True Stories_identifier from Nature}

\ex.\dep{This_identified is *Kim#IDENTIFICATION*}

\ex.\dep{Obamas *Sonderberaterin#SOCIAL-RELATION* Kori Schulman_m-identifier}

In English, the preposition \emph{of} has an identifying sense, which can also
be metaphorical:

\ex.\dep{the *island#CLASS* of Pultanella_m-identifier}

\ex.\dep{the *stallion#CLASS* of Rumour_m-identifier}

Likewise, \emph{in} has an identifying sense:

\ex.\dep{In answer_m-identifier , he_x-sender *repeated#MESSAGE-INIT* : Please, draw_target-content me a sheep !}


\newpage\subsection{\emoji{round-pushpin} \fr{LOCATION}}
\label{sec:LOCATION}

Describes \rl{has-location} as located or moving wrt. respect to \rl{location}.

\ex.\dep{the *hat#CLASS* in the box_m-location}

\ex.\dep{Kim_has-location *lives#LOCATION* in Boston_location}

\ex.\dep{Kim_has-location *went#LOCATION-CHANGE* from the living room_initial-location through the door_transitory-location into the kitchen_target-location}

\ex.\dep{Kim_x-causer *placed#LOCATION-CHANGE* the hat_has-location on the table_target-location}

\ex.\dep{house_m-location *music#MESSAGE*}

\ex.\dep{music_m-has-location *hall#CLASS*}

\ex.\dep{sugar_m-has-location *cane#CLASS*}

\ex.\dep{cane_m-initial-location *sugar#CLASS*}


\newpage\subsection{\emoji{gem-stone} \fr{ADORNMENT-TARNISHMENT}}
\label{sec:ADORNMENT-TARNISHMENT}

Special case of \fr{LOCATION} where \rl{ornament} (aka \rl{has-location}) sits on \rl{surface} (aka \rl{location}).

\dep{Kim_x-causer *decorated#ADORNMENT-TARNISHMENT* the balcony_surface with fairy lights_ornament}

\dep{Kim_x-causer *splashed#ADORNMENT-TARNISHMENT-INIT* Sandy_surface with water_ornament}

\dep{Kim_x-causer *washed#ADORNMENT-TARNISHMENT-DEINIT* the dirt_ornament off Sandy_initial-surface}

\dep{Kim_x-causer *washed#ADORNMENT-TARNISHMENT-DEINIT* Sandy_initial-surface}


\newpage\subsection{\emoji{sweat-droplets} \fr{EXCRETION}}
\label{sec:EXCRETION}

Special case of \fr{LOCATION-DEINIT} where \rl{excreter} (aka
\rl{initial-location}) excretes \rl{excreted} (aka \rl{has-location}).

\ex.\dep{Kim_excreter *threw#EXCRETION* up the pretzel_excreted}


\newpage\subsection{\emoji{cricket-game} \fr{HITTING}}
\label{sec:HITTING}

Special case of \fr{LOCATION-INIT} where \rl{hitting} (aka \rl{has-location})
comes into contact with \rl{hit} (aka \rl{target-location}).

\ex.\dep{Kim_x-causer *hit#HITTING* Sandy_hit}

\ex.\dep{Kim_x-causer *hit#HITTING* Sandy_hit with a stick_hitting}

\ex.\dep{The stick_hitting *hit#HITTING* Sandy_hit}

\ex.\dep{Kim_x-causer *hit#HITTING* Sandy_hit__x-whole on the **head#CLASS**_x-attribute with a pool noodle_hitting}

\ex.\dep{Kim_x-causer *kicked#HITTING* Sandy_hit}


\newpage\subsection{\emoji{bowl-with-spoon} \fr{INGESTION}}
\label{sec:INGESTION}

Special case of \fr{LOCATION-INIT} where \rl{ingester} (aka
\rl{target-location}) ingests \rl{ingested} (aka \rl{has-location}).

\ex.\dep{Kim_ingester *ate#INGESTION* an apple_ingested}

\ex.\dep{Kim_ingester *nibbled#INGESTION* on the pretzel_ingested}


\newpage\subsection{\emoji{leaf-fluttering-in-wind} \fr{UNANCHORED-MOTION}}
\label{sec:UNANCHORED-MOTION}

Special case of \rl{LOCATION-CHANGE} where no initial or target location is indicated.

\ex.\dep{Kim_in-motion is *running#UNANCHORED-MOTION* along the river_transitory-location}

\ex.\dep{I_x-causer learned to *pilot#UNANCHORED-MOTION* airplanes_in-motion}

\ex.\dep{Kim_in-motion is *dancing#UNANCHORED-MOTION* around the room_transitory-location with Sandy_m-accompanier}

\ex.\dep{Kim_in-motion is an avid_m-quality *unicyclist#UNANCHORED-MOTION*}

%\todo[inline]{define clearly when dancing etc. is UNANCHORED-MOTION and when it is ACTIVITY}


\newpage\subsection{\emoji{t-shirt} \fr{WRAPPING-WEARING}}
\label{sec:WRAPPING-WEARING}

Special case of \fr{LOCATION} where \rl{wearer} (aka \rl{location}) wears or is
wrapped in \rl{wrapper} (aka \rl{has-location}).

\ex.\dep{Kim_wearer is *wearing#WRAPPING-WEARING* a shirt_wrapper}

\ex.\dep{Kim_wearer is *wearing#WRAPPING-WEARING* glasses_wrapper}

\ex.\dep{The shroud_wrapper *wraps#WRAPPING-WEARING* the scepter_wearer}

\ex.\dep{Kim_target-wearer *put#WRAPPING-WEARING-INIT* on a sweater_wrapper}

\ex.\dep{Kim_initial-wearer *took#WRAPPING-WEARING-DEINIT* off their glasses_wrapper}


\newpage\subsection{\emoji{hammer} \fr{MEANS}}
\label{sec:MEANS}

\rl{has-means} is a scene caused by something via an intermediary \rl{means}.

\ex.\dep{Kim_x-causer *cut#STATE-CHANGE* the cake_has-state with a knife_m-means}

\ex.\dep{Kim_x-causer *painted#ADORNMENT-TARNISHMENT* the room_surface by exploding_m-means a paint bomb}

\ex.\dep{Kim_x-causer__x-causer *used#MEANS* a pen_means to **get#LOCATION-DEINIT**_has-means the lid__has-location off__initial-location}

\ex.\dep{You_x-causer *used#MEANS* me_means !}

\ex.\dep{oil_m-means *lamp#CLASS*}


\newpage\subsection{\emoji{left-speech-bubble} \fr{MESSAGE}}
\label{sec:MESSAGE}

A message about \rl{topic} with content \rl{content} exists in perceived,
measured, or recorded recorded form. When a message is created through
expression or observation, use \fr{MESSAGE-INIT}. When \rl{content} and
\rl{topic} are both realized, \rl{content} must assign a role to \rl{topic}.

Predicates of expression use \fr{MESSAGE-INIT}:

\ex.\dep{Kim_x-sender *yelped#MESSAGE-INIT*}

\ex.\dep{Kim_x-sender *said#MESSAGE-INIT* : it 's fine_target-content}

\ex.\dep{Kim_x-sender *said#MESSAGE-INIT* it was fine_target-content}

\ex.\dep{Kim_x-sender *called#MESSAGE-INIT* Sandy_topic__x-sender a **liar#MESSAGE**_target-content}

\ex.\dep{Kim_x-sender *told#MESSAGE-INIT* Sandy_x-experiencer a secret_target-content}

\ex.\dep{Kim_x-sender *talked#MESSAGE-INIT* about Sandy_topic}

\ex.\dep{Kim_x-sender *talked#MESSAGE-INIT* **shit#MESSAGE**_target-content about Sandy_topic__topic}

\ex.\dep{Kim_x-sender and Sandy_x-sender *conversed#MESSAGE-INIT*}

\ex.\dep{Kim_x-sender *conversed#MESSAGE-INIT* with Sandy_x-accompanier}

Gesture is a kind of expression, too:

\ex.\dep{Kim_x-sender *curtseyed#MESSAGE-INIT* to the Queen_x-experiencer}

\ex.\dep{Kim_x-causer>>x-sender *shook#UNANCHORED-MOTION>>MESSAGE-INIT* their head_in-motion>>x-noncomp no_x-sent>>target-content}

Performance of a work of art is framed as \fr{MESSAGE} where the work of art is the \rl{topic}:

\ex.\dep{Kim_x-sender *played#MESSAGE-INIT* a little tune_topic on their tuba_x-means}

\ex.\dep{They_x-sender *performed#MESSAGE-INIT* the play_topic}

\ex.\dep{Kim_x-sender *sang#MESSAGE-INIT* a song_topic}

What is depicted gets the \rl{topic} role:

\ex.\dep{Kim_x-sender *drew#MESSAGE-INIT* a heron_topic}

\ex.\dep{a *picture#MESSAGE* of the heron_topic}

\ex.\dep{The concert_topic was *recorded#MESSAGE-INIT* on tape_x-target-location}

Recordings of information are framed as messages:

\ex.\dep{history_topic *book#MESSAGE*}

\ex.\dep{a *book#MESSAGE* about the primeval forest_topic}

\ex.\dep{sales_topic *target#MESSAGE*}

The result of recording something gets the \rl{target-content} role:

\ex.\dep{Kim_x-sender *drew#MESSAGE-INIT* a picture_target-content}

\ex.\dep{Kim_x-sender *wrote#MESSAGE-INIT* Sandy_x-experiencer a letter_target-content}

\ex.\dep{Kim_x-sender *wrote#MESSAGE-INIT* the message_target-content__has-quality onto a piece_x-target-location of paper with a pen_m-means in big red **letters#QUALITY**_x-depictive}

\ex.\dep{The band_x-sender *recorded#MESSAGE-INIT* an album_target-content}

Predicates of perception use \fr{MESSAGE}, including mental perception:

\ex.\dep{Kim_x-experiencer *saw#MESSAGE* a flower_topic}

\ex.\dep{Kim_x-experiencer *found#MESSAGE* the flower_topic__has-quality **beautiful#QUALITY**_content}

\ex.\dep{Kim_x-experiencer *thinks#MESSAGE* Sandy is a liar_content}

\ex.\dep{Kim_x-experiencer *thinks#MESSAGE* Sandy_topic__x-sender a **liar#MESSAGE**_content}

\ex.\dep{Kim_x-experiencer *saw#MESSAGE* Sandy_topic__in-motion **swim#UNANCHORED-MOTION**_content}

\ex.\dep{Kim_x-experiencer__in-motion *wants#MESSAGE* to **swim#UNANCHORED-MOTION**_content}

\ex.\dep{Kim_x-experiencer *wants#MESSAGE* Sandy_topic__in-motion to **swim#UNANCHORED-MOTION**_content}

\ex.\dep{Kim_topic__x-experiencer *seems#MESSAGE* **happy#MESSAGE**_content}

\ex.\dep{Kim_topic__x-experiencer *seems#MESSAGE* **happy#MESSAGE**_content to Sandy_x-experiencer}

\ex.\dep{Sandy_x-experiencer is a *professor#MESSAGE* of linguistics_topic}

\ex.\dep{sun_topic *worship#MESSAGE*}

\ex.\dep{They_x-experiencer *revered#MESSAGE* God_topic}

Predicates that denote the initiation of perception (e.g., by acquiring
knowledge, or observation, or reasoning), use \fr{MESSAGE-INIT}:

\ex.\dep{The Thought Police_x-experiencer *observed#MESSAGE-INIT* Winston_topic}

\ex.\dep{Kim_x-experiencer *studies#MESSAGE-INIT* linguistics_topic}

\ex.\dep{Kim_x-experiencer *noticed#MESSAGE-INIT* the bird_topic}

\ex.\dep{Kim_x-sender *taught#MESSAGE-INIT* Sandy_x-experiencer Spanish_topic}

\ex.\dep{Kim_x-experiencer *measured#MESSAGE-INIT* the elasticity_topic}

\ex.\dep{The jury_x-experiencer *found#MESSAGE-INIT* Kim_topic__participant___is-active **guilty#SCENE**_target-content of the ***crime#ACTIVITY***__scene}

Predicates that denote the deinitiation of perception use \fr{MESSAGE-DEINIT}:

\ex.\dep{Kim_x-experiencer *forgot#MESSAGE-DEINIT* everything they knew_initial-content}

\ex.\dep{Kim_x-experiencer *forgot#MESSAGE-DEINIT* about the cake_topic}

And finally, perception (here: remembering something) that was meant to happen
but didn't is framed as \fr{MESSAGE-PREVENTION}:

\ex.\dep{Kim_x-experiencer *forgot#MESSAGE-PREVENTION* to take_target-content the trash out}


\newpage\subsection{\emoji{red-question-mark} \fr{MODE}}
\label{sec:MODE}

Used for adverbial modifiers that have no arguments other than the phrase they
modify, and that, rouhgly speaking, indicate the modal strength of what is
expressed and/or its relation to the discourse.

\ex.\dep{Even_m-mode *Kim#IDENTIFICATION* did n't know that}

\ex.\dep{They_x-causer only_m-mode *rinsed#ADORNMENT-TARNISHMENT-DEINIT* the dishes_initial-surface}

\ex.\dep{*Passt#COMPARISON* das_compared eh_m-mode ?}

\ex.\dep{Kim_x-experiencer probably_m-mode *knows#MESSAGE* that_content}

\ex.\dep{That_has-quality 's really_m-mode *great#QUALITY*}

\ex.\dep{Kim_has-location is not_m-mode *here#LOCATION*}


\newpage\subsection{\emoji{spider-web} \fr{NONCOMP}}
\label{sec:NONCOMP}

Used to mark syntactic arguments that are thought of as part of the predicate, as in verbal idioms, weather verbs, inherently reflexive verbs, existential \emph{there}, or other fixed expressions.

\ex.\dep{Kim_destroyed *kicked#DESTRUCTION* the bucket_x-noncomp}

\ex.\dep{It_x-noncomp is *raining#STATE*}

\ex.\dep{I_x-sender *address#MESSAGE-INIT* myself_x-noncomp to you_x-experiencer}

\ex.\dep{There_x-noncomp *was#SCENE* a famine_scene}

\ex.\dep{fountain_m-noncomp *pen#CLASS*}

Light verbs, on the other hand, are treated with \fr{SCENE}, see Section~\ref{sec:SCENE}.


\newpage\subsection{\emoji{puzzle-piece} \fr{PART-WHOLE}}
\label{sec:PART-WHOLE}

\rl{part} is part of \rl{whole}.

\ex.\dep{Kim_m-whole 's *leg#CLASS*}

\ex.\dep{a *man#CLASS* with a mustache_m-part}

\ex.\dep{*part#PART-WHOLE* of the year_whole}

\ex.\dep{wheat_whole *contains#PART-WHOLE* gluten_part}

\ex.\dep{orange_m-whole *seed#CLASS*}

\ex.\dep{seed_m-part *orange#CLASS*}

\ex.\dep{car_m-whole *motor#CLASS*}

\ex.\dep{motor_m-part *car#CLASS*}

\ex.\dep{cube_m-whole *sugar#CLASS*}

\ex.\dep{sugar_m-part *cube#CLASS*}


\newpage\subsection{\emoji{light-bulb} \fr{EXAMPLE}}
\label{sec:EXAMPLE}

Special case of \fr{PART-WHOLE} where \fr{example} (aka \fr{part}) is given as an example of \fr{exemplified} (aka \fr{whole}).

\ex.\dep{*birds#CLASS* such as storks_m-example}

\ex.\dep{Sie_x-experiencer *lernen#MESSAGE-INIT*__example **beispielsweise#EXAMPLE**_m-scene wissenschaftliche Methoden_target-content}


\newpage\subsection{\emoji{service-dog} \fr{POSSESSION}}
\label{sec:POSSESSION}

\rl{possessor} possesses or controls the \rl{possessed}.

\ex.\dep{Kim_m-possessor 's *house#CLASS*}

\ex.\dep{Kim_possessor *owns#POSSESSION* a house_possessed}

\ex.\dep{The house_possessed *belongs#POSSESSION* to Kim_possessor}

\ex.\dep{the *owner#POSSESSION* of the house_possessed}

\ex.\dep{Kim_possessor *has#POSSESSION* Sandy 's phone_possessed}

\ex.\dep{Kim_target-possessor *bought#POSSESSION-CHANGE* a house_possessed from Sandy_initial-possessor}

\ex.\dep{Sandy_initial-possessor *sold#POSSESSION-CHANGE* Kim_target-possessor the house_possessed}

\ex.\dep{Kim_initial-possessor *kept#POSSESSION-CONTINUATION* the house_possessed}

\ex.\dep{Kim_initial-possessor *lost#POSSESSION-DEINIT* the house_possessed}

\ex.\dep{Caesar_target-possessor *conquered#POSSESSION-INIT* Gaul_possessed}

\ex.\dep{Caesar_target-possessor 's *conquest#POSSESSION-INIT* of Gaul_possessed}

\ex.\dep{Kim_initial-possessor *owes#POSSESSION-CHANGE-NECESSITY* Sandy_target-possessor money_possessed}

\ex.\dep{family_m-possessor *estate#CLASS*}


\newpage\subsection{\emoji{abacus} \fr{QUANTITY}}
\label{sec:QUANTITY}

\rl{quantity} is the quantity, degree, or extent of \rl{has-quantity}.

\ex.\dep{three_m-quantity *burgers#CLASS*}

\ex.\dep{three_m-quantity *liters#QUANTITY* of coke_has-quantity}

\ex.\dep{We_x-sender *discourage#MESSAGE-INIT* this_topic emphatically_m-quantity}


\newpage\subsection{\emoji{1st-place-medal} \fr{RANK}}
\label{sec:RANK}

\rl{rank} indicates the order that \rl{has-rank} has in some sequence.

\ex.\dep{*Chapter#MESSAGE* 1_m-rank}

\ex.\dep{my_m-sender first_m-rank *drawing#MESSAGE*}


\newpage\subsection{\emoji{performing-arts} \fr{SCENE}}
\label{sec:SCENE}

A ``meta'' frame for predicates where the main frame is invoked by \rl{scene},
and the predicate adds some temporal, aspectual, modal, etc., meaning, or just
acts as a light verb. If there is a \rl{participant}, it is assigned a role by
\rl{scene}, which needs an extra dependency link. In the following examples, we
show the annotations for both the matrix predicate and the embedded predicate
in one graph.

\ex.\dep{The **concert#MESSAGE-INIT**_target-scene *began#SCENE-INIT*}

\ex.\dep{The **concert#MESSAGE-INIT**_initial-scene *continued#SCENE-CONTINUATION*}

\ex.\dep{The **concert#MESSAGE-INIT**_initial-scene *finished#SCENE-DEINIT*}

\ex.\dep{The **shouting#MESSAGE-INIT**_initial-scene *intensified#SCENE-CONTINUATION*}

\ex.\dep{The **shouting#MESSAGE-INIT**_initial-scene *faded#SCENE-DEINIT*}

\ex.\dep{A **coup#EVENT**_target-scene was *attempted#SCENE-INIT*}

\ex.\dep{Kim_participant__is-active *finished#SCENE-DEINIT* their **work#ACTIVITY**_initial-scene}

\ex.\dep{Swift action_x-causer *prevented#SCENE-PREVENTION* an **outbreak#SCENE-INIT**_target-scene of ***measles#EVENT***__target-scene}

\ex.\dep{Kim_participant__has-location *refrained#SCENE-PREVENTION* from **going#LOCATION-CHANGE**_target-scene}

\ex.\dep{Kim_x-causer *prevented#SCENE-PREVENTION* Sandy_participant__has-location from **going#LOCATION-CHANGE**_target-scene}

\ex.\dep{Kim_x-causer *saved#SCENE-PREVENTION* Sandy_participant__x-experiencer from the **dragon#CLASS**_target-scene}

\ex.\dep{Kim_participant__is-active *plays#SCENE* **tennis#ACTIVITY**_scene}

\ex.\dep{Kim_participant__participant___is-active *used#SCENE* to **play#SCENE**_scene ***tennis#ACTIVITY***__scene}

\ex.\dep{Kim_participant_x-causer *gave#SCENE* Sandy_participant__hit a **kick#HITTING**_scene}

The modifier relation \rl{m-scene} is used when a syntactic dependeny points
from an argument to a predicate, as, e.g., with relative clauses or sentence
adverbs.

\ex.\dep{the *clown#CLASS*__topic I__x-experiencer **saw#MESSAGE**_m-scene smiled}

\ex.\dep{**Fortunately#EXPERIENCE**_m-scene for Sandy__experiencer , Kim_has-location is *here#LOCATION*__experienced}

\ex.\dep{I_x-causer *devoted#SCENE-INIT*__follows myself_participant **instead#SEQUENCE**_m-scene to geography_target-scene}


\newpage\subsection{\emoji{zzz} \fr{STATE}}
\label{sec:STATE}

\rl{state} indicates a state of \rl{has-state}. Typically used with predicates
that do not, in fact, have a \rl{state} role, because the state is already
incorporated into the meaning of the predicate.

\ex.\dep{when I_has-state was six years_x-quantity *old#STATE*}

\ex.\dep{Boa constrictors swallow their prey_has-state *whole#STATE*}

\ex.\dep{they_has-state *sleep#STATE*}

\ex.\dep{they_x-causer swallow their prey whole without *chewing#STATE-CHANGE* it_has-state}

\ex.\dep{the six months that they_x-causer need for *digestion#STATE-CHANGE*}

\ex.\dep{And that_x-causer hasn't much *improved#STATE-CHANGE* my opinion_has-state of them}


\newpage\subsection{\emoji{red-apple} \fr{QUALITY}}
\label{sec:QUALITY}

Special case of \fr{STATE} -- a quality is a bit more permanent than a state;
the \rl{has-quality} (aka \rl{has-state}) is not expected to change back and
forth between \rl{quality}s (aka \rl{state}s) regularly.  Also used to describe
qualitites of events, i.e., manners.

\ex.\dep{a magnificent_m-quality *picture#MESSAGE*}

\ex.\dep{I_x-experiencer *pondered#MESSAGE-INIT* deeply_m-quality over the adventures_topic of the jungle}

\ex.\dep{a skilled_m-quality *surgeon#CLASS*}

\ex.\dep{such_m-quality *knowledge#MESSAGE* is valuable}

\ex.\dep{The leaves_has-quality *reddened#QUALITY-INIT*}


\newpage\subsection{\emoji{deciduous-tree} \fr{CLASS}}
\label{sec:CLASS}

Special case of \fr{QUALITY} -- a class is even more permanent, in the sense
that if the \rl{has-class} (aka \rl{has-state}) takes on a new \rl{class} (aka
\rl{state}), it becomes a new kind of entity.

Most prototypically evoked by common nouns with no arguments.

\ex.\dep{swallowing an animal#CLASS}

\ex.\dep{Kim_participant__has-class *became#SCENE-INIT* a **teacher#CLASS**_target-scene}

Indefinite pronouns also evoke \fr{CLASS}.

\ex.\dep{She saw *one#CLASS*}

\ex.\dep{*Nothing#CLASS* about him_m-whole suggested a child}

\ex.\dep{Why would *anyone#CLASS* be frightened by a hat?}

\ex.\dep{*Something#CLASS* is broken}

\ex.\dep{Where I live *everything#CLASS* is small}


\newpage\subsection{\emoji{skull} \fr{DESTRUCTION}}
\label{sec:DESTRUCTION}

Special case of \fr{STATE-CHANGE} where \rl{destroyed} (aka \rl{has-state}) goes out of existence.

\ex.\dep{Sam_destroyed 's *death#DESTRUCTION*}

\ex.\dep{Sam_x-causer 's *destruction#DESTRUCTION* of the city_destroyed}

When something is broken but not completely destroyed, use \fr{STATE}.

\ex.\dep{Something_has-state was *broken#STATE* in my enginge_m-location}


\newpage\subsection{\emoji{megaphone} \fr{SENDING}}
\label{sec:SENDING}

\rl{sender} originates a message, \rl{sent}, that can be experienced.

\ex.\dep{According to Kim_m-sender , it_x-noncomp is *raining#STATE*}

\ex.\dep{song_sent *bird#CLASS*}

\ex.\dep{bird_sender *song#MESSAGE*}

For more uses, see \fr{MESSAGE} (Section~\ref{sec:MESSAGE}).


\newpage\subsection{\emoji{right-arrow} \fr{SEQUENCE}}
\label{sec:SEQUENCE}

\rl{follows} follows \rl{followed}, e.g., temporally, logically, by rank, as heir, etc.

\ex.\dep{Form_follows *follows#SEQUENCE* function_followed}

\ex.\dep{Cook_follows is Jobs_followed 's *successor#SEQUENCE*}

\ex.\dep{Das_follows *fußt#SEQUENCE* auf einer falschen Vorstellung_followed}

\ex.\dep{Kim_x-experiencer *deduced#SEQUENCE* the truth_follows from the clues_followed}

\ex.\dep{Given that I 'm tired_m-followed , I_has-location wo n't be *there#LOCATION*}

Also used to indicate proportional amounts: for each scoop (\rl{followed}), it costs 1 euro (\rl{follows}).

\ex.\dep{It costs 1 *euro#QUANTITY* per scoop_m-followed}


\newpage\subsection{\emoji{joystick} \fr{CAUSATION}}
\label{sec:CAUSATION}

Special case of \fr{SEQUENCE} where \rl{causer} (aka \rl{followed}) causes \rl{result} (aka \rl{follows}).

\ex.\dep{Kim_x-causer *broke#STATE-CHANGE* the glass_has-state}

\ex.\dep{The knife_x-causer *cut#STATE-CHANGE* the bread_has-state}

\ex.\dep{Kim_x-causer *cut#STATE-CHANGE* the bread_has-state with a knife_m-means}

\ex.\dep{The war_causer *caused#CAUSATION* a famine_result}

\ex.\dep{There_x-noncomp *was#SCENE* a famine_scene because of the war_m-causer}

\ex.\dep{Der Wasserdruck_has-quantity *stieg#QUANTITY-CHANGE*, wodurch der Brunnen überfloss_m-result}

\ex.\dep{Die Qualität_result ist der Motivation_causer *geschuldet#CAUSATION*}

\ex.\dep{tear_m-result *gas#CLASS*}

\ex.\dep{sun_m-causer *burn#STATE-CHANGE*}

\ex.\dep{honey_m-result *bee#CLASS*}

\ex.\dep{Kim_has-location *went#LOCATION-CHANGE* to town_target-location because they wanted_m-causer to buy food}

Note how the last example expresses a purpose, but expresses it as a cause, so
\rl{m-causer} lis the right label to use. Compare this to construal as a
purpose:

\ex.\dep{Kim_has-location *went#LOCATION-CHANGE* to town_target-location to buy_m-purpose food}


\newpage\subsection{\emoji{scroll} \fr{CONDITION}}
\label{sec:CONDITION}

Special case of \fr{SEQUENCE} where \rl{condition} (aka \rl{followed}) is a
condition to \rl{has-condition} (aka \rl{follows}).

\ex.\dep{I_has-social-relation will *join#SOCIAL-RELATION-INIT* the club if they ask_m-condition me}

\ex.\dep{The start date_has-condition is *contingent#CONDITION* on their approval_condition}

\ex.\dep{Eine Aussöhung_has-condition *bedingt#SEQUENCE* eine Entschuldigung_condition}


\newpage\subsection{\emoji{prohibited} \fr{EXCEPTION}}
\label{sec:EXCEPTION}

Special case of \fr{SEQUENCE} where \rl{exception} (aka \rl{followed}) is an
exception (a negative condition, if you will) to \rl{has-exception} (aka \rl{follows}).

\ex.\dep{Except for Kim_m-exception , everybody_has-social-relation *joined#SOCIAL-RELATION-INIT*}


\newpage\subsection{\emoji{collision} \fr{REACTION}}
\label{sec:REACTION}

Special case of \fr{CAUSATION} where \rl{trigger} (aka \rl{causer}) triggers a
\rl{reaction} (aka \rl{result}) in the \rl{x-causer}.

\ex.\dep{Kim_x-causer__x-sender *reacted#SEQUENCE* to the allegations_trigger with a **denial#MESSAGE-INIT**_reaction}


\newpage\subsection{\emoji{sneezing-face} \fr{RESULTATIVE}}
\label{sec:RESULTATIVE}

Special case of \fr{CAUSATION} where \rl{resultative} (aka \rl{result}) assigns
an argument of \rl{has-resultative} (aka \rl{causer}) a role. We treat the
English resultative construction as a valency-changing operation that adds one
or two arguments to the matrix predicate, so we use \rl{x-resultative} rather
than \rl{m-resultative}.

\ex.\dep{Kim_x-causer *hammered#HITTING* the metal_hit__has-state **flat#STATE**_x-resultative}

\ex.\dep{Kim_x-causer *painted#ADORNMENT-TARNISHMENT* the room_surface__has-quality **red#QUALITY**_x-resultative}

\ex.\dep{Kim_excreter *sneezed#EXCRETION* the napkin_x-resultative__x-initial-location:has-location off the **table#CLASS**_x-resultative}

In the last example, we use \rl{x-initial-location:has-location} to specify not
only the role of the napkin in the resulting event (\rl{has-location}) but also
that of the table (\rl{initial-location}). Using \rl{x-has-location} would be
imprecise because we would then assume that the table has \rl{location}.


\newpage\subsection{\emoji{handshake} \fr{SOCIAL-RELATION}}
\label{sec:SOCIAL-RELATION}

\rl{has-social-relation} is an individual that is in some socially constructed
relationship with \rl{social-relation}. \rl{social-relation} might, e.g., be a
relative, a friend, an organization, a responsibility, or a judicial sentence.

\ex.\dep{Kim_has-social-relation 's *friend#SOCIAL-RELATION*}

\ex.\dep{Kim_social-relation is my_has-social-relation *cousin#SOCIAL-RELATION*}

\ex.\dep{Kim_social-relation and Sandy are *friends#SOCIAL-RELATION*}

\ex.\dep{Kim_has-social-relation is *friends#SOCIAL-RELATION* with Sandy_social-relation}

\ex.\dep{Kim_has-social-relation *works#SOCIAL-RELATION* at Google_social-relation}

\ex.\dep{Kim_has-social-relation *works#SOCIAL-RELATION* for Sandy_social-relation}

\ex.\dep{Kim_has-social-relation *emcees#SOCIAL-RELATION*}

\ex.\dep{Kim_has-social-relation is *hosting#SOCIAL-RELATION* the party_social-relation}

\ex.\dep{Kim_has-social-relation is under house *arrest#SOCIAL-RELATION*}

\ex.\dep{Kim_has-social-relation 's *sentence#SOCIAL-RELATION* was suspended}

\ex.\dep{Kim_has-social-relation *married#SOCIAL-RELATION-INIT* Sandy_target-social-relation}

\ex.\dep{The official_x-causer *married#SOCIAL-RELATION-INIT* Kim_has-social-relation to Sandy_target-social-relation}

\ex.\dep{The official_x-causer *married#SOCIAL-RELATION-INIT* Kim_has-social-relation and Sandy}

\ex.\dep{Kim_has-social-relation *divorced#SOCIAL-RELATION-DEINIT* Sandy_initial-social-relation}

\ex.\dep{Kim_has-social-relation *befriended#SOCIAL-RELATION-INIT* Sandy_target-social-relation}

\ex.\dep{Kim_has-social-relation *took#SOCIAL-RELATION-INIT* the job_target-social-relation}

\ex.\dep{Kim_has-social-relation *joined#SOCIAL-RELATION-INIT* Google_target-social-relation}

\ex.\dep{Kim_has-social-relation *joined#SOCIAL-RELATION-INIT* a union_target-social-relation}

\ex.\dep{Sandy_x-causer *fired#SOCIAL-RELATION-DEINIT* Kim_has-social-relation from their job_initial-social-relation}

\ex.\dep{Kim_has-social-relation *left#SOCIAL-RELATION-DEINIT* Google_initial-social-relation}

\ex.\dep{Kim_has-social-relation *assumed#SOCIAL-RELATION-INIT* office_target-social-relation}

\ex.\dep{The judge_x-causer *sentenced#SOCIAL-RELATION-INIT* Kim_has-social-relation to three days_target-social-relation in prison}

\ex.\dep{Kim_has-social-relation was *pardoned#SOCIAL-RELATION-DEINIT*}


\newpage\subsection{\emoji{alarm-clock} \fr{TIME}}
\label{sec:TIME}

\rl{time} indicates when, how often, or for how long \rl{has-time} takes place. Also evoked by time expressions without arguments.

\ex.\dep{Kim_in-motion *swims#UNANCHORED-MOTION* on Monday_m-time}

\ex.\dep{Kim_excreter *sneezed#EXCRETION* twice_m-time}

\ex.\dep{Kim_in-motion *swam#UNANCHORED-MOTION* for an hour_m-time}

\ex.\dep{Kim_x-sender *says#MESSAGE-INIT* hello_content whenever I meet_m-time them}

\ex.\dep{*Once#TIME* when I was six years old}

\ex.\dep{summer_m-time *job#ACTIVITY*}

\ex.\dep{golf_has-time *season#TIME*}


\newpage\section{Argument Structure and Frame Choice}

\subsection{Prefer Core over Non-core Arguments}

When an argument fills both a core and a non-core role, it is more important to
annotate the former.

\ex.\dep{Kim_has-location *drove#LOCATION-CHANGE* to Boston_target-location}

\ex.\dep{Kim_x-causer *drove#LOCATION-CHANGE* the car_has-location to Boston_target-location}

\ex.\dep{They_target-possessor *plundered#POSSESSION-CHANGE* Rome_initial-possessor}

\ex.\dep{Kim_initial-wearer *undressed#WRAPPING-WEARING-DEINIT*}

Also, when in doubt, choose the frame so that you can use core roles rather
than resorting to non-core roles. For example, in the following sentence, we
should use \fr{LOCATION-INIT} rather than \fr{UNANCHORED-MOTION} so that we can
use \rl{target-location} and do not have to resort to \fr{x-target-location}.

\ex.\dep{Kim_x-accompanier *followed#LOCATION-INIT* Sandy_has-location into the room_target-location}


\newpage\subsection{Arguments Determine Frames}

The most important criterion in choosing a frame for a predicate is that there
should be suitable roles for the predicate's arguments, even if they are
unrealized (implicit) in the annotated instance. For example, while
\emph{drawing} denotes a \fr{CLASS} of things, it can occur with a
prepositional argument denoting a \rl{topic}, so \fr{MESSAGE} is a better
choice.

\ex.
\a.\dep{my_m-sender first_m-rank *drawing#MESSAGE*}
\b.\dep{my_m-sender first_m-rank *drawing#MESSAGE* of a snake_topic}

\ex.
\a.\dep{Kim_x-causer *helped#SCENE-INIT* Sandy_participant}
\b.\dep{Kim_x-causer *helped#SCENE-INIT* Sandy_participant clean_target-scene the dishes}

\ex.
\a.\dep{Kim_x-causer *worked#EVENT*}
\b.\dep{Kim_x-causer *worked#EVENT* on the drawing_undergoer}

For nouns, you have to decide whether they are nonrelational nouns (\fr{CLASS}) or relation/event nouns. A useful test is to try and add an argument, i.e., a dependent that is assigned a specific role by the noun. For example:

\ex.\dep{a *tree#CLASS*}

\ex.\dep{*venue#LOCATION* of the event_has-location}

\ex.\dep{Kim_has-social-relation 's *brother#SOCIAL-RELATION*}

Note that in \emph{Kim 's tree}, Kim's role is that of \rl{possessor}, but it is not assigned by the noun \emph{tree} but by the possessive construction, so \emph{tree} is still \fr{CLASS} and we annotate \emph{Kim} as a modifier.

\ex.\dep{Kim_m-0possessor 's *tree#CLASS*}


\newpage\subsection{Shadow and Default Arguments}

Arguments that determine a predicate's superframe include \emph{shadow
arguments} and \emph{default arguments}
\citep{pustejovsky-1995-generative,di-fabio-etal-2019-verbatlas}, i.e.,
arguments that do not appear in the syntactic argument structure because they
are incorporated into the predicate or logically implied, like the bones in
\ref{ex:debone}, mucus and air in \ref{ex:sneeze}, groceries in
\ref{ex:deliver}, or sun in \ref{ex:rise}.

\ex.\label{ex:debone}\dep{Kim_x-causer *deboned#PART-WHOLE-DEINIT* the fish_initial-whole}

\ex.\label{ex:sneeze}\dep{Kim_excreter *sneezed#EXCRETION*}

\ex.\label{ex:deliver}\dep{Our local supermarket_x-causer *delivers#LOCATION-INIT*}

\ex.\label{ex:rise}\dep{at *sunrise#LOCATION-CHANGE>>TIME*}


\newpage\subsection{Predicates that Refer to a Shadow Argument}

A special case of shadow argument are those that the predicate itself refers
to. For example, the predicate \emph{friend} evokes a \fr{SOCIAL-RELATION}
frame, but also refers to the filler of that frame's \rl{social-relation} role.
And the predicate \emph{model} evokes a \fr{MESSAGE} frame, but also refers to
the filler of that frame's \rl{topic} role, and so on.

\ex.\dep{Kim_has-social-relation 's *friend#SOCIAL-RELATION*}

\ex.\dep{the drawing and its_content *model#MESSAGE*}

\ex.\dep{ein *Großteil#PART-WHOLE* des digitalen Übergangs_whole}

\ex.\dep{Obama_social-relation special_m-quality>>m-noncomp *assistant#SOCIAL-RELATION*}


\newpage\subsection{A Participant whose Syntactic Argument Position is Occupied Should Not Be Treated like an Implicit Argument}

For example, consider \ref{ex:cut1}, Here, \emph{The knife} occupies the subject position and should be treated as the causer of the cutting. We could add the person handling the knife as the causer, and treat the knife as an instrument. However, to add the former to the sentence, we would not merely have to add another realized argument, but also change the syntactic argument structure so that the the subject position goes to that causer, as in \ref{ex:cut2}. Thus, we treat this as a different framing with a different causer, rather than a more explicit version of the same framing. Likewise, \ref{ex:high1} and \ref{ex:high2} are two different framings, one with \emph{price} as \rl{has-state}, and one with \emph{butter}.

\ex.\label{ex:cut1}\dep{The knife_x-causer *cut#STATE-CHANGE* the butter_has-state}

\ex.\label{ex:cut2}\dep{Kim_x-causer *cut#STATE-CHANGE* the butter_has-state with the knife_x-means}

\ex.\label{ex:high1}\dep{The price_has-quantity is *high#QUANTITY*}

\ex.\label{ex:high2}\dep{The butter_has-quantity__has-quality is *high#QUANTITY* in **price#QUALITY**_x-attribute}


\newpage\subsection{When in Doubt, Treat Different Syntactic Frames of the Same Predicate Consistently}

For example, in \ref{ex:chase1}, \emph{chase} could be framed as caused motion
with Kim as \rl{x-causer} or as accompanied motion with Kim as
\rl{x-accompanier}. Because the latter works for other syntactic frames of
\emph{chase} as well, as in \ref{ex:chase2}, prefer it.

\ex.\label{ex:chase1}\dep{Kim_x-accompanier *chased#UNACHORED-MOTION* Sandy_in-motion around the block_transitory-location}

\ex.\label{ex:chase2}\dep{Kim_x-accompanier *chased#UNACHORED-MOTION* after Sandy_in-motion}


\newpage\subsection{However, Different Senses of a Predicate Can Have Different Arguments and Therefore Different Superframes}

One special case of this is when a predicate occurs as part of an opaque fixed
expression, like \emph{hand} in \emph{close at hand}. In this case, \emph{hand}
is not annotated with \fr{CLASS}, but with \fr{NONCOMP}.

\ex.\dep{I have seen them_has-location intimately , *close#LOCATION* at **hand#NONCOMP**_m-noncomp}


\newpage\subsection{Look Up Unfamiliar Words in a Dictionary}

When you come across an unfamilar predicate, you might not be able to determine
what arguments it has, and consequenlty what the most appropriate superframe is,
from this one context alone. Use a dictionary such as Wiktionary in this case.
In the following example, I found that \emph{toss off} can mean ``to assemble
hastily''\footnote{\url{https://en.wiktionary.org/w/index.php?title=toss\_off\&oldid=77814489}, retrieved 2024-05-28},
thus went for the \fr{TRANSFORMATION-CREATION} frame.

\ex.\dep{So_m-explanation I_x-causer *tossed#TRANSFORMATION-CREATION* off_m-noncomp this drawing_created}


\newpage\subsection{Symmetric Argument Pairs}

Some predicates have a pair of arguments that are semantically symmetric. In
such cases, assign the first role to the syntactically less oblique argument.

\ex.\dep{*distinguish#COMPARISON* China_compared from Arizona_reference}


\newpage\subsection{When to Use \fr{SCENE}}

\fr{SCENE} should definitely be used if a predicate can add aspectual meaning
to predicates of more than one type. For example, English \emph{make} can be
used with states and activities, so \emph{make} itself should be neither
\fr{STATE} nor \fr{ACTIVITY} but \fr{SCENE}.

\ex.\dep{Kim_x-causer *made#SCENE-INIT* Sandy_participant__is-active **dance#ACTIVITY**_scene}

\ex.\dep{Kim_x-causer *made#SCENE-INIT* Sandy_participant__has-state **tired#STATE**_scene}

On the other hand, if a predicate is restricted to subordinate predicates of a
certain type, it can have the same type.

\ex.\dep{I_undergoer *lived#EVENT* my life_event}

\ex.\dep{They_has-quantity *number#QUANTITY* in the thousands_quantity}


\newpage\section{Aspect, Mode, and Polarity}

\subsection{Aspect Annotation is wrt. the Superframe, Not the Predicate}

\ex.\dep{Kim_initial-possessor *lost#POSSESSION-DEINIT* the house_possessed}

In \Last, losing is framed as \fr{POSSESSION-DEINIT} because a state of
possession ends. \fr{POSSESSION-INIT} would be incorrect because although a
losing event begins, the state that the superframe \fr{POSSESSION} describes
ends. In general, aspectual suffixes modify superframes, they do not
necessarily indicate the aspectual class of the predicate (here: \emph{lost}).


\newpage\section{Construction-specific Guidelines}

\subsection{Participant Nouns}

Some nouns denote a person who participates in a specific type of scene in a
specific role. In such cases, use the most appropriate frame for that scene.
For example, in a narrative where the narrator has just been criticized by a
stranger, you could annotate as follows:

\ex.\dep{With that, my_topic *critic#MESSAGE* sat down again}

In other cases, such nouns rather denote a person's profession or expertise or
their role in a social context:

\ex.\dep{He_has-class is a *teacher#CLASS*}

\ex.\dep{He_social-relation is our_has-social-relation *teacher#SOCIAL-RELATION*}

\ex.\dep{She_has-social-relation is the *president#SOCIAL-RELATION* of our club_social-relation}


\newpage\subsection{Particle Verbs}

In UD, particle verbs are connected to their particle via the
\texttt{compound:prt} relation.

If the particle can be interpreted as an adposition with an elided complement
(often the case with spatial meanings), label that relation as the elided
complement would be labeled:

\ex.\dep{*get#LOCATION-DEINIT* the lid_has-location off_initial-location (the jar)}

\ex.\dep{You_has-location may *go#LOCATION-INIT* in_target-location (-to the room) now_m-time}

Also treat separated and nonseparated adpositional adverbs this way:

\ex.\dep{*Komm#LOCATION-INIT* herein_target-location !}

\ex.\dep{*Geh#LOCATION-INIT* da_target-location jetzt_m-time rein_target-location !}

Otherwise, use \rl{m-noncomp}:

\ex.\dep{*eat#INGESTION* up_m-noncomp the cookies_ingested}

\ex.\dep{*do#EVENT* somebody_undergoer in_m-noncomp}

\ex.\dep{Es_x-sender *stellte#MESSAGE* ein riesiges Gesicht_topic dar_m-noncomp}


\newpage\subsection{Pronouns with Arguments}

Definite pronouns are normally annotated with \fr{IDENTIFICATION}, indefinite
ones with \fr{CLASS}, and they do not have any arguments. However, sometimes
they do have arguments, in which case give them their antecendent's superframe:

\ex.\dep{The picture was *that#MESSAGE* of the boa_topic}

\ex.\dep{I drew a picture of a dog , *one#MESSAGE* of a cat_topic , and **another#MESSAGE** of a sheep__topic}


\newpage\subsection{Nominal Copula Constructions}

In nominal copula constructions, the copula subject is interpreted as a
non-core argument -- typically \rl{x-has-class} if the predicate is indefinite,
and \rl{x-identified} if it is definite.

\ex.\dep{This_x-identified is the *book#MESSAGE* I like_m-scene}

\ex.\dep{My drawing_x-has-class was not_m-mode a *picture#MESSAGE* of a hat_topic}


\newpage\subsection{Predicative Adpositions}

At the moment, Superframes follows UD's principle of treating adpositions like
case markers, dependent on their objects. This greatly simplifies the
annotation of adpositional arguments. On the other hand, it sometimes creates
problems. An adposition, added to a noun, can cause a new superframe to be
evoked, which it would be simpler to annotate if we could just label the
adposition with it. Consider the following examples, where we nonstandardly
treat the adpositions \emph{in}, \emph{out of}, and \emph{from} as adpositions.
The annotation is quite natural:

\ex.
\a.\dep{Ich_x-experiencer__has-location *will#MESSAGE* **in#LOCATION-CHANGE**_content die ***Berge#CLASS***__target-location}
\b.\dep{I_has-location am *out#LOCATION-NEG* of the office_location today_m-time}
\c.\dep{*sleep#STATE* a thousand miles__m-quantity **from#LOCATION**_m-location any human habitation__location}

But since we don't treat adpositions as predicates, we are forced to choose the following, more opaque and less detailed annotation:

\ex.
\a.\label{ex:berge}\dep{Ich_x-experiencer__x-has-location *will#MESSAGE* in die **Berge#CLASS**_content}
\b.\label{ex:office}\dep{I_x-has-location am out of the *office#CLASS* today_m-time}
\c.\label{ex:miles}\dep{*sleep#STATE* a thousand miles__m-quantity from any human **habitation#LOCATION**_m-location}

In \ref{ex:berge} and \ref{ex:office}, we are forced to give \emph{Berge} and
\emph{office} an \rl{x-has-location} role, which is not part of the frame
evoked by these words alone; we have to assume it is added by adding the
adposition. We also do not have a way to indicate that the additional
superframe introduced by the non-core subject is \fr{LOCATION-INIT} and
\fr{LOCATION-NEG}, respectively. In \ref{ex:miles}, there is an even more
severe problem: the quantity modifier \emph{a thousand miles} semantically
modifies the \fr{LOCATION} frame evoked by the adposition \emph{from}, but we
have to attach it to \emph{habitation}, which evokes a \emph{different}
\fr{LOCATION} frame which does not have a quantity modifier. Confusion ensues,
but for now we have to live with these issues.


\newpage\section{TODO}

codify the general principle somewhere: if superframe and ARG1 have the same
name (quasi-unary relations), we can just use m-rel. Otherwise, use m-scene.

Treatment of valency-changing operations:

\begin{enumerate}
    \item (obligatory) resultative
    \item V one's way P N
    \item comparative
    \item ...
\end{enumerate}

Clearer criteria for distinguishing between LVCs and idioms (or somehow eliminate it).

Make POSSESSION a special case of SOCIAL-RELATION. Rename SOCIAL-RELATION to
something like OBLIGATION?

\bibliographystyle{apalike}
\bibliography{anthology,custom}

\end{document}

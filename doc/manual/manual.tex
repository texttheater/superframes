\documentclass[a4paper]{article}

\usepackage[utf8]{inputenc}

\usepackage{tgpagella}
\usepackage[T1]{fontenc}

%opening
\title{Superframes v0 Manual}
\author{Kilian Evang}

\usepackage{hyperref}

\usepackage{booktabs}
\usepackage{linguex}
\usepackage{relsize}
\usepackage{tikz-dependency}

\begin{document}

\maketitle

\tableofcontents

\section{Introduction}

Superframes is an annotation scheme for semantic roles. It has the following goals:

\begin{enumerate}
    \item Annotation should be lexicon-free. With fine-grained schemes like FrameNet or PropBank, annotators have to constantly look up which frames exist and which roles are defined for them. Lexicons are also perennially incomplete, and the process of extending them is complicated. Superframes defines only a small number of coarse-grained frames with the aim of making annotation quick and easy across all languages and domains.
    \item Choosing frames and roles should be obvious. In VerbNet-like semantic role inventories, roles are semantically defined only vaguely and ambiguously. For example, the subject of the English verb \emph{watch} can be described as an Agent as well as an Experiencer. Prior approaches to resolving this ambiguity involve the creation of a lexicon (see above) or giving up on the idea of categorial role labels altogether. The Superframes approach is to proceed in two steps: first pick a superframe for each predicate, then the core roles are clearly defined. Additional argument roles are handled via mixin roles (see below).
    \item Comprehensive annotation: Superframes is a comprehensive and unified inventory of coarse semantic roles applicable to all types of contentful morphosyntactic dependencies, including argument roles, modifier roles, discourse relations, compound relations, etc. It is designed to be annotated on top of existing morphosyntactic dependency graphs (e.g., UD). This has the advantage that the markables are pre-identified and that an explicit annotation of the morphosyntax-semantics interface emerges.
    \item Ambiguity-tolerant: not all ambiguities in choosing a superframe can be resolved. Superframes encourages annotators to annotate multiple possibilities, in particular in the case of metaphorical language.
\end{enumerate}

\section{Types of Roles}

\begin{table}
    \begin{tabular}{llll}
        \toprule
        \textbf{relation} & \textbf{description} & \textbf{domain} & \textbf{range} \\
        \midrule
        \multicolumn{4}{l}{\emph{entity-entity}} \\
        asg & assignment & point & value \\
        cmp & comparison & compared & reference \\
        loc & location & located & location \\
        msg & message & topic & comment \\
        pss & possession/control & possessee & possessor \\
        qnt & quantity & of what & how much \\
        scn & scene & participant & scene \\
        soc & social & somebody & relative/org/task... \\
        whl & part-whole & part & whole \\
        \midrule
        \multicolumn{4}{l}{\emph{scene-entity}} \\
        ast & asset & scene & asset \\
        ben & beneficiary & scene & beneficiary \\
        cau & causer & scene & cause(r) \\
        ctx & context & scene & background \\
        mnr & manner & scene & manner \\
        mns & means & scene & means \\
        rcp & recipient & scene & recipient \\
        snd & sender & scene & sender \\
        tmp & temporal & scene & time/frequency/... \\
        %\midrule
        %\multicolumn{4}{l}{\emph{scene-scene}} \\
        %cnc & concession & happens anyway & happens admittedly \\
        %cnd & condition & happens conditionally & condition \\
        %cnt & continuation & happens & happens then \\
        %ctx & context & happens & context/circumstance \\
        %ela & elaboration & less specific & more specific \\
        %prp & purpose & scene & purpose \\
        \midrule
        \multicolumn{4}{l}{\emph{constructional}} \\
        anc & ancillary & predicate & ancillary entity \\
        att & attribute & predicate & attribute \\
        dis & discourse & token & discourse function \\
        dpc & depictive & predicate & secondary predicate\\
        exp & expletive & predicate & expletive \\
        nuc & nucleus & predicate & same predicate \\
        rsd & resultative & predicate & affected entity \\
        rsr & resultative & predicate & result \\ 
        \bottomrule
    \end{tabular}
    \caption{The inventory of superframes}
    \label{tab:inventory}
\end{table}

The Superframes annotation scheme is based on the binary relations shown in Table~\ref{tab:inventory}. Borrowing terminology from AMR, we call the first relate the ``domain'' and the second relate the ``range''. This inventory is then used to annotate bilexical dependencies with roles. We distinguish a number of types of roles, as explained in the following.

\subsection{Type I: Modifier Roles}

Modifier dependencies are annotated with a plain binary relation or its inverse (denoted by \textsf{-of}, as in AMR). This applies equally to verb, noun, and other modifiers.

\ex. \dep{*partied* at home_loc}
     \dep{a *man* with a mustache_whl-of}

\subsection{Type II: Core Argument Roles}

Predicates such as verbs (but also adjectives, event/state nouns, relational nouns, etc.) evoke their own superframe. The core arguments are those that correspond to the domain and the range, respectively. We denote them by the suffixes \textsf{d} and \textsf{r}, respectively.

\ex. \dep{Kim_locd *went* to Boston_locr}
     \dep{Kim_pssr *owns* a house_pssd}

The first example illustrates that Superframes abstract away from aktionsart: it does not matter for the choice of superframe or roles whether a state (Kim is in Boston) or an event bringing that state about (Kim went to Boston) is described. Borrowing terminology from UCCA CITE, we collectively call states and events ``scenes''.

\subsection{Type IIa: Initial and Intermediate Range Roles}

However, some predicates denote the dissolution of a relation between a domain and an initial range, and the establishment of the domain and a new range. To distinguish the initial from the final range, we use the prefix \textsf{ir} instead of \textsf{r} for it. Likewise, we use \textsf{mr} for intermediate ranges.

\ex. \dep{Kim_locd *went* from Chicago_locir via Pittsburg_locmr to Boston_locr}\\
     \dep{Kim_pssir *kept* the house_pssd}
     \dep{Kim_pssir *lost* the house_pssd}

\subsection{Type III: Non-core Argument Roles}

Some predicates have more than just the domain and range arguments. For such arguments, annotators should choose the binary relation that describes the relation between the scene and the argument best, and prefix it with \textsf{x} to distinguish it from a modifier. Particularly frequent non-core roles are \textsf{xcau}, \textsf{xsnd}, and \textsf{xrcp}.

\ex. \dep{Sandy_xcau *brought* Kim_locd to Boston_locr}
     \dep{Kim_xsnd *talked* about Sandy_msgd}\\
     \dep{Kim_xrcp *saw* Sandy_msgd swim_msgr}
     \dep{Kim_xrcp *searched* the woods_xloc for Sandy_msgd}

\subsection{Dual Framing}

For predicates that seem to fit two superframes equally well, annotate both. Likewise, if both a literal and a metaphorical meaning are accessible to you, annotate both.

\ex.
\begin{dependency}
    \begin{deptext}
        Kim \& refused \& to \& eat \\
    \end{deptext}
    \depedge[edge height=\baselineskip]{2}{1}{xsnd}
    \depedge[edge height=\baselineskip]{2}{4}{msgd}
    \depedge[edge below,edge height=\baselineskip]{2}{1}{scnd:locr}
    \depedge[edge below,edge height=\baselineskip]{2}{4}{scnr}
\end{dependency}
\begin{dependency}
    \begin{deptext}
        A \& hush \& passed \& over \& the \& group \\
    \end{deptext}
    \depedge[edge height=\baselineskip]{3}{2}{locd}
    \depedge[edge height=\baselineskip]{3}{6}{locmr}
    \depedge[edge below,edge height=\baselineskip]{3}{2}{scnr}
    \depedge[edge below,edge height=\baselineskip]{3}{6}{scnd:xsnd}
\end{dependency}

\section{Guidelines by Superframe}

% Template:
% "definition"
% modifier examples
% static argument examples
% dynamic argument examples
% subframes

\subsection{Scene (\textsf{scn})}
\label{sec:scn}

\textsf{scnd} is the single core participant of a scene specified by the
predicate, or a participant in a scene specified by the \textsf{scnr}.

\subsubsection{States and Properties}

\ex. \dep{Kim_scnd is *tall*}
     \dep{The painting_scnd *improved*}
     \dep{Kim_xcau *improved* the painting_scnd}

\subsubsection{Activities}

``Activities'' are events with a single core participant that can be
characterized as actively participating in the event.

\ex. \dep{Kim_scnd *partied*}
     \dep{Kim_scnd *partied* with Sandy_anc}
     \dep{Kim_scnd had *sex* with Sandy_xanc}

\subsubsection{Experiences}

``Experiences'' ar events with a single core participant that can be
characterized as undergoing the event.

\ex. \dep{Kim_xcau *attacked* Sandy_scnd}

\subsubsection{Transformation and Creation}

\textsf{scnd} denotes the entity undergoing the transformation, and
\textsf{scnr} denotes its new state. Creation is framed as transformation of
some material (\textsf{scnd}) into the newly created entity (\textsf{scnr}).

\ex. \dep{The ice_scnd *turned* into water_scnr}
     \dep{God_xcau *made* people_scnr out of clay_scnd}
     \dep{A rock_scnr *formed*}
     \dep{Kim_xcau *creatd* a work_scnr of art}

\subsubsection{Destruction}

\ex. \dep{The vase_scnd *broke*}
     \dep{Kim_xcau *broke* the vase_scnd}
     \dep{Kim_xcau *killed* Sandy_scnd}
     \dep{Sandy_scnd *died*}

\subsubsection{Auxilary Verbs}

Auxiliary verbs, when annotating on top of SUD rather than UD, also fall under
\textsf{scn}.

\ex. \dep{We_scnd *will* see_scnr}
     \dep{*Have* you_scnd eaten_scnr ?}
     \dep{Kim_scnd *must* go_scnr}

\subsubsection{Light Verbs}

\ex. \dep{Kim_scnd *took* a bath_scnr}

\subsubsection{Phase}

\ex. \dep{The wound_scnd *began* to heal_scnr}
     \dep{A commotion_scnr *started*}
     \dep{Kim_xcau *started* a commotion_scnr}
     \dep{The concert_scnir *ended*}
     \dep{The concert_scnir *continued*}
     \dep{Kim_xcau *interrupted* the session_scnir}

\subsubsection{Causation}

\ex. \dep{Kim_xcau *let* Sandy_scnd join_scnr}
     \dep{Kim_xcau *made* Sandy_scnd join_scnr}
     \dep{Kim_xcau *allowed* Sandy_scnd to join_scnr}

\subsubsection{Prevention}

\ex. \dep{Kim_xcau *kept* Sandy_scnd from joining_scnr}
     \dep{Kim_xcau *saved* Sandy_scnd from the dragon_scnr}

In the last example, understand \emph{dragon} metonymically as a scene in which
the dragon causes harm to Sandy.

\subsubsection{Relative Clauses}

For relative clauses, the \textsf{scn} label is used as a modifier relation:

\ex. \dep{a *package* that was too heavy_scn}

\subsubsection{Additional Markables}

TBD

\subsection{Social (\textsf{soc})}
\label{sec:soc}

The domain is an individual that is in some socially constructed relationship with the range. The range might e.g. be a relative, a friend, an organization, a responsibility, or a judicial sentence.

\subsubsection{Static Examples}

\ex. \smaller
     \dep{Kim_socr is my_socd *cousin*}
     \dep{Kim_socd and Sandy are *friends*}
     \dep{Kim_socd is *friends* with Sandy_socr}
     \dep{Kim_socd *works* at Google_socr}
     \dep{Kim_socd *works* for Sandy_socr}
     \dep{Kim_socd *emcees*}
     \dep{Kim_socd is *hosting* the party_socr}
     \dep{Kim_socd is under house *arrest*}

\subsubsection{Dynamic Examples}

\ex. \dep{Kim_socd *married* Sandy_socr}
     \dep{The official_xcau *married* Kim_socd to Sandy_socr}
     \dep{The official_xcau *married* Kim_socd and Sandy}
     \dep{Kim_socd *divorced* Sandy_socir}
     \dep{Kim_socd *befriended* Sandy_socr}
     \dep{Kim_socd *took* the job_socr}
     \dep{Kim_socd *joined* a Google_socr}
     \dep{Kim_socd *joined* a union_socr}
     \dep{Sandy_xcau *fired* Kim_socd from their job_socir}
     \dep{Kim_socd *left* Google_socir}
     \dep{Kim_socd *assumed* office_socr}
     \dep{The judge_xcau *sentenced* Kim_socd to three days_socr in prison}
     \dep{Kim_socd was *pardoned*}

\section{Difficult Cases}

I need you.

He responded to the provocation with violence.

He began the party with a speech.

Spend money/time on something

\end{document}

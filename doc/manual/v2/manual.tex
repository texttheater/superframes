\documentclass[a4paper]{article}

\usepackage[utf8]{inputenc}

%\usepackage{natbib}
\usepackage{tgpagella}
\usepackage[T1]{fontenc}

\usepackage{tcolorbox}

\title{Superframes Manual v2}
\author{Kilian Evang}

\begin{document}

\maketitle

\section{Introduction}

\subsection{The Superframes}

MESSAGE: sicher sein, dass ... topic oder content?
Wenn etwas topic UND content ist, content priorisieren: Er ist sich sicher, dass es 1984 ist (content). Vs.: Er liebt, dass es 1984 ist (topic).

modifier, die aber irgendwie sender/origin, experiencer ausdrücken: ich fand ein bild in einem buch; im publikum hört man pfeifen; aus dem publikum hört man pfeifen

symmetric argument pairs: choose x-sender over x-experiencer

x-attribute vs. x-has-attribute: "Die Haare sind ihm blond"

numeralpronomen vs. indefinitpronomen

"a professor of numerology and one of geology" -> ellipsis, how to annotate that?

META immer mit zweitem Frame? (Und wenn es SITUATION ist?)

Gebrauch als Begleiter (zwei Menschen, andere Menschen) vs. Gebrauch als Pronomen: wo zieht man hier die Grenze zwischen QUANTITY/COMPARISON auf der einen Seite und CLASS auf der anderen Seite? Sollte man die Entscheidung an die Syntax delegieren und PRON nicht annotieren?)

"werden zu" bei scenes vs. nonscenes, wie kriegen wir das in einen Frame?

"ziegiges Meckern" serbisches Beispiel s. 232 "ovčje blejanje" -> Wurzel-Wurzel-Relationen

made-from
named-after
fulfills

\subsection{Habitual Aspect}

\subsection{Phase}

\subsection{Modality}

\subsection{Negation}

\section{Basic Annotation}

\begin{tcolorbox}
    Example of a colorbox
\end{tcolorbox}

\subsection{Annotating Static Verbs}

\subsection{Annotating Dynamic Verbs}

\subsection{Annotating Nouns}

\subsection{Annotating Oblique Modifiers}

\subsection{Annotating Adjectives}

\subsection{Annotating Adverbs}

\section{Superframes Reference}

\section{Advanced Issues in Annotation}

\subsection{Literal and Figurative Annotation}

\footnote{Similar issues arise in syntax where phrases like \emph{the North Sea} or \emph{Gone with the Wind} could be annotated according to their internal syntactic structure, or as flat referring expressions}

\subsection{Choosing between Frames}

\subsection{Valency-changing Alternations}

\subsection{Nonlocal Dependencies}

\subsection{Multiword Predicates}

der Gin verbreitete einen Geruch

embedded like scenes: they number in the thousands, den Schmerz fühlen, eine Nachricht senden

\subsection{Exocentric Constructions}

%\bibliographystyle{apalike}
%\bibliography{main}

\end{document}

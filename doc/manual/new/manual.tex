\documentclass[a4paper]{article}

\usepackage[utf8]{inputenc}

\usepackage{natbib}
\usepackage{tgpagella}
\usepackage[T1]{fontenc}

\usepackage{amstext}
\usepackage{booktabs}
\usepackage{hyperref}
\usepackage{linguex}
\usepackage{mathtext}
\usepackage{relsize}
\usepackage{tikz-dependency}
\usepackage{todonotes}

\title{Superframes Manual}
\author{Kilian Evang}
\date{Last updated: \today}

% frame and role names
\newcommand{\fr}[1]{\textsf{#1}}
\newcommand{\frs}[1]{\mbox{\textsf{#1}}} % frame suffixes start with hyphen, prevent line break
\newcommand{\rl}[1]{\textsf{#1}}

\begin{document}

% less white space in examples
\setlength{\Exindent}{0pt}
\setlength{\Exlabelsep}{0pt}
\setlength{\SubExleftmargin}{0pt}
\setlength{\SubSubExleftmargin}{0pt}

\maketitle

%\begin{abstract}
%\end{abstract}

\tableofcontents

\section{Introduction}

\begin{table}
    \resizebox{\textwidth}{!}{
        \begin{tabular}{lllllll}
            \toprule
            Superframe & Roles & & & & & Sec. \\
            \midrule
            \fr{SCENE} & \rl{initial-scene} & \rl{participant} & \rl{scene} & \rl{transitory-scene} & \rl{target-scene} & \ref{sec:SCENE} \\
            \fr{IDENTIFICATION} & & \rl{identified} & \rl{identifier} & & & \ref{sec:IDENTIFICATION} \\
            \fr{ORDER} & & \rl{has-order} & \rl{order} & & & \ref{sec:ORDER} \\
            \fr{CLASS} & \rl{initial-class} & \rl{has-class} & \rl{class} & & \rl{target-class} & \ref{sec:CLASS} \\
            \fr{EXISTENCE} & & & \rl{exists} & & & \ref{sec:EXISTENCE} \\
            \fr{Transformation-Creation} & & \rl{material} & & & \rl{created} & \ref{sec:Transformation-Creation} \\
            \fr{Reproduction} & & \rl{original} & & & \rl{copy} & \ref{sec:Reproduction} \\
            %\fr{SUBCLASS} & & \rl{has-subclass} & \rl{subclass} & & \ref{sec:SUBCLASS} \\
            \fr{QUALITY} & & \rl{has-quality} & \rl{quality} & & & \ref{sec:QUALITY} \\
            \fr{STATE} & \rl{initial-state} & \rl{has-state} & \rl{state} & & \rl{target-state} & \ref{sec:STATE} \\
            %\fr{STATE-CHANGE} & & \rl{has-state} & & \rl{target-state} & \ref{sec:STATE-CHANGE} \\
            \fr{Destruction} & & \rl{destroyed} & & & & \ref{sec:Destruction} \\
            \fr{EXPERIENCE} & & \rl{experiencer} & \rl{experienced} & & & \ref{sec:EXPERIENCE} \\
            \fr{ACTIVITY} & & \rl{is-active} & \rl{activity} & & & \ref{sec:ACTIVITY} \\
            \fr{MARKER} & & \rl{has-marker} & \rl{marker} & & & \ref{sec:MARKER} \\
            \midrule
            \fr{ACCOMPANIMENT} & & \rl{accompanied} & \rl{accompanier} & & & \ref{sec:ACCOMPANIMENT} \\
            \fr{ATTRIBUTE} & & \rl{has-attribute} & \rl{attribute} & & & \ref{sec:ATTRIBUTE} \\
            \fr{Depictive} & & \rl{has-depictive} & \rl{depictive} & & & \ref{sec:Depictive} \\
            \fr{ASSET} & & \rl{has-asset} & \rl{asset} & & & \ref{sec:ASSET} \\
            \fr{CAUSATION} & & \rl{result} & \rl{causer} & & & \ref{sec:CAUSATION} \\
            \fr{Resultative} & & \rl{has-resultative} & \rl{resultative} & & & \ref{sec:Resultative} \\
            \fr{COMPARISON} & & \rl{compared} & \rl{reference} & & & \ref{sec:COMPARISON} \\
            \fr{Concession} & & \rl{assertion} & \rl{conceded} & & & \ref{sec:Concession} \\
            \fr{EXPLANATION} & & \rl{explained} & \rl{explanation} & & & \ref{sec:EXPLANATION} \\
            \fr{Purpose} & & \rl{has-purpose} & \rl{purpose} & & & \ref{sec:Purpose} \\
            \fr{LOCATION} & \rl{initial-location} & \rl{has-location} & \rl{location} & \rl{transitory-location} & \rl{target-location} & \ref{sec:LOCATION} \\
            \fr{Wrapping-Wearing} & & \rl{worn} & \rl{wearer} & & & \ref{sec:Wrapping-Wearing} \\
            %\fr{Wrapping-Wearing-Init} & & \rl{worn} & & \rl{target-wearer} & \ref{sec:Wrapping-Wearing-Init} \\
            %\fr{Wrapping-Wearing-Deinit} & \rl{initial-wearer} & \rl{worn} & & & \ref{sec:Wrapping-Wearing-Deinit} \\
            \fr{Adornment-Tarnishment} & \rl{initial-surface} & \rl{ornament} & \rl{surface} & & \rl{target-surface} & \ref{sec:Adornment-Tarnishment} \\
            %\fr{Adornment-Tarnishment-Init} & & \rl{ornament} & & \rl{target-surface} & \ref{sec:Adornment-Tarnishment-Init} \\
            %\fr{Adornment-Tarnishment-Deinit} & \rl{initial-surface} & \rl{ornament} & & & \ref{sec:Adornment-Tarnishment-Deinit} \\
            \fr{Touching} & & \rl{touching} & \rl{touched} & & & \ref{sec:Touching} \\
            %\fr{LOCATION-HABIT} & & \rl{has-location} & \rl{location} & & \ref{sec:LOCATION-HABIT} \\
            %\fr{LOCATION-INIT} & & \rl{has-location} & \rl{transitory-location} & \rl{target-location} & \ref{sec:LOCATION-INIT} \\
            \fr{Ingestion} & & \rl{ingested} & & \rl{transitory-location} & \rl{ingester} & \ref{sec:Ingestion} \\
            \fr{Hitting} & & \rl{hitting} & & & \rl{hit} & \ref{sec:Hitting} \\
            %\fr{LOCATION-DEINIT} & \rl{initial-location} & \rl{has-location} & \rl{transitory-location} & & \ref{sec:LOCATION-DEINIT} \\
            \fr{Excretion} & \rl{excreter} & \rl{excreted} & & \rl{transitory-location} & & \ref{sec:Excretion} \\
            %\fr{LOCATION-CHANGE} & \rl{initial-location} & \rl{has-location} & \rl{transitory-location} & \rl{target-location} & \ref{sec:LOCATION-CHANGE} \\
            \fr{Motion} & & \rl{has-location} & & \rl{transitory-location} & & \ref{sec:Motion} \\
            \fr{MEANS} & & \rl{has-means} & \rl{means} & & & \ref{sec:MEANS} \\
            \fr{MESSAGE} & & \rl{topic} & \rl{content} & & & \ref{sec:MESSAGE} \\
            %\fr{MESSAGE-INIT} & & \rl{topic} & & \rl{target-message} & \ref{sec:MESSAGE-INIT} \\
            %\fr{MESSAGE-DEINIT} & \rl{initial-message} & \rl{topic} & & & \ref{sec:MESSAGE-DEINIT} \\
            %\fr{MESSAGE-INIT-HABIT} & & \rl{topic} & \rl{message} & & \ref{sec:MESSAGE-HABIT} \\
            \fr{PART-WHOLE} & initial-whole & \rl{part} & \rl{whole} & & target-whole & \ref{sec:PART-WHOLE} \\
            \fr{POSSESSION} & initial-possessor & \rl{possessed} & \rl{possessor} & & target-possessor & \ref{sec:POSSESSION} \\
            %\fr{POSSESSION-INIT} & & \rl{possessed} & & \rl{target-possessor} & \ref{sec:POSSESSION-INIT} \\
            %\fr{POSSESSION-DEINIT} & \rl{initial-possessor} & \rl{possessed} & & & \ref{sec:POSSESSION-DEINIT} \\
            %\fr{POSSESSION-CHANGE} & \rl{initial-possessor} & \rl{possessed} & \rl{target-possessor} & & \ref{sec:POSSESSION-CHANGE} \\
            %\fr{POSSESSION-CHANGE-NECESSITY} & \rl{initial-possessor} & \rl{possessed} & \rl{target-possessor} & & \ref{sec:POSSESSION-CHANGE-NECESSITY} \\
            %\fr{POSSESSION-CONTINUATION} & \rl{initial-possessor} & \rl{possessed} & & & \ref{sec:POSSESSION-CONTINUATION} \\
            \fr{QUANTITY} & & \rl{has-quantity} & \rl{quantity} & & & \ref{sec:QUANTITY} \\
            \fr{SENDING} & & \rl{sent} & \rl{sender} & & & \ref{sec:SENDING} \\
            \fr{SEQUENCE} & & \rl{follows} & \rl{followed} & & & \ref{sec:SEQUENCE} \\
            \fr{SOCIAL-RELATION} & initial-social-relation & \rl{has-social-relation} & \rl{social-relation} & & target-social-relation & \ref{sec:SOCIAL-RELATION} \\
            %\fr{SOCIAL-RELATION-INIT} & & \rl{has-social-relation} & & \rl{target-social-relation} & \ref{sec:SOCIAL-RELATION-INIT} \\
            %\fr{SOCIAL-RELATION-DEINIT} & \rl{initial-social-relation} & \rl{has-social-relation} & & & \ref{sec:SOCIAL-RELATION-DEINIT} \\
            \fr{TIME} & & \rl{has-time} & \rl{time} & & & \ref{sec:TIME} \\
            %\fr{SCENE-INIT} & & \rl{participant} & & \rl{target-scene} & \ref{sec:SCENE-INIT} \\
            %\fr{SCENE-DEINIT} & \rl{initial-scene} & \rl{participant} & & & \ref{sec:SCENE-DEINIT} \\
            %\fr{SCENE-CONTINUATION} & \rl{initial-scene} & \rl{participant} & & & \ref{sec:SCENE-CONTINUATION} \\
            %\fr{SCENE-PREVENTION} & & \rl{participant} & & \rl{target-scene} & \ref{sec:SCENE-PREVENTION} \\
            %\fr{SCENE-NECESSITY} & & \rl{participant} & & \rl{target-scene} & \ref{sec:SCENE-NECESSITY} \\
            %\fr{SCENE-POSSIBILITY} & & \rl{participant} & & \rl{target-scene} & \ref{sec:SCENE-POSSIBILITY} \\
            \midrule
            \fr{NONCOMP} & & \rl{has-noncomp} & \rl{noncomp} & & & \ref{sec:NONCOMP} \\
            \bottomrule
        \end{tabular}
    }
    \caption{The superframes and their roles.}
    \label{tab:superframes}
\end{table}

Superframes is an annotation scheme for semantic roles. Like other such
schemes, it is essentially about pinning down, in a machine-readable form,
``who did what to whom''. It is different from other such schemes, such as
FrameNet \citep{baker-etal-1998-berkeley}, VerbNet
\citep{kipper-schuler-2005-verbnet}, PropBank
\citep{palmer-etal-2005-proposition}, VerbAtlas
\citep{di-fabio-etal-2019-verbatlas}, or WiSER \citep{feng-etal-2022-widely} in
a number of ways. It aims to avoid a number of practical problems in annotating
with those schemes. Here's how Superframes annotation works, in a nutshell:

\begin{enumerate}
    \item Every content word (verb, noun, pronoun, adjective, or adverb) is a
        \emph{predicate}. Every predicate evokes one of a few dozen
        \emph{superframes}, which determines its coarse semantic class and the
        possible role labels for its arguments.
    \item The syntactic \emph{dependents} of a predicate can be
        \emph{core arguments}, in which case they get one of the role labels
        defined by the superframe of the predicate, or \emph{external
        arguments} or \emph{modifiers}, in which case they are treated as
        evoking their own frame in which the predicate serves as a core argument.
    \item There are only two main core role labels per superframe.
    \item For predicates denoting change (or lack thereof) over time,
        some superframes have \emph{aspectual variants} with role variants that
        allow to distinguish participants before, during, and after an event.
        This avoids having \texttt{Source} and \texttt{Target} as roles in
        their own right, which indicate the time sequence but suppress
        information about the nature of the relation that is changing.
        %like \texttt{-CHANGE},
        %\texttt{-INTRO}, \texttt{-EXTRO}, \texttt{-PREEMPTION},
        %\texttt{-CONTINUATION}, \texttt{-PROCESS}, \texttt{-HABIT},
        %\texttt{-NECESSITY}, and \texttt{-POSSIBILITY}
    \item Similarly, Superframes do not have the \texttt{Agent} role, which is
        often in conflict with roles indicating more specifically the agent's
        relation to other participants.
    \item Doubt, ambiguity, and figurativity are systematically treated. If there
        is not one clear solution, the solution is to give two or more
        alternative labels.
\end{enumerate}

Table~\ref{tab:superframes} shows the superframes and their roles.

\subsection{Core Arguments}

The most prototypical predicate is a verb, and the simplest case is a verb with only one argument. It can for example denote an experience or an activity:

\ex.\dep{Kim_experiencer is *sleeping#EXPERIENCE*}

\ex.\dep{Kim_is-active is *partying#ACTIVITY*}

With two core arguments, a verb denotes a relation that holds between them:

\ex.\dep{Kim_possessor *owns#POSSESSION* a house_possession}

\ex.\dep{The house_possession *belongs#POSSESSION* to Kim_possessor}

\ex.\dep{Kim_topic *seems#MESSAGE* happy_content}

\subsection{Frame Variants for Tense, Aspect, and Mood}

Rather than a static relationship between two entities, many verbs (and other
predicates) denote a change (or absence of change) in such a relationship. We
sort such predicates into a few coarse aspectual classes. For example,
initiation (\frs{-INIT}) means a state is begun or worked towards, deinitiation
(\frs{-DEINIT}) means a state is ended, completed, or its end is worked towards,
change (\frs{-CHANGE}) combines both, where one state is replaced by another,
and continuation (\frs{-CONT}) means a state persists or is even intensified.
Accordingly, roles with \rl{target-}, \rl{initial-}, or \rl{transitory-} mark
participants at/beyond the end of, at the beginning of, or at some point during
the event, respectively.

\ex.\dep{Kim_target-possessor *got#POSSESSION-INIT* the house_possession}

\ex.\dep{Kim_initial-possessor *lost#POSSESSION-DEINIT* the house_posesssion}

\ex.\dep{Kim_initial-possessor *sold#POSSESSION-CHANGE* the house_possession to Sandy_target-possessor}

\ex.\dep{Kim_initial-possessor *kept#POSSESSION-CONT* the house_possession}

\ex.\dep{Kim_has-location *went#LOCATION-CHANGE* from Chicago_initial-location via Pittsburgh_transitory-location to Boston_target-location}

\ex.\label{ex:fall}\dep{The vase_has-location *fell#LOCATION-CHANGE* to the ground_target-location}

\ex.\dep{The vase_has-state *broke#STATE-CHANGE*}

\ex.\dep{Kim_has-social-relation *befriended#SOCIAL-RELATION-INIT* Sandy_target-social-relation}

\ex.\dep{Kim_has-social-relation *married#SOCIAL-RELATION-INIT* Sandy_target-social-relation}

\ex.\dep{Kim_has-social-relation *divorced#SOCIAL-RELATION-DEINIT* Sandy_initial-social-relation}

The \fr{SCENE} superframe is often evoked by ``light'' verbs that contribute an aspectual or modal meaning. Thus, its aspectual variants are especially common.

\ex.\dep{The concert_target-scene *began#SCENE-INIT*}

\ex.\dep{The concert_initial-scene *continued#SCENE-CONT*}

\ex.\dep{The concert_initial-scene *finished#SCENE-DEINIT*}

\ex.\dep{The shouting_initial-scene *intensified#SCENE-CONT*}

\ex.\dep{The shouting_initial-scene *faded#SCENE-DEINIT*}

\ex.\dep{A coup_target-scene was *attempted#SCENE-INIT*}

In addition, we use the suffixes \frs{-NECESSITY}, \frs{-POSSIBILITY},
\frs{-HABIT}, and \frs{-TIME} to mark the corresponding tense/aspect/mode
categories.

\ex.\dep{Change_target-scene is *necessary#SCENE-NECESSITY*}

\ex.\dep{Change_target-scene is *possible#SCENE-POSSIBILITY*}

\ex.\dep{Kim_participant *plays#SCENE-HABIT* tennis_scene}

\ex.\dep{Kim_participant *used#SCENE-TIME* to play_scene tennis}

\ex.\dep{Kim_is-active is an avid_m-quality *unicyclist#ACTIVITY-HABIT*}

\ex.\dep{Kim_initial-possessor *owes#POSSESSION-CHANGE-NECESSITY* Sandy_target-possessor money_possessed}

\subsection{Non-core Arguments}

Core arguments always get role labels from the superframe the predicate evokes.
But many verbs have more arguments. One common case is a subject that is
presented as the causer of the scene. For example, compare \ref{ex:throw} with
\ref{ex:fall}. The core scene is the same (same superframe, same arguments). We
now assume there is an additional \fr{CAUSATION} scene with \emph{Kim} as the
\rl{causer} and the core scene as the \rl{result}. We denote this by giving
\emph{Kim} the \rl{causer} role label, with an \rl{x-} prefix to mark it as a
non-core role.

\ex.\label{ex:throw}\dep{Kim_x-causer *threw#LOCATION-CHANGE* the vase_has-location to the ground_target-location}

\ex.\dep{Kim_x-causer *broke#STATE-CHANGE* the vase_has-state}

Two other common non-core arguments are the senders and recipients (experiencers) of messages.

\ex.\dep{Kim_x-sender *talked#MESSAGE* to Sandy_x-experiencer about Bali_topic}

Other non-core arguments are usually rather predicate-specific.

\ex.\dep{Kim_x-experiencer *searched#MESSAGE* the woods_x-location for Sandy_topic}

\ex.\dep{Kim_initial-possessor *sold#POSSESSION-CHANGE* Sandy_target-possessor the house_possession for a million dollars_x-asset}

\subsection{Modifiers}

Like non-core arguments, modifiers are assumed to evoke an additional frame,
and labeled with the role they fill in that frame, but with a prefix marking
them as modifiers: \rl{m-}.

\ex.\dep{Kim_excreter is *sweating#EXCRETION* profusely_m-quantity in the sauna_m-location}

\subsection{Nonverbal Predicates}

So far, we have only looked at verbal predicates. But of course, there are
other types of predicates. An ordinary noun like \emph{tree} evokes the
\fr{CLASS} frame, marking the entity it refers to as being a member of a class
(in this case: the class of trees). There are no arguments here because the
predicate itself doubles as a referent. However, the predicate can of course be
modified:

\ex.\dep{a *tree#CLASS* in the garden_m-location}

\ex.\dep{Kim_m-possessor 's *tree#CLASS*}

Event nouns evoke event frames and have arguments:

\ex.\dep{Kim_x-causer 's *breaking#STATE-CHANGE* of the vase_has-state}

Relational nouns evoke relational frames and have arguments:

\ex.\dep{Kim_has-social-relation 's *friend#SOCIAL-RELATION*}

Pronouns and names evoke the \fr{IDENTIFICATION} frame, meaning that they
identify their referent as soe entity (via naming or anaphora resolution).

\ex.\dep{*Kim#IDENTIFICATION*}

\ex.\dep{*they#IDENTIFICATION*}

Predicate adjectives most typically denote states or qualities.

\ex.\dep{I_has-quality am *despicable#QUALITY*}

\ex.\dep{the dog_has-state is *tired#STATE*}

With attributive adjectives, the dependency relation is reversed, and the role label is changed accordingly.

\ex.\dep{despicable_m-quality *me#IDENTIFICATION*}

\ex.\dep{the tired_m-state *dog#CLASS*}

Similarly for adverbs denoting, e.g, manner (\rl{quality}) or extent (\rl{quantity}):

\ex.\dep{Kim_has-location *ran#Motion* fast_m-quality}

\ex.\dep{Kim_has-location *ran#Motion* far_m-quantity}

\subsection{Control Relations}
\label{sec:control}

\todo[inline]{spell out strategies for consistent detection (xcomp, MESSAGE/SCENE frames, special cases...)}

Many constructions systematically introduce semantic predicate-dependent
dependencies that do not correspond to (surface) syntactic dependencies. In such cases, we add those dependency links.

\ex.\dep{the song_topic I_x-experiencer *like#MESSAGE*} (relative clause)

\ex.\dep{Kim_has-location promised Sandy to *come#LOCATION-CHANGE*} (subject control)

\ex.\dep{Kim persuaded Sandy_has-location to *come#LOCATION-CHANGE*} (object control)

\ex.\dep{Kim_has-location seemed to *fly#Motion*} (raising)

\ex.\dep{Kim_x-sender entered the room *singing#MESSAGE*} (depictive)

\ex.\dep{You're talking me_has-state *silly#STATE*} (resultative)

\ex.\dep{Kim_has-location has come to *stay#LOCATION-CONTINUATION*} (subjectless adverbial clause)

\ex.\dep{Kim_x-causer left after *trashing#STATE-CHANGE* the room_has-state} (subjectless adverbial clause)

\ex.\dep{Kim_topic is hard to *love#MESSAGE*} (\emph{tough} construction)

\ex.\dep{the question_topic we_x-sender raised without *answering#MESSAGE*} (parasitic gap)

\subsection{Figurativity and Idiomaticity}

Difficulties in choosing frames often arise because a predicate literally evokes
one frame, but is used in a way that perhaps fits another frame equally well or
better. In such cases, annotate both the more literal frame and roles, followed
by the \texttt{>}\texttt{>} operator, followed by the more figurative frame and
roles.

\ex.\dep{primeval_m-time>>m-subclass *forest#CLASS*}

\ex.\dep{colored_m-quality>>m-subclass *pencil#CLASS*}

\ex.\dep{to *lay#LOCATION-CHANGE>>MESSAGE-DEINIT* aside_target-location>>x-noncomp my drawings_has-location>>topic}

\section{Superframes Reference}

\subsection{\fr{SCENE}}
\label{sec:SCENE}

TBD

\subsection{\fr{IDENTIFICATION}}
\label{sec:IDENTIFICATION}

The \rl{identifier} identifies the \rl{identified}.

Evoked by pronouns, names, and other identifiers, as well as predicates
denoting naming relationships.

\ex.\dep{*I#IDENTIFICATION* saw a picture}

\ex.\dep{I can distinguish *China#IDENTIFICATION* from Arizona}

\ex.\dep{a book_identified *called#IDENTIFICATION* True Stories_identifier from Nature}

\ex.\dep{This_identified is *Kim#IDENTIFICATION*}

\ex.\dep{This_x-identified is the *book#MESSAGE* I like_m-scene}

\subsection{\fr{ORDER}}
\label{sec:ORDER}

\rl{order} indicates the order that \rl{has-order} has in some sequence.

\ex.\dep{*Chapter#MESSAGE* 1_m-order}

\ex.\dep{my_m-sender first_m-order *drawing#MESSAGE*}

\subsection{\fr{CLASS}}
\label{sec:CLASS}

\rl{class} indicates the class of entity that \rl{has-class} represents.

Most prototypically evoked by common nouns with no arguments.

\ex.\dep{swallowing an animal#CLASS}

\subsection{\fr{EXISTENCE}}
\label{sec:EXISTENCE}

\rl{exists} exists. Use this only for non-scene entities; for scenes, use the \fr{SCENE} frame.

\ex.\dep{There_x-expletive *is#EXISTENCE* a hill_exists}

\ex.\dep{There_x-expletive *is#SCENE* a hubbub_scene}

\subsection{\fr{Transformation-Creation}}
\label{sec:Transformation-Creation}

\rl{created} is newly created from \rl{material}, or \rl{material} is
transformed to acquire a new class indicated by \rl{created}.

\ex.\dep{I_x-causer succeeded in *making#Transformation-Creation* my first drawing_created}

\ex.\dep{Kim_x-causer *built#Transformation-Creation* a castle_created out of sand_material}

\subsection{\fr{Reproduction}}
\label{sec:Reproduction}

\rl{original} continues to exist, and a (modified) \rl{copy} comes into existence.

\ex.\dep{Here is a *copy#Reproduction* of the drawing_original}

\ex.\dep{This_copy is a *translation#Reproduction* of the pamphlet_original into English_x-quality}

%\subsection{\fr{SUBCLASS}}
%\label{sec:SUBCLASS}
%
%\rl{has-class} represents a class of entities, and more specifically the \rl{subclass}.
%
%Evoked chiefly by modifiers that in combination with a modifiee denoting a
%class of entities denote a specific subclass.
%
%\ex.\dep{colored *pencil#CLASS*}

\subsection{\fr{QUALITY}}
\label{sec:QUALITY}

\rl{quality} indicates a (permanent) quality/property/manner of \rl{has-quality}.

\ex.\dep{when I_has-quality was six years_x-quantity *old#QUALITY*}

\ex.\dep{a magnificent_m-quality *picture#MESSAGE*}

\ex.\dep{I_x-experiencer *pondered#MESSAGE* deeply_m-quality over the adventures_topic of the jungle}

\subsection{\fr{STATE}}
\label{sec:STATE}

\rl{state} indicates a (temporary) state of \rl{has-state}.

\ex.\dep{Boa constrictors swallow their prey_has-state *whole#STATE*}

\ex.\dep{they_has-state *sleep#STATE*}

\subsection{\fr{STATE-CHANGE}}
\label{sec:STATE-CHANGE}

A \fr{STATE} changes.

\ex.\dep{they_x-causer swallow their prey whole without *chewing#STATE-CHANGE* it_has-state}

\ex.\dep{the six months that they_x-causer need for *digestion#STATE-CHANGE*}

\ex.\dep{And that_x-causer hasn't much *improved#STATE-CHANGE* my opinion_has-state of them}

\subsection{\fr{Destruction}}
\label{sec:Destruction}

\rl{destroyed} goes out of existence.

\ex.\dep{Sam_destroyed 's *death#Destruction*}

\ex.\dep{Sam_x-causer 's *destruction#Destruction* of the city}

\subsection{\fr{EXPERIENCE}}
\label{sec:EXPERIENCE}

\rl{experienced} indicates an experience that \rl{experiencer} undergoes.

Used for dynamic scenes where the \rl{experiencer} is not necessarily active,
and that cannot well be framed as a state change. Also used for sensory and
mental perception, addressees in communication, beneficiaries, and for ``bystander'' roles.

\ex.\dep{Kim_experiencer 's *adventures#EXPERIENCE* in the jungle_m-location}

\ex.\dep{Kim_x-causer *attacked#EXPERIENCE* Sandy_experiencer}

\ex.\dep{I_x-experiencer *saw#MESSAGE* a magnificent picture_topic}

\ex.\dep{I_x-experiencer *pondered#MESSAGE* deeply_m-quality}

\ex.\dep{Kim_x-sender *talked#MESSAGE* to Sandy_x-experiencer}

\ex.\dep{Kim_participant *did#SCENE* something_scene nice for Sandy_m-experiencer}

\ex.\dep{Kim_x-experiencer cooked a meal only to *have#SCENE* Sandy_participant spurn_scene it}

\subsection{\fr{ACTIVITY}}
\label{sec:ACTIVITY}

\rl{is-active} actively participates in \rl{activity}.

Used for dynamic scenes where \rl{is-active} has agency and that cannot well be
framed as a state change.

\ex.\dep{Kim_is-active *worked#ACTIVITY*}

\ex.\dep{Kim_is-active *partied#ACTIVITY*}

\ex.\dep{Kim_is-active *danced#ACTIVITY*}

\ex.\dep{Kim_is-active had *sex#ACTIVITY*}

\ex.\dep{after some *work#ACTIVITY* with a colored pencil_m-means}

\ex.\dep{I_is-active devoted myself to *geography#ACTIVITY*}

\subsection{\fr{MARKER}}
\label{sec:MARKER}

\rl{marker} marks \rl{has-marker} for modal strength, aspect, discourse function, etc.

Umbrella frame for various kinds of predicates that denote properties of propositions rather than scenes, often realized as ``sentence adverbs''.

\ex.\dep{Fortunately_m-marker, Kim_x-experiencer probably_m-marker even_m-marker *knows#MESSAGE* that_content}

\subsection{\fr{ACCOMPANIMENT}}
\label{sec:ACCOMPANIMENT}

\rl{accompanier} accompanies \rl{accompanied}, meaning that it occurs together
with it or participates equally in the same scene.

\ex.\dep{*veggies#CLASS* with rice_m-accompanier}

\ex.\dep{The veggies_accompanied *come#ACCOMPANIMENT* with rice_accompanier}

\ex.\dep{Kim_x-causer *added#ACCOMPANIMENT-INIT* rice_accompanier to the veggies_accompanied}

\ex.\dep{Rolling thunder_accompanier *accompanies#ACCOMPANIMENT* the rain_accompanied}

Often, the accompanier denotes not the accompanying scene but an entity
participating in it, and must be metonymically understood as the scene.

\ex.\dep{Kim *cycled#LOCATION-CHANGE* to Rome_target-location with Sandy_m-accompanier}

\ex.\dep{Kim_is-active *danced#ACTIVITY* with Sandy_x-accompanier}

\ex.\dep{Kim_participant *had#SCENE* sex_scene with Sandy_x-accompanier}

\ex.\dep{Kim_x-accompanier *chased#Motion* Sandy_has-location around the block_transitory-location}

\ex.\dep{Kim_x-accompanier *accompanied#ACCOMPANIMENT* Sandy_accompanied}

\ex.\dep{Kim_x-accompanier *accompanied#ACCOMPANIMENT* Sandy_accompanied on the piano_x-means}

\subsection{\fr{Depictive}}
\label{sec:Depictive}

Special case of \fr{ACCOMPANIMENT} where \rl{accompanier} assigns
\rl{accompanied} a role (cf. Sec.~\ref{sec:control}).

\ex.\dep{Kim_has-location__x-sender *entered#LOCATION-INIT* the room_target-location **singing#MESSAGE**_m-depictive}

\subsection{\fr{ATTRIBUTE}}
\label{sec:ATTRIBUTE}

In a scene \rl{has-attribute}, \rl{attribute} is the part or attribute of one or more participants that is most directly involved in the scene.

\ex.\dep{Kim_compared__has-quality *exceeds#COMPARISON* Sandy_reference__has-quality in **height#QUALITY**_x-attribute}

\ex.\dep{That_has-quality__has-quality is *great#QUALITY* in terms of **ROI#QUALITY**_m-attribute}

\ex.\dep{Kim_hitting__x-whole ist auf den **Kopf#CLASS**_x-attribute *gefallen#HITTING*}

\ex.\dep{Kim_x-causer *hit#HITTING* Sandy_hit__x-whole on the **head#CLASS**_x-attribute with a stick_hitting}

\todo[inline]{Control relations?}

\subsection{\fr{ASSET}}
\label{sec:ASSET}

In a scene \rl{has-asset}, \rl{asset} is given or offered in an exchange or wager.

\ex.\dep{Kim_target-possessor *bought#POSSESSION-CHANGE* the house_possession for a million dollars_x-asset}

\ex.\dep{Kim_x-sender *offered#MESSAGE* Sandy_x-experiencer a million dollars_message for the house_x-asset}

\ex.\dep{I_x-sender *bet#MESSAGE* you_x-experiencer 30 bucks_x-asset to an apple_x-reference he will win_message}

\subsection{\fr{CAUSATION}}
\label{sec:CAUSATION}

\rl{causer} causes \rl{result}.

\ex.\dep{Kim_x-causer *broke#STATE-CHANGE* the glass_has-state}

\ex.\dep{The knife_x-causer *cut#STATE-CHANGE* the bread_has-state}

\ex.\dep{Kim_x-causer *cut#STATE-CHANGE* the bread_has-state with a knife_m-means}

\ex.\dep{The war_causer *caused#CAUSATION* a famine_result}

\ex.\dep{There_x-expletive *was#SCENE* a famine_scene because of the war_m-causer}

\ex.\dep{Der Wasserdruck_has-quantity *stieg#QUANTITY-CHANGE*, wodurch der Brunnen überfloss_m-result}

\ex.\dep{Die Qualität_result ist der Motivation_causer *geschuldet#CAUSATION*}

\ex.\dep{Kim_has-location *went#LOCATION-CHANGE* to town_target-location because they wanted_m-causer to buy food}

Note how the last example expresses a purpose, but expresses it as a cause, so
\rl{m-causer} lis the right label to use. Compare this to construal as a
purpose:

\ex.\dep{Kim_has-location *went#LOCATION-CHANGE* to town_target-location to buy_m-explanation food}

\subsection{\fr{Resultative}}
\label{sec:Resultative}

Special case of \fr{CAUSATION} where \rl{result} assigns an argument of
\rl{causer} a role. We treat the English resultative construction as a
valency-changing operation that adds one or two arguments to the matrix
predicate, so we use \rl{x-resultative} rather than \rl{m-resultative}.

\ex.\dep{Kim_x-causer *hammered#HITTING* the metal_hit__has-state **flat#STATE**_x-resultative}

\ex.\dep{Kim_experiencer *sneezed#EXPERIENCE* the napkin_x-resultative__m-has-location off the **table#CLASS**_x-resultative}

\todo[inline]{The last example shows the limits of our coding, we don't really
have a token to label as \fr{LOCATION-DEINIT}. If labeling atop SUD, we'd have
\emph{off}, but there's probably languages that express the same thing through
case marking rather than an adposition, then we'd have the same problem again.}

\subsection{\fr{COMPARISON}}
\label{sec:COMPARISON}

\rl{compared} is characterized with respect to \rl{reference}.

Examples of comparing scenes:

\ex.\dep{Compared_m-reference to Sandy, Kim_has-quality is *tall#QUALITY*}

\ex.\dep{Sandy_has-quality is *short#QUALITY* whereas Kim is tall_m-comparison}

\ex.\dep{They_x-sender *demonize#MESSAGE* the left_topic while doing_m-reference nothing about the right}

Examples of comparing non-scene entities:

\ex.\dep{Kim_compared *outranks#COMPARISON* Sandy_reference}

\ex.\dep{Kim_compared *exceeds#COMPARISON* Sandy_reference in height_x-attribute}

\ex.\dep{The Polish restaurant_compared *compared#COMPARISON* favorably_x-quality to the Spanish one_reference}

\ex.\dep{Kim_x-experiencer *compared#COMPARISON* Coke_compared to Pepsi_reference}

\ex.\dep{Kim_compared *ran#COMPARISON* afoul_x-noncomp of Fielding 's constraints_reference}

\subsection{\fr{Concession}}
\label{sec:Concession}

Special case of \fr{COMPARISON}, where \rl{compared} is what's asserted and \rl{reference} is what's conceded.

\ex.\dep{Kim_has-location *went#LOCATION-CHANGE* out_target-location despite the rain_m-conceded}

\ex.\dep{It_x-expletive *rained#STATE*, but Kim went_m-asserted went out}

\ex.\dep{Kim_sender *sent#SENDING* Sandy_x-experiencer a letter_sent, but it never arrived_m-asserted}

\ex.\dep{Kim_has-location *came#LOCATION-CHANGE* although Sandy had told_m-conceded them not to}

\subsection{\fr{EXPLANATION}}
\label{sec:EXPLANATION}

TBD

\subsection{\fr{Purpose}}
\label{sec:Purpose}

TBD

\subsection{\fr{LOCATION}}
\label{sec:LOCATION}

TBD

\subsection{\fr{Wrapping-Wearing}}
\label{sec:Wrapping-Wearing}

TBD

\subsection{\fr{Wrapping-Wearing-Init}}
\label{sec:Wrapping-Wearing-Init}

TBD

\subsection{\fr{Wrapping-Wearing-Deinit}}
\label{sec:Wrapping-Wearing-Deinit}

TBD

\subsection{\fr{Adornment-Tarnishment}}
\label{sec:Adornment-Tarnishment}

TBD

\subsection{\fr{Adornment-Tarnishment-Init}}
\label{sec:Adornment-Tarnishment-Init}

TBD

\subsection{\fr{Adornment-Tarnishment-Deinit}}
\label{sec:Adornment-Tarnishment-Deinit}

TBD

\subsection{\fr{LOCATION-INIT}}
\label{sec:LOCATION-INIT}

TBD

\subsection{\fr{Touching}}
\label{sec:Touching}

\subsection{\fr{Ingestion}}
\label{sec:Ingestion}

TBD

\subsection{\fr{Hitting}}
\label{sec:Hitting}

TBD

\subsection{\fr{LOCATION-DEINIT}}
\label{sec:LOCATION-DEINIT}

TBD

\subsection{\fr{Excretion}}
\label{sec:Excretion}

TBD

\subsection{\fr{LOCATION-CHANGE}}
\label{sec:LOCATION-CHANGE}

TBD

\subsection{\fr{Motion}}
\label{sec:Motion}

TBD

\subsection{\fr{MEANS}}
\label{sec:MEANS}

TBD

\subsection{\fr{MESSAGE}}
\label{sec:MESSAGE}

TBD

\subsection{\fr{MESSAGE-INIT}}
\label{sec:MESSAGE-INIT}

TBD

\subsection{\fr{MESSAGE-DEINIT}}
\label{sec:MESSAGE-DEINIT}

TBD

\subsection{\fr{PART-WHOLE}}
\label{sec:PART-WHOLE}

TBD

\subsection{\fr{POSSESSION}}
\label{sec:POSSESSION}

TBD

\subsection{\fr{POSSESSION-INIT}}
\label{sec:POSSESSION-INIT}

TBD

\subsection{\fr{POSSESSION-DEINIT}}
\label{sec:POSSESSION-DEINIT}

TBD

\subsection{\fr{POSSESSION-CHANGE}}
\label{sec:POSSESSION-CHANGE}

TBD

\subsection{\fr{POSSESSION-CHANGE-NECESSITY}}
\label{sec:POSSESSION-CHANGE-NECESSITY}

TBD

\subsection{\fr{POSSESSION-CONTINUATION}}
\label{sec:POSSESSION-CONTINUATION}

TBD

\subsection{\fr{QUANTITY}}
\label{sec:QUANTITY}

TBD

\subsection{\fr{SENDING}}
\label{sec:SENDING}

TBD

\subsection{\fr{SEQUENCE}}
\label{sec:SEQUENCE}

TBD

\subsection{\fr{SOCIAL-RELATION}}
\label{sec:SOCIAL-RELATION}

TBD

\subsection{\fr{SOCIAL-RELATION-INIT}}
\label{sec:SOCIAL-RELATION-INIT}

TBD

\subsection{\fr{SOCIAL-RELATION-DEINIT}}
\label{sec:SOCIAL-RELATION-DEINIT}

TBD

\subsection{\fr{TIME}}
\label{sec:TIME}

TBD

\subsection{\fr{SCENE-INIT}}
\label{sec:SCENE-INIT}

TBD

\subsection{\fr{SCENE-DEINIT}}
\label{sec:SCENE-DEINIT}

TBD

\subsection{\fr{SCENE-CONTINUATION}}
\label{sec:SCENE-CONTINUATION}

TBD

\subsection{\fr{SCENE-PREVENTION}}
\label{sec:SCENE-PREVENTION}

TBD

\subsection{\fr{SCENE-NECESSITY}}
\label{sec:SCENE-NECESSITY}

TBD

\subsection{\fr{SCENE-POSSIBILITY}}
\label{sec:SCENE-POSSIBILITY}

TBD

\subsection{\fr{NONCOMP}}
\label{sec:NONCOMP}

TBD

\section{Memos}

\subsection{Arguments Determine Frames}

The most important criterion in choosing a frame for a predicate is that there
should be suitable roles for the predicate's arguments, even if they are
unrealized in the annotated instance. For example, while \emph{drawing} denotes
a \fr{CLASS} of things, it can occur with a prepositional argument denoting a
\rl{topic}, so \fr{MESSAGE} is a better choice.

\ex.\dep{my_m-sender first_m-order *drawing#MESSAGE*}

\subsection{Prefer Core over Non-core Arguments}

When an argument fills both a core and a non-core role, it is more important to
annotate the former.

\ex.\dep{Kim_has-location *drove#LOCATION-CHANGE* to Boston_target-location}

\ex.\dep{Kim_x-causer *drove#LOCATION-CHANGE* the car_has-location to Boston_target-location}

\bibliographystyle{apalike}
\bibliography{anthology,custom}

\end{document}

\documentclass[a4paper]{article}

\usepackage[utf8]{inputenc}

\usepackage{natbib}
\usepackage{tgpagella}
\usepackage[T1]{fontenc}

\usepackage{amstext}
\usepackage{booktabs}
\usepackage{hyperref}
\usepackage{linguex}
\usepackage{mathtext}
\usepackage{relsize}
\usepackage{tikz-dependency}

\title{Superframes Manual}
\author{Kilian Evang}
\date{Last updated: \today}

% frame and role names
\newcommand{\fr}[1]{\textsf{#1}}
\newcommand{\rl}[1]{\textsf{#1}}

\begin{document}

% less white space in examples
\setlength{\Exindent}{0pt}
\setlength{\Exlabelsep}{0pt}
\setlength{\SubExleftmargin}{0pt}
\setlength{\SubSubExleftmargin}{0pt}

\maketitle

%\begin{abstract}
%\end{abstract}

\tableofcontents

\section{Introduction}

Superframes is an annotation scheme for semantic roles. Like other such
schemes, it is essentially about pinning down, in a machine-readable form,
``who did what to whom''. It is different from other such schemes, such as
FrameNet \citep{baker-etal-1998-berkeley}, VerbNet
\citep{kipper-schuler-2005-verbnet}, PropBank
\citep{palmer-etal-2005-proposition}, VerbAtlas
\citep{di-fabio-etal-2019-verbatlas}, or WiSER \citep{feng-etal-2022-widely} in
a number of ways. It aims to avoid a number of practical problems in annotating
with those schemes. Here's how Superframes annotation works, in a nutshell:

\begin{enumerate}
    \item Every content word (verb, noun, pronoun, adjective, or adverb) is a
        \emph{predicate}. Every predicate evokes one of a few dozen
        \emph{superframes}, which determines its coarse semantic class and the
        possible role labels for its arguments.
    \item The syntactic \emph{dependents} of a predicate can be
        \emph{core arguments}, in which case they get one of the role labels
        defined by the superframe of the predicate, or \emph{external
        arguments} or \emph{modifiers}, in which case they are treated as
        evoking their own frame in which the predicate serves as a core argument.
    \item There are only two main core role labels per superframe.
    \item For predicates denoting change (or lack thereof) over time,
        some superframes have \emph{aspectual variants} with role variants that
        allow to distinguish participants before, during, and after an event.
        This avoids having \texttt{Source} and \texttt{Target} as roles in
        their own right, which indicate the time sequence but suppress
        information about the nature of the relation that is changing.
        %like \texttt{-CHANGE},
        %\texttt{-INTRO}, \texttt{-EXTRO}, \texttt{-PREEMPTION},
        %\texttt{-CONTINUATION}, \texttt{-PROCESS}, \texttt{-HABIT},
        %\texttt{-NECESSITY}, and \texttt{-POSSIBILITY}
    \item Similarly, Superframes do not have the \texttt{Agent} role, which is
        often in conflict with roles indicating more specifically the agent's
        relation to other participants.
    \item Doubt, ambiguity, and figurativity are systematically treated. If there
        is not one clear solution, the solution is to give two or more
        alternative labels.
\end{enumerate}

Table~\ref{tab:superframes} shows the superframes and their roles.

\begin{table}
    \resizebox{\textwidth}{!}{
        \begin{tabular}{lllll}
            \toprule
            \fr{SCENE} & & \rl{participant} & \rl{scene} & \\
            \fr{SCENE-INIT} & & \rl{participant} & & \rl{target-scene} \\
            \fr{Creation} & & \rl{material} & & \rl{created} \\
            \fr{SCENE-DEINIT} & \rl{initial-scene} & \rl{participant} & & \\
            \fr{SCENE-CONTINUATION} & \rl{initial-scene} & \rl{participant} & & \\
            \fr{SCENE-PREVENTION} & & \rl{participant} & & \rl{target-scene} \\
            \fr{SCENE-NECESSITY} & & \rl{participant} & & \rl{target-scene} \\
            \fr{SCENE-POSSIBILITY} & & \rl{participant} & & \rl{target-scene} \\
            \midrule
            \fr{IDENTIFICATION} & & \rl{identified} & \rl{identifier} & \\
            \fr{ORDER} & & \rl{has-order} & \rl{order} & \\
            \fr{CLASS} & & \rl{has-class} & \rl{class} & \\
            \fr{SUBCLASS} & & \rl{has-subclass} & \rl{subclass} & \\
            \fr{QUALITY} & & \rl{has-quality} & \rl{quality} & \\
            \fr{STATE} & & \rl{has-state} & \rl{state} & \\
            \fr{STATE-CHANGE} & & \rl{has-state} & & \rl{target-state} \\
            \fr{EXPERIENCE} & & \rl{experiencer} & \rl{experienced} & \\
            \fr{ACTIVITY} & & \rl{is-active} & \rl{activity} & \\
            \midrule
            \fr{ACCOMPANIMENT} & & \rl{accompanied} & \rl{accompanier} & \\
            \fr{Depictive} & & \rl{has-depictive} & \rl{depictive} & \\
            \fr{ASSET} & & \rl{has-asset} & \rl{asset} & \\
            \fr{CAUSATION} & & \rl{caused} & \rl{causer} & \\
            \fr{COMPARISON} & & \rl{compared} & \rl{reference} & \\
            \fr{EXPLANATION} & & \rl{explained} & \rl{explanation} & \\
            \fr{LOCATION} & & \rl{has-location} & \rl{location} & \\
            \fr{LOCATION-INIT} & & \rl{has-location} & \rl{transitory-location} & \rl{target-location} \\
            \fr{Ingestion} & & \rl{ingested} & \rl{transitory-location} & \rl{ingester} \\
            \fr{LOCATION-DEINIT} & \rl{initial-location} & \rl{has-location} & \rl{transitory-location} & \\
            \fr{Excretion} & \rl{excreter} & \rl{excreted} & \rl{transitory-location} & \\
            \fr{LOCATION-CHANGE} & \rl{initial-location} & \rl{has-location} & \rl{transitory-location} & \rl{target-location} \\
            \fr{MEANS} & & \rl{has-means} & \rl{means} & \\
            \fr{MESSAGE} & & \rl{topic} & \rl{message} & \\
            \fr{MESSAGE-INIT} & & \rl{topic} & & \rl{target-message} \\
            \fr{MESSAGE-DEINIT} & \rl{initial-message} & \rl{topic} & & \\
            \fr{MESSAGE-HABIT} & & \rl{topic} & \rl{message} & \\
            \fr{PART-WHOLE} & & \rl{part} & \rl{whole} & \\
            \fr{POSSESSION} & & \rl{possessed} & \rl{possessor} & \\
            \fr{POSSESSION-INIT} & & \rl{possessed} & & \rl{target-possessor} \\
            \fr{POSSESSION-DEINIT} & \rl{initial-possessor} & \rl{possessed} & & \\
            \fr{POSSESSION-CHANGE} & \rl{initial-possessor} & \rl{possessed} & \rl{target-possessor} & \\
            \fr{POSSESSION-CONTINUATION} & \rl{initial-possessor} & \rl{possessed} & & \\
            \fr{QUANTITY} & & \rl{has-quantity} & \rl{quantity} & \\
            \fr{SENDING} & & \rl{sent} & \rl{sender} & \\
            \fr{SEQUENCE} & & \rl{follows} & \rl{followed} & \\
            \fr{SOCIAL-RELATION} & & \rl{has-social-relation} & \rl{social-relation} & \\
            \fr{SOCIAL-RELATION-INIT} & & \rl{has-social-relation} & & \rl{target-social-relation} \\
            \fr{SOCIAL-RELATION-DEINIT} & \rl{initial-social-relation} & \rl{has-social-relation} & & \\
            \fr{TIME} & & \rl{has-time} & \rl{time} & \\
            \midrule
            \fr{NONCOMP} & & \rl{has-noncomp} & \rl{noncomp} \\
            \bottomrule
        \end{tabular}
    }
    \caption{The superframes and their roles. TODO: what to do about processes like piloting an airplane, visiting someone, having sex with someone? Frame them as EXPERIENCEs and ACTIVITIEs done WITH someone or something? Or rather as LOCATION/SOCIAL-RELATION PROCESSES?}
    \label{tab:superframes}
\end{table}

\section{Tutorial Example 1: \fr{LOCATION}}

The superframe \fr{LOCATION} has the two main roles \rl{has-location} and
\rl{location} and indicates the location of the \rl{has-location}, typically with
respect to something, the \rl{location}. The exact spatial configuration does
not matter, it's all the same superframe.

\ex.
\dep{Kim_has-location is *sitting#LOCATION* on a chair_location}

\ex.
\dep{The board_has-location was *leaning#LOCATION* against the wall_location}

A change in location (i.e., motion) is expressed by the aspect frame
\fr{LOCATION-CHANGE}. It provides the aspect roles \rl{initial-location},
\rl{intermediate-location}, and \rl{target-location}.

\ex.
\dep{The vase_has-location *fell#LOCATION-CHANGE* to the ground_target-location}

\ex.
\dep{Kim_has-location *left#LOCATION-CHANGE* Boston_initial-location}

\ex.
\dep{Kim_has-location *went#LOCATION-CHANGE* from the kitchen_initial-location via the living room_intermediate-location to the balcony_target-location}

Motion with no clear start or end is framed as \fr{LOCATION-EVENT}.

\ex.
\dep{The jet_has-location *soared#LOCATION-EVENT* across the sky_intermediate-location}

\ex.
\dep{Kim_has-location *danced#LOCATION-EVENT* around the room_intermediate-location}

\ex.
\dep{Kim_has-location *trembled#LOCATION-EVENT*}

There are three futher aspect tags: \fr{-HABIT} for habitual states, \fr{-CONTINUATION}, and \fr{-PREVENTION}.

\ex.
\dep{Kim_has-location *lives#LOCATION* in Boston_location}

\ex.
\dep{Kim_has-location *stayed#LOCATION-CONTINUATION* in Boston_initial-location}

\ex.
\dep{The car_has-location *avoided#LOCATION-PREVENTION* the lamppost_target-location}

When a predicate is \emph{modified}, the modifier is treated as evoking another frame, of which the predicate is one argument. Thus, the syntactic dependency in this case goes from the argument to the frame instead of the other way around. To indicate this, we use the normal role label, but suffix it with \rl{-in} to indicate that the predicate fills this role \emph{in} the frame evoked by the modifier. For example, the highlighted predicates in the following examples play the \rl{has-location} role in a \fr{LOCATION} frame evoked by the modifier.

\ex.
\dep{Kim_has-social-relation is *partying#SOCIAL-EVENT* in the kitchen_has-location-in}

\ex.
\dep{a *vase#CLASS* on the table_has-location-in}

Non-core arguments, that is, arguments of predicates that do not fill a role in the main frame evoked by it, are treated much like modifiers. For example, the subject in the following example is the causer of the \fr{LOCATION-CHANGE}, so it is treated like a \fr{CAUSATION} modifier assigning the \textsf{LOCATION-CHANGE} frame the \rl{has-causer} role.

\ex.
\dep{Kim_has-causer-in *put#LOCATION-CHANGE* the vase_has-location onto the table_target-location}

Note that the frames in the previous examples for modifiers and non-core arguments are implicit, e.g., in the last example, we did not write \fr{CAUSATION} anywhere. Rather, the fact that the non-core argument evokes a \fr{CAUSATION} frame is implicitly communicated by using a role from that frame in the edge label.

\section{Tutorial Example 2: \fr{POSSESSION}}

In this section, we work through similar configurations as in the previous, but this time using the \fr{POSSESSION} superframe as an example. First, two examples of verbs indicating states of possession, illustrating that the semantic arguments can be switched around wrt. the syntactic arguments for different predicates:

\ex.
\dep{Kim_possessor *owns#POSSESSION* a house_possessed}

\ex.
\dep{The house_possessed *belongs#POSSESSION* to Kim_possessor}

Next, examples of possession change:

\ex.
\dep{Kim_target-possessor *bought#POSSESSION-CHANGE* a house_possessed}

\ex.
\dep{Kim_initial-possessor *lost#POSSESSION-CHANGE* the house_possessed}

\ex.
\dep{Kim_target-possessor *bough#POSSESSION-CHANGE* a house_possessed from Sandy_initial-possessor}

\ex.
\dep{Sandy_initial-possessor *sold#POSSESSION-CHANGE* Kim_target-possessor the house_possessed}

The predicate \emph{owe} expresses a legal necessity to transfer possession of something, giving us an opportunity to introduce the \emph{modal tag} \fr{-NECESSITY}:

\ex.
\dep{Kim_initial-possessor *owes#POSSESSION-CHANGE-NECESSITY* Sandy_target-possessor money_possessed}

Finally, an example of a \fr{POSSESSION} modifier:

\ex.
\dep{Kim_possessed-in 's *house#CLASS*}

\section{Aspect and Modality Tags}

\section{Superframes}

\section{Figurativity and Idiomaticity}

\section{Odds and Ends}

\bibliographystyle{apalike}
\bibliography{anthology,custom}

\end{document}

\documentclass[a4paper]{article}

\usepackage[utf8]{inputenc}

\usepackage{natbib}
\usepackage{tgpagella}
\usepackage[T1]{fontenc}

\usepackage{amstext}
\usepackage{booktabs}
\usepackage{hyperref}
\usepackage{linguex}
\usepackage{mathtext}
\usepackage{relsize}
\usepackage{tikz-dependency}

\title{Superframes Manual}
\author{Kilian Evang}
\date{Last updated: \today}

% frame and role names
\newcommand{\fr}[1]{\textsf{#1}}
\newcommand{\rl}[1]{\textsf{#1}}

\begin{document}

% less white space in examples
\setlength{\Exindent}{0pt}
\setlength{\Exlabelsep}{0pt}
\setlength{\SubExleftmargin}{0pt}
\setlength{\SubSubExleftmargin}{0pt}

\maketitle

%\begin{abstract}
%\end{abstract}

\tableofcontents

\section{Introduction}

\begin{table}
    \resizebox{\textwidth}{!}{
        \begin{tabular}{lllll}
            \toprule
            \fr{SCENE} & & \rl{participant} & \rl{scene} & \\
            \fr{SCENE-INIT} & & \rl{participant} & & \rl{target-scene} \\
            \fr{SCENE-DEINIT} & \rl{initial-scene} & \rl{participant} & & \\
            \fr{SCENE-CONTINUATION} & \rl{initial-scene} & \rl{participant} & & \\
            \fr{SCENE-PREVENTION} & & \rl{participant} & & \rl{target-scene} \\
            \fr{SCENE-NECESSITY} & & \rl{participant} & & \rl{target-scene} \\
            \fr{SCENE-POSSIBILITY} & & \rl{participant} & & \rl{target-scene} \\
            \midrule
            \fr{IDENTIFICATION} & & \rl{identified} & \rl{identifier} & \\
            \fr{ORDER} & & \rl{has-order} & \rl{order} & \\
            \fr{CLASS} & & \rl{has-class} & \rl{class} & \\
            \fr{Transformation-Creation} & & \rl{material} & & \rl{created} \\
            \fr{Reproduction} & & \rl{original} & & \rl{copy} \\
            \fr{SUBCLASS} & & \rl{has-subclass} & \rl{subclass} & \\
            \fr{QUALITY} & & \rl{has-quality} & \rl{quality} & \\
            \fr{STATE} & & \rl{has-state} & \rl{state} & \\
            \fr{STATE-CHANGE} & & \rl{has-state} & & \rl{target-state} \\
            \fr{Destruction} & & \rl{destroyed} & & \\
            \fr{EXPERIENCE} & & \rl{experiencer} & \rl{experienced} & \\
            \fr{ACTIVITY} & & \rl{is-active} & \rl{activity} & \\
            \fr{ASPECT} & & \rl{has-aspect} & \rl{aspect} & \\
            \midrule
            \fr{ACCOMPANIMENT} & & \rl{accompanied} & \rl{accompanier} & \\
            \fr{Depictive} & & \rl{has-depictive} & \rl{depictive} & \\
            \fr{ASSET} & & \rl{has-asset} & \rl{asset} & \\
            \fr{CAUSATION} & & \rl{caused} & \rl{causer} & \\
            \fr{Resultative} & & \rl{has-resultative} & \rl{resultative} & \\
            \fr{COMPARISON} & & \rl{compared} & \rl{reference} & \\
            \fr{Concession} & & \rl{assertion} & \rl{conceded} & \\
            \fr{EXPLANATION} & & \rl{explained} & \rl{explanation} & \\
            \fr{Purpose} & & \rl{has-purpose} & \rl{purpose} & \\
            \fr{LOCATION} & & \rl{has-location} & \rl{location} & \\
            \fr{Wrapping-Wearing} & & \rl{worn} & \rl{wearer} & \\
            \fr{Wrapping-Wearing-Init} & & \rl{worn} & & \rl{target-wearer} \\
            \fr{Wrapping-Wearing-Deinit} & \rl{initial-wearer} & \rl{worn} & & \\
            \fr{Adornment-Tarnishment} & & \rl{ornament} & \rl{surface} & \\
            \fr{Adornment-Tarnishment-Init} & & \rl{ornament} & & \rl{target-surface} \\
            \fr{Adornment-Tarnishment-Deinit} & \rl{initial-surface} & \rl{ornament} & & \\
            %\fr{LOCATION-HABIT} & & \rl{has-location} & \rl{location} & \\
            \fr{LOCATION-INIT} & & \rl{has-location} & \rl{transitory-location} & \rl{target-location} \\
            \fr{Ingestion} & & \rl{ingested} & \rl{transitory-location} & \rl{ingester} \\
            \fr{Hitting} & & \rl{hitting} & & \rl{hit} \\
            \fr{LOCATION-DEINIT} & \rl{initial-location} & \rl{has-location} & \rl{transitory-location} & \\
            \fr{Excretion} & \rl{excreter} & \rl{excreted} & \rl{transitory-location} & \\
            \fr{LOCATION-CHANGE} & \rl{initial-location} & \rl{has-location} & \rl{transitory-location} & \rl{target-location} \\
            \fr{Motion} & & \rl{has-location} & \rl{transitory-location} & \\
            \fr{MEANS} & & \rl{has-means} & \rl{means} & \\
            \fr{MESSAGE} & & \rl{topic} & \rl{content} & \\
            \fr{MESSAGE-INIT} & & \rl{topic} & & \rl{target-message} \\
            \fr{MESSAGE-DEINIT} & \rl{initial-message} & \rl{topic} & & \\
            %\fr{MESSAGE-HABIT} & & \rl{topic} & \rl{message} & \\
            \fr{PART-WHOLE} & & \rl{part} & \rl{whole} & \\
            \fr{POSSESSION} & & \rl{possessed} & \rl{possessor} & \\
            \fr{POSSESSION-INIT} & & \rl{possessed} & & \rl{target-possessor} \\
            \fr{POSSESSION-DEINIT} & \rl{initial-possessor} & \rl{possessed} & & \\
            \fr{POSSESSION-CHANGE} & \rl{initial-possessor} & \rl{possessed} & \rl{target-possessor} & \\
            \fr{POSSESSION-CHANGE-NECESSITY} & \rl{initial-possessor} & \rl{possessed} & \rl{target-possessor} & \\
            \fr{POSSESSION-CONTINUATION} & \rl{initial-possessor} & \rl{possessed} & & \\
            \fr{QUANTITY} & & \rl{has-quantity} & \rl{quantity} & \\
            \fr{SENDING} & & \rl{sent} & \rl{sender} & \\
            \fr{SEQUENCE} & & \rl{follows} & \rl{followed} & \\
            \fr{SOCIAL-RELATION} & & \rl{has-social-relation} & \rl{social-relation} & \\
            \fr{SOCIAL-RELATION-INIT} & & \rl{has-social-relation} & & \rl{target-social-relation} \\
            \fr{SOCIAL-RELATION-DEINIT} & \rl{initial-social-relation} & \rl{has-social-relation} & & \\
            \fr{TIME} & & \rl{has-time} & \rl{time} & \\
            \midrule
            \fr{NONCOMP} & & \rl{has-noncomp} & \rl{noncomp} \\
            \bottomrule
        \end{tabular}
    }
    \caption{The superframes and their roles.}
    \label{tab:superframes}
\end{table}

Superframes is an annotation scheme for semantic roles. Like other such
schemes, it is essentially about pinning down, in a machine-readable form,
``who did what to whom''. It is different from other such schemes, such as
FrameNet \citep{baker-etal-1998-berkeley}, VerbNet
\citep{kipper-schuler-2005-verbnet}, PropBank
\citep{palmer-etal-2005-proposition}, VerbAtlas
\citep{di-fabio-etal-2019-verbatlas}, or WiSER \citep{feng-etal-2022-widely} in
a number of ways. It aims to avoid a number of practical problems in annotating
with those schemes. Here's how Superframes annotation works, in a nutshell:

\begin{enumerate}
    \item Every content word (verb, noun, pronoun, adjective, or adverb) is a
        \emph{predicate}. Every predicate evokes one of a few dozen
        \emph{superframes}, which determines its coarse semantic class and the
        possible role labels for its arguments.
    \item The syntactic \emph{dependents} of a predicate can be
        \emph{core arguments}, in which case they get one of the role labels
        defined by the superframe of the predicate, or \emph{external
        arguments} or \emph{modifiers}, in which case they are treated as
        evoking their own frame in which the predicate serves as a core argument.
    \item There are only two main core role labels per superframe.
    \item For predicates denoting change (or lack thereof) over time,
        some superframes have \emph{aspectual variants} with role variants that
        allow to distinguish participants before, during, and after an event.
        This avoids having \texttt{Source} and \texttt{Target} as roles in
        their own right, which indicate the time sequence but suppress
        information about the nature of the relation that is changing.
        %like \texttt{-CHANGE},
        %\texttt{-INTRO}, \texttt{-EXTRO}, \texttt{-PREEMPTION},
        %\texttt{-CONTINUATION}, \texttt{-PROCESS}, \texttt{-HABIT},
        %\texttt{-NECESSITY}, and \texttt{-POSSIBILITY}
    \item Similarly, Superframes do not have the \texttt{Agent} role, which is
        often in conflict with roles indicating more specifically the agent's
        relation to other participants.
    \item Doubt, ambiguity, and figurativity are systematically treated. If there
        is not one clear solution, the solution is to give two or more
        alternative labels.
\end{enumerate}

Table~\ref{tab:superframes} shows the superframes and their roles.

\subsection{Core Arguments}

The most prototypical predicate is a verb, and the simplest case is a verb with only one argument. It can for example denote an experience or an activity:

\ex.\dep{Kim_experiencer is *sleeping#EXPERIENCE*}

\ex.\dep{Kim_is-active is *partying#ACTIVITY*}

With two core arguments, a verb denotes a relation that holds between them:

\ex.\dep{Kim_possessor *owns#POSSESSION* a house_possession}

\ex.\dep{The house_possession *belongs#POSSESSION* to Kim_possessor}

\ex.\dep{Kim_topic *seems#MESSAGE* happy_content}

\subsection{Aspect}

Rather than a static relationship between two entities, many verbs (and other
predicates) denote a change (or absence of change) in such a relationship. We
sort such predicates into a few coarse aspectual classes. For example,
initiation (\fr{-INIT}) means a state is begun or worked towards, deinitiation
(\fr{-DEINIT}) means a state is ended, completed, or its end is worked towards,
change (\fr{-CHANGE}) combines both, where one state is replaced by another,
and continuation (\fr{-CONT}) means a state persists or is even intensified.
Accordingly, roles with \rl{target-}, \rl{initial-}, or \rl{transitory-} mark
participants at/beyond the end of, at the beginning of, or at some point during
the event, respectively.

\ex.\dep{Kim_target-possessor *got#POSSESSION-INIT* the house_possession}

\ex.\dep{Kim_initial-possessor *lost#POSSESSION-DEINIT* the house_posesssion}

\ex.\dep{Kim_initial-possessor *sold#POSSESSION-CHANGE* the house_possession to Sandy_target-possessor}

\ex.\dep{Kim_initial-possessor *kept#POSSESSION-CONT* the house_possession}

\ex.\dep{Kim_has-location *went#LOCATION-CHANGE* from Chicago_initial-location via Pittsburgh_transitory-location to Boston_target-location}

\ex.\label{ex:fall}\dep{The vase_has-location *fell#LOCATION-CHANGE* to the ground_target-location}

\ex.\dep{The vase_has-state *broke#STATE-CHANGE*}

\ex.\dep{Kim_has-social-relation *befriended#SOCIAL-RELATION-INIT* Sandy_target-social-relation}

\ex.\dep{Kim_has-social-relation *married#SOCIAL-RELATION-INIT* Sandy_target-social-relation}

\ex.\dep{Kim_has-social-relation *divorced#SOCIAL-RELATION-DEINIT* Sandy_initial-social-relation}

\subsection{Non-core Arguments}

Core arguments always get role labels from the superframe the predicate evokes.
But many verbs have more arguments. One common case is a subject that is
presented as the causer of the scene. For example, compare \ref{ex:throw} with
\ref{ex:fall}. The core scene is the same (same superframe, same arguments). We
now assume there is an additional \fr{CAUSATION} scene with \emph{Kim} as the
\rl{causer} and the core scene as the \rl{caused}. We denote this by giving
\emph{Kim} the \rl{caused} role label, with an \rl{x-} prefix to mark it as a
non-core role.

\ex.\label{ex:throw}\dep{Kim_x-causer *threw#LOCATION-CHANGE* the vase_has-location to the ground_target-location}

\ex.\dep{Kim_x-causer *broke#STATE-CHANGE* the vase_has-state}

Two other common non-core arguments are the senders and recipients (experiencers) of messages.

\ex.\dep{Kim_x-sender *talked#MESSAGE* to Sandy_x-experiencer about Bali_topic}

Other non-core arguments are usually rather predicate-specific.

\ex.\dep{Kim_x-experiencer *searched#MESSAGE* the woods_x-location for Sandy_topic}

\ex.\dep{Kim_initial-possessor *sold#POSSESSION-CHANGE* Sandy_target-possessor the house_possession for a million dollars_x-asset}

\subsection{Modifiers}

Like non-core arguments, modifiers are assumed to evoke an additional frame,
and labeled with the role they fill in that frame, but with a prefix marking
them as modifiers: \rl{m-}.

\ex.\dep{Kim_excreter is *sweating#EXCRETION* profusely_m-quantity in the sauna_m-location}

\subsection{Nonverbal Predicates}

So far, we have only looked at verbal predicates. But of course, there are other types of predicates. An ordinary noun like \emph{tree} evokes the \fr{CLASS} frame, marking the entity it refers to as being a member of a class (in this case: the class of trees). There are no arguments here because the predicate itself doubles as a referent. However, the predicate can of course be modified:

\ex.\dep{a *tree#CLASS* in the garden_m-location}

\ex.\dep{Kim_m-possessor 's *tree#CLASS*}

Event nouns evoke event frames and have arguments:

\ex.\dep{Kim_x-causer 's *breaking#STATE-CHANGE* of the vase_has-state}

Relational nouns evoke relational frames and have arguments:

\ex.\dep{Kim_has-social-relation 's *friend#SOCIAL-RELATION*}

Pronouns and names evoke the \fr{IDENTIFICATION} frame, meaning that they
identify their referent as soe entity (via naming or anaphora resolution).

\ex.\dep{*Kim#IDENTIFICATION*}

\ex.\dep{*they#IDENTIFICATION*}

Predicate adjectives most typically denote states or qualities.

\ex.\dep{I_has-quality am *despicable#QUALITY*}

\ex.\dep{the dog_has-state is *tired#STATE*}

With attributive adjectives, the dependency relation is reversed, and the role label is changed accordingly.

\ex.\dep{despicable_m-quality *me#IDENTIFICATION*}

\ex.\dep{the tired_m-state *dog#CLASS*}

Similarly for adverbs denoting, e.g, manner (\rl{quality}) or extent (\rl{quantity}):

\ex.\dep{Kim_has-location *ran#Motion* fast_m-quality}

\ex.\dep{Kim_has-location *ran#Motion* far_m-quantity}

\subsection{Figurativity and Idiomaticity}

Difficulties in choosing frames often arise because predicate literally evokes
one frame, but is used in a way that perhaps fits another frame equally well or
better. In such cases, annotate both the more literal frame and roles, followed
by the \texttt{>}\texttt{>} operator, followed by the more figurative frame and roles.

\ex.\dep{primeval_m-time>>m-subclass *forest#CLASS*}

\ex.\dep{colored_m-quality>>m-subclass *pencil#CLASS*}

\ex.\dep{to *lay#LOCATION-CHANGE>>MESSAGE-DEINIT* aside_target-location>>x-noncomp my drawings_has-location>>topic}

\section{Superframes Reference}

\section{Memos}

\subsection{Prefer Core over Non-core Arguments}

\bibliographystyle{apalike}
\bibliography{anthology,custom}

\end{document}
